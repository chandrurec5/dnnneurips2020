\section{Deep Gated Network}\label{sec:pathgate}
\begin{comment}
\begin{table}[!htb]
\begin{minipage}{0.5\columnwidth}
\resizebox{1.0\columnwidth}{!}{
\begin{tabular}{|l|}\hline
\multicolumn{1}{|c|}{Path Quantities}\\\hline 
$\I_0\colon [P]\ra [d_{in}]$, $\I_{l}\colon [P]\ra [w], l\in[d-1]$\\\hline
$v_t(p)\stackrel{def}=\Pi_{l=1}^d \Theta_t(l,\I_{l-1}(p),\I_l(p))$ \\\hline
$A_{t}(x,p)\stackrel{def}{=}\Pi_{l=1}^{d-1} G_{x,t}(l,\I_l(p))$\\\hline
\end{tabular}
}
\end{minipage}
\begin{minipage}{0.5\columnwidth}
%\resizebox{\columnwidth}{!}{
%\begin{tabular}{|l|l|}\hline
%\multicolumn{2}{|c|}{Deep Gated Network}\\\hline 
%Input layer & $z_{x,\Theta_t}(0)=x$ \\\hline
%Pre-activation & $q_{x,\Theta_t}(l)={\Theta_t(l)}^\top z_{x,\Theta_t}(l-1)$\\\hline
%Layer output & $z_{x,\Theta_t}(l)=q_{x,\Theta_t}(l)\odot G_{x,t}(l)$ \\\hline
%Final output & $\hat{y}_t(x)={\Theta_t(d)}^\top z_{x,\Theta_t}(d-1)$\\\hline
%\end{tabular}
%}
\resizebox{\columnwidth}{!}{
\begin{tabular}{|c|c|c|}\hline
&Gating Network & Weight Network\\\hline
Input Layer&$z_{x,\Tg_t}(0)=x$ & $z_{x,\Tv_t}(0)=x$ \\\hline
Pre-activation& $q_{x,\Tg_t}(l)={\Tg_t(l)}^\top z_{x,\Tg_t}(l-1)$ & $q_{x,\Tv_t}(l)={\Tv_t(l)}^\top z_{x,\Tv_t}(l-1)$\\\hline
Layer Output& $z_{x,\Tg_t}(l)=q_{x,\Tg_t}(l)\odot G_{x,\Tg}(l)$ & $z_{x,\Tv_t}(l)=q_{x,\Tv_t}(l)\odot G_{x,\Tg_t}(l)$ \\\hline
Gating& \multicolumn{2}{|c|}{$\hat{y}_{t}(x)={\Tv_t(d)}^\top z_{x,\Tv_t}(d-1)$}\\\hline 
 &\multicolumn{2}{|c|}{$\begin{aligned}\beta >0: G_{x,\Tg_t}(l,i)&=\chi_{\epsilon}(-\beta q_{x,\Tg_t}(l,i)), \\ \beta=\infty: G_{x,\Tg_t}(l,i)&=\mathbbm{1}_{\{q_{x,\Tg_t}(l,i)>0\}}\end{aligned}$}\\\hline 
\end{tabular}
}
\end{minipage}
\caption{Here $\Theta(l,i,j)$ denotes the weight connecting node $i$ of layer $l-1$ to node $j$ of layer $l$, and $\odot$ stands for the \emph{Hadamard} product.}
\label{tb:dgn-path}
\end{table}


We consider a dataset given by $(x_s,y_s)_{s=1}^n\in \R^{d_{in}}\times \R$.
\begin{table}[!htb]
\begin{tabular}{|c|c|}\hline
Gating Network & Value Network\\\hline
$z_{x,\Tg_t}(0)=x$ & $z_{x,\Tv_t}(0)=x$ \\\hline
$q_{x,\Tg_t}(l)={\Tg_t(l)}^\top z_{x,\Tg_t}(l-1)$ & $q_{x,\Tv_t}(l)={\Tv_t(l)}^\top z_{x,\Tv_t}(l-1)$\\\hline
$z_{x,\Tg_t}(l)=q_{x,\Tg_t}(l)\odot G_{x,\Tg}(l)$ & $z_{x,\Tv_t}(l)=q_{x,\Tv_t}(l)\odot G_{x,\Tg_t}(l)$ \\\hline
 \multicolumn{2}{|c|}{$\hat{y}_{t}(x)={\Tv_t(d)}^\top z_{x,\Tv_t}(d-1)$}\\\hline 
\multicolumn{2}{|c|}{$\begin{aligned}\beta >0: G_{x,\Tg_t}(l,i)&=\chi_{\epsilon}(-\beta q_{x,\Tg_t}(l,i)), \\ \beta=\infty: G_{x,\Tg_t}(l,i)&=\mathbbm{1}_{\{q_{x,\Tg_t}(l,i)>0\}}\end{aligned}$}\\\hline 
\end{tabular}
\caption{Shows a parameterised deep gated network. Here, $\epsilon>0$ is a hyper-parameter.}
\label{tb:dgn}
\end{table}
\end{comment}

\begin{table}[!htb]
\begin{minipage}{0.5\columnwidth}
\resizebox{\columnwidth}{!}{
\begin{tabular}{|l|l|}\hline
\multicolumn{2}{|c|}{Deep Gated Network: $\N(\Theta_t;\G_t)$}\\\hline 
Input layer & $z_{x,\Theta_t}(0)=x$ \\\hline
Pre-activation & $q_{x,\Theta_t}(l)={\Theta_t(l)}^\top z_{x,\Theta_t}(l-1)$\\\hline
Layer output & $z_{x,\Theta_t}(l)=q_{x,\Theta_t}(l)\odot G_{x,t}(l)$ \\\hline
Final output & $\hat{y}_t(x)={\Theta_t(d)}^\top z_{x,\Theta_t}(d-1)$\\\hline
Gating Values& $\begin{aligned}\G_t\stackrel{def}=\{G_{x_{s},t}(l,i), \forall s\in[n],\\l\in[d-1],i\in[w]\}\end{aligned}$\\\hline
\end{tabular}
}
\end{minipage}
\begin{minipage}{0.5\columnwidth}
\resizebox{\columnwidth}{!}{
\begin{tabular}{|c|c|}\hline
Gating Network: $\G(\Tg_t,\beta)$\\\hline
$z_{x,\Tg_t}(0)=x$  \\\hline
$q_{x,\Tg_t}(l)={\Tg_t(l)}^\top z_{x,\Tg_t}(l-1)$ \\\hline
$z_{x,\Tg_t}(l)=q_{x,\Tg_t}(l)\odot G_{x,\Tg_t}(l)$ \\\hline
{$\begin{aligned}\beta >0: G_{x,\Tg_t}(l,i)&=\frac{1+\epsilon}{1+\exp(-\beta q_{x,\Tg_t}(l,i))} \\ \beta=\infty: G_{x,\Tg_t}(l,i)&=\mathbbm{1}_{\{q_{x,\Tg_t}(l,i)>0\}}\end{aligned}$}\\\hline 
\end{tabular}
}
\end{minipage}
\caption{Here $\Theta(l,i,j)$ denotes the weight connecting node $i$ of layer $l-1$ to node $j$ of layer $l$, and $\odot$ stands for the \emph{Hadamard} product. The left table shows a DGN, and the right table shows a gating network. }
\label{tb:dgn}
\end{table}
We consider a dataset given by $(x_s,y_s)_{s=1}^n\in \R^{d_{in}}\times \R$, and fully connected deep gated networks with $d$ layers (i.e., depth $=d$), and $w$ hidden units per layer (i.e., width $=w$). In \Cref{tb:dgn}, $q(l),z(l)$ and $G(l)$ are $w$-dimensional quantities, where $G(l)$ are the gating values.  At at time $t$, by specifying the gating values for all the $n$ input examples as $\G_t\stackrel{def}=\{G_{x_{s},t}(l,i), \forall s\in[n],l\in[d-1],i\in[w]\}$, we can recover the outputs $\hat{y}_t(x_s)\in \R$ for all the inputs $\{x_s\}_{s=1}^n$ in the dataset.  A DGN is denoted by $\N(\Theta_t;\G_t)$, i.e., the weights and the gating values are decoupled, and need to be specified to obtain a complete description. This enables the study of following networks:\\
%$1.$ \emph{Fixed Random:} Here, for each of the $n$ input examples in the dataset, we sample the $w(d-1)$ corresponding gating values at random from a Bernoulli distribution with parameter $\mu$. Once sampled, these gating values are collected in $\G_0$ and can be held constant ($\G_t=\G_0,\forall t\geq 0$) throughout training. It is obvious that a network with this gating method cannot generalise. However, this gating method is useful in studying the optimisation question.\\
$1.$ \emph{Soft:} Here, the gating values belong to $(0,1+\epsilon)$ ($\epsilon>0$ is a hyper-parameter). As seen in the bottom of the table on the right-hand side in \Cref{tb:dgn}, the soft gate replaces the hard indicator in the ReLU by a sigmodial function. A desirable feature here is that the gating values are differentiable with respect to the underlying parameter. We often use the prefix soft to denote the use of soft-gates.\\
$2.$ \emph{Gated Linear Unit (GaLU):} Here, we have a separate gating network parameterised by $\Tg\inrdnet$, which, provides the gating values denoted by $\G(\Tg_t,\beta)$. We refer the network $\N(\Theta_t;\G(\Tg,\beta))$ as soft-GaLU network and .the network $\N(\Theta_t;\G(\Tg,\infty))$ as GaLU network.\\
$3.$ \emph{ReLU:} Here, the gating values are derived implicitly from $\Theta_t$. We refer the network $\N(\Theta_t;\G(\Theta_t,\beta))$ as soft-ReLU network and the network $\N(\Theta_t;\G(\Theta_t,\infty))$ as ReLU network.

