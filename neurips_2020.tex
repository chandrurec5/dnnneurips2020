\documentclass{article}

% if you need to pass options to natbib, use, e.g.:
%     \PassOptionsToPackage{numbers, compress}{natbib}
% before loading neurips_2020

% ready for submission
% \usepackage{neurips_2020}

% to compile a preprint version, e.g., for submission to arXiv, add add the
% [preprint] option:
%     \usepackage[preprint]{neurips_2020}

% to compile a camera-ready version, add the [final] option, e.g.:
%     \usepackage[final]{neurips_2020}

% to avoid loading the natbib package, add option nonatbib:
\usepackage[preprint]{neurips_2020}
\usepackage[utf8]{inputenc} % allow utf-8 input
\usepackage[T1]{fontenc}    % use 8-bit T1 fonts
\usepackage{hyperref}       % hyperlinks
\usepackage{url}            % simple URL typesetting
\usepackage{booktabs}       % professional-quality tables
\usepackage{amsfonts}       % blackboard math symbols
\usepackage{nicefrac}       % compact symbols for 1/2, etc.
\usepackage{microtype}      % microtypography
\usepackage{bbm}
\input{pack.tex}
\usepackage{hyperref}
\usepackage[capitalize]{cleveref}
\usepackage{caption}
\captionsetup{belowskip=0pt}
\usepackage{wrapfig,lipsum}
\usepackage[linewidth=1.2pt,linecolor=red]{mdframed}


\usepackage{comment}
\usepackage{cancel}
\usepackage{changepage}
\usepackage{bbm}
\usepackage{pgfplots}
\usepackage{filecontents}
\setcounter{MaxMatrixCols}{32}
\usepackage{placeins}


\title{Deep Learning: A study in paths and gates}

% The \author macro works with any number of authors. There are two commands
% used to separate the names and addresses of multiple authors: \And and \AND.
%
% Using \And between authors leaves it to LaTeX to determine where to break the
% lines. Using \AND forces a line break at that point. So, if LaTeX puts 3 of 4
% authors names on the first line, and the last on the second line, try using
% \AND instead of \And before the third author name.

\author{Chandrashekar Lakshminarayanan and Amit Vikram Singh, \\ Indian Institute of Technology, Palakkad}

\begin{document}

\maketitle
\begin{abstract}
We study optimisation and generalisation in deep neural networks (DNNs) with ReLU activations. Central idea in the paper is to regard paths and gates as basic building blocks of a deep neural network. At time $t$, for an input $x\in\R^{d_{in}}$ to the DNN, we define a novel \emph{neural path feature} $\phi_{x,t}$ whose dimension is equal to total the number of paths, and whose co-ordinate corresponding to path is the product of the signal at the input node and the activity\footnote{A path is active if all the gates in the path are active.} of the path. We show that the associated \emph{neural path kernel} $H_t(x,x')=\phi^\top_{x,t}\phi_{x',t}$, plays an important role in optimisation and generalisation. In particular, $H_t(x,x')=(x^\top x')\lambda_t(x,x')$, where $\lambda_t(x,x')$ is a count of total number of paths that are active simultaneously for both inputs $x,x'\in \R^{d_{in}}$. 

When the activations and as a consequence the neural path features are frozen (i.e. do not change over time), and under symmetric Bernoulli initialisation, we show (in theory and experiments) that i) increasing depth till a point helps in training and ii) increasing depth beyond hurts training. We verify via experiments that the NPFs and NPK are learnt during training by showing that the norm of the labelling function measured with respect to the inverse of the trace normalised NPK reduces with time.

\end{abstract}
\section{Introduction}
Understanding optimisation and generalisation of deep neural networks (DNNs) trained using first-order method such as (stochastic) gradient descent is an important problem in machine learning. In this paper, we throw light on the following two questions:%Despite the fact that the optimisation of DNNs is a non-convex problem in general, simple (stochastic) gradient descent (SGD) type procedures are able to achieve zero training error when the DNN is sufficiently over-parameterised and when the parameters are initialised at random. Further, it has also been empirically observed that, practical DNNs can achieve zero training error on standard datasets even when the labels are flipped at random. 
\begin{comment}
Recent works based on the \emph{trajectory} analysis \cite{} have shown that, sufficiently over-parameterised DNNs can be optimised to achieve zero training error by a randomly initialised SGD procedure. The gist of the \emph{trajectory} based analysis is the following: consider the dataset given by $(x_s,y_s)_{i=1}^n\in \R^{d_{in}\times} \R$, and for an input $x_s\in \R^{d_{in}},i\in[n]$\footnote{We denote the set $\{1,\ldots, n\}$ by $[n]$.}, let $\hat{y}_{\Theta_t}(x_s)$ be predicted output of the DNN whose parameters/tunable weights at time $t$ is $\Theta_t\in \R^{d_{net}}$. Say one is interested in minimising the squared loss given by $L_{\Theta_t}=\frac{1}{2}\sum_{i=1}^n\left(\hat{y}_{\Theta_t}(x_s)-y_s\right)^2$ by a SGD procedure, then the idea behind the trajectory analysis is to look at the dynamics of the error term defined as $e_t(i)\stackrel{def}{=}\hat{y}_{\Theta_t}(x_s) -y_s$. Denoting $e_t=(e_t(i),i\in[n])\in \R^n$, one can study the following error recursion:
\begin{align}\label{eq:trajecbasic}
e_{t+1}=e_t-\alpha_t K_t e_t,
\end{align}
where $\alpha_t>0$ is a small stepsize, $K_t\in \R^{n\times n}$ is a Gram matrix. This gram matrix is in turn expressed as $K_t=\Psi_t^\top \Psi_t$, wherein, $\Psi_t$ is a $d_{net}\times n$ matrix known as the \emph{neural tangent feature} (NTF) matrix, which is the collection of the gradient of the network output with respect to the network parameters, and whose $s^{th}$ column is given by $\Psi_t(s)=(\frac{\partial \hat{y}_{\Theta_t}(x_s)}{\partial \theta},\theta \in \Theta)$.
\end{comment}

\textbf{Question I (Optimisation):} \emph{What is the role of depth in training of DNNs? Why increasing depth till a point helps in training? Why increasing the depth beyond hurts training?}\\
\begin{comment}We call the above  questions are the depth phenomena. \cite{dudnn} show that, when it comes to training, residual networks are better than simple FC-DNNs. However, the depth phenomena in the case of simple FC-DNNs is still unresolved.
\end{comment}
\textbf{Question II (Generalisation):} \emph{What are the hidden features in a DNN? Are these features learned? and if so, how?}
\begin{comment}The general consensus is that, the DNNs learn hidden representations progressively in each of the intermediate layers, and the final layer learns a linear model using features obtained in the penultimate layer.  This view, while conceptually simple, however, does not provide us any analytical insight regarding the above question.  A more analytically appealing candidate for the hidden representation (used in some of the recent works \cite{}) is the \emph{neural tangent random feature} (NTRF) which is the NTF evaluated at randomised initialisation of an infinitely wide DNN. \cite{} provides generalisation bounds in terms of the Rademacher complexity of the class of functions defined by the NTRF and also in terms of an associated neural tangent kernel (NTK). \cite{} uses NTK to set a significant new benchmark for pure-kernel based learning.  An issue with the NTRF/NTK approach is that the features do not change over the training of the DNN, thus implying no feature learning is happening, and yet experimental evidence (in \cite{} as well as \Cref{sec:generalisation-exp}) shows that DNNs perform significantly better than pure-kernel learning with NTK.
\end{comment}

The results in this paper fit within the the framework of the trajectory method, used in recent works \cite{} to show that SGD achieves zero training error in over-parameterised DNNs. In particular, the error trajectory can be given by $e_{t+1}=e_t-\alpha_tK_te_t$, where $e_t\in \R^n$ is the error in the DNN prediction (i.e., difference between predicted and true values), $\alpha_t$ is a small stepsize (of the SGD) and $K_t\in\R^{n\times n}$ is a Gram matrix. 


\textbf{Highlights Of Our Contributions:}
%In this paper, we are interested in fully connected (FC) deep gated networks (DGN), wherein, the output of a neuron can be expressed as a product of its pre-activation input and its gating value. DNNs with ReLU activations (in this paper, we refer to them as simply DNNs) are special cases of DGNs. 
%We consider networks with $d$ layers, and $w$ hidden neurons per layer.
 %We now describe in brief the conceptual novelties (CN), the corresponding conceptual gains/insights (CG) and the key results in the paper.

$\bullet$ \textbf{Paths:}  A \emph{path} starts from an input node, passes through exactly one weight and one activation in each layer, and finally ends at the output node. Let $[P]=\{1,\ldots,P\}$ be an enumerate of all the paths, the zeroth and first order quantities of a DNN can then be expressed as:
\begin{align*}
%\begin{split}
\text{(Zeroth Order)}\quad: \quad \hat{y}_t(x)&=\sum_{p\in [P]}x(\I(p))A_t(x,p)v_t(p)={\phi^\top_{x,t}} v_t\\
\text{(First Order)}\quad: \quad \partial \hat{y}_t(x)&=\sum_{p\in [P]}x(\I(p)) \partial(A_t(x,p)) v_t(p)+ \sum_{p\in [P]}x(\I(p)) A_t(x,p) \partial((v_t(p))\\
&=\phi^\top_{x_s,t} {\partial} v_t + {\partial} \phi^\top_{x_s,t} v_t,
%\end{split}
\end{align*}
where $\I(p)$ is the input node at which a path $p$ starts, $v_t(p)$ is its value given by the product of its weights and $A_t(x,p)$ is its activation level given by the product of the gating values in the path.

$\bullet$ \textbf{Path Feature and Kernel:}  $\phi_{x,t}=\left( x(\I(p))A_t(x,p),p\in[P]\right)\in \R^P$ is the \emph{neural path feature} (NPF), and the \emph{neural path kernel} (NPK) given by $H_t(x,x')=\phi^\top_{x,t}\phi_{x',t}$. We show that the NPK has a special structure, in that, $H_t(x,x')=(x^\top x')\lambda_t(x,x')$, where $\lambda_t(x,x')$ is a measure of similarity based on the path activation levels for inputs $x,x'\in\R^{d_{in}}$. %For instance, in the case of DNNs with ReLU, $\lambda_t(s,s')$ is a measure of the sub-networks that are simultaneously \emph{on} for both the inputs, capturing the intuition that if $s$ and $s'$ are similar, then their corresponding active sub-networks might also be similar. 

$\bullet$ \textbf{Optimisation:} We note that, in $\phi_{x,t}/H_t$ only $A_t/\lambda_t$ changes during training. %We first study the frozen gating regime, wherein, the activations and hence $\lambda_t$ do not change with time.
We first study \emph{gated linear unit} (GaLU) networks with frozen gates, wherein, the activations $A_t(\cdot,p)=A_0(\cdot,p),\forall t\geq 0,p\in[P]$ do not change with time. In GaLU networks, the gating values for each input are copied from a different network called gating network (whose weights are held constant during training). Thus the NPF as well as the NPK do not change with time, i.e., $\phi_{\cdot,t}=\phi_{\cdot,0}, H_t=H_0,\forall t\geq 0$ . 
For GaLU networks, when the weights are sampled $\stackrel{iid}\sim Ber(\frac{1}{2})$ over the set $\{-\sigma,+\sigma\}$, we have the following results that throw light on Question I: 

$1.$ $\E{K_0(s,s')}=d\sigma^{2(d-1)}H_0(s,s')=d\sigma^{2(d-1)}(x_s^\top x_{s'})\lambda_0(s,s')$. We argue that the non-diagonal entries of $\lambda_0$ decay at an exponential rate (in comparison with the diagonal entries), i.e., the Gram matrix whitens with depth.
%The ratio $\frac{{\lambda_0}(s,s')}{{\lambda_0}(s,s)}$ is the fractional overlap of active sub-networks, say at each layer the overlap of active gates is $\mu\in (0,1)$, then for a depth $d$, the fractional overlap decays at exponential rate, i.e., $\frac{{\lambda_0}(s,s')}{{\lambda_0}(s,s)}\leq \mu^d$, leading to whitening. Thus increasing depth till a point helps in training.

$2.$ $Var\left[K_0(s,s')\right]\leq O(\max\{\frac{d^2}{w}, \frac{d^3}{w^2}\})$ (for $\sigma^2=O(\frac{1}w)$). For a fixed $d$, increasing $w$ ensures $K_0$ converges to $\E{K_0}$. However, for a fixed $w$, increasing $d$ makes the entries of $K_0$ deviate from $\E{K_0}$, thus degrading the spectrum of $K_0$. Thus increasing depth beyond a point hurts training performance.

\begin{comment}

\textbf{ReLU artefact or NTF vs NPF:} Note that the left-hand side of \eqref{eq:npfgrad} is the neural tangent feature. We note that, in the case when $A_t(x,p)\in\{0,1\}$ (such as in DNN with ReLU activations)  $\partial_{\phi}=0$, and hence is not accounted for in the SGD update as well as the analysis. However, due to the SGD update, the gating value changes during training, and consequently the activations, the NPF, the NPK change during training (see the experiments in \Cref{sec:generalisation}). We believe the change of activations is key for generalisation, a fact which can explain the gap in generalisation performance \cite{} between the DNNs and the pure-kernel methods based on their static NTK counterparts.
\end{comment}
\begin{comment}
$\bullet$ \textbf{Separation of Gates:} In order to better understand the role of $A_t(\cdot,\cdot)$ and $\partial_{\phi}$ in the optimisation and generalisation of DNNs, we handle the gating values as independent variables. This way we consider networks with i) fixed random gates, wherein, the gating values for the $n$ input examples are generated at random at $t=0$, and held fixed throughout training, ii) gated linear units (GaLU), wherein, the gates are copied from a different network (called gating network) with identical architecture, and iii) soft-ReLU gates, wherein, the hard-max in the ReLU is replaced by a soft-max sigmoidal function.
%In our theory as well as experiments, we study the following useful gating regimes (GR): GR-(i)(\emph{frozen}), wherein, the activations $A_t(\cdot,\cdot)$ do not change with time, GR-(ii) (\emph{decoupled}), wherein, $A_t(\cdot,\cdot)$ is derived from a separate \emph{gating} network parameterised by $\Tg\in \R^{d_{net}}$,  GR-(iii) (\emph{adaptable}), wherein, $A_t(\cdot,\cdot)$ changes during training (this is the case of the standard DNNs with ReLU gates), and GR-(iv) (\emph{soft}), wherein, the hard-max indicator function of the ReLU is replaced by a soft-max sigmoidal function.

$\bullet$ \textbf{Optimisation (Key Results I):} For GaLU networks, when the weights are sampled $\stackrel{iid}\sim Ber(\frac{1}{2})$ over the set $\{-\sigma,+\sigma\}$, we have the following results that throw light on Question I: 

$1.$ $\E{K_0(s,s')}=d\sigma^{2(d-1)}H_0(s,s')=d\sigma^{2(d-1)}(x_s^\top x_{s'})\lambda_0(s,s')$. We argue that the non-diagonal entries of $\lambda_0$ decay at an exponential rate (in comparison with the diagonal entries), i.e., the Gram matrix whitens with depth.
%The ratio $\frac{{\lambda_0}(s,s')}{{\lambda_0}(s,s)}$ is the fractional overlap of active sub-networks, say at each layer the overlap of active gates is $\mu\in (0,1)$, then for a depth $d$, the fractional overlap decays at exponential rate, i.e., $\frac{{\lambda_0}(s,s')}{{\lambda_0}(s,s)}\leq \mu^d$, leading to whitening. Thus increasing depth till a point helps in training.

$2.$ $Var\left[K_0(s,s')\right]\leq O(\max\{\frac{d^2}{w}, \frac{d^3}{w^2}\})$ (for $\sigma^2=O(\frac{1}w)$). For a fixed $d$, increasing $w$ ensures $K_0$ converges to $\E{K_0}$. However, for a fixed $w$, increasing $d$ makes the entries of $K_0$ deviate from $\E{K_0}$, thus degrading the spectrum of $K_0$. Thus increasing depth beyond a point hurts training performance.

%The results on $\E{K_0}$ and $Var\left[K_0\right]$ cannot be applied to DNNs with ReLU gates, since the weights and the gates are not independent. However, the results apply to GaLU networks, where the gating pattern is generated by a different network, and hence the statistical independence of weights and the gating pattern can be ensured. We also present experimental results to support the theory.

\end{comment}
$\bullet$ \textbf{Generalisation:} We show via experiments on standard datasets such as MNIST and CIFAR that better generalisation happens when the activations $A_t$ change with time. In particular, for MNIST restricted to two labels, we experimentally verify that the quantity $y^\top (\widehat{H_t})^{-1}y$\footnote{For a matrix $H$, $\hat{H}=\frac{1}{trace(H)}H$ .} reduces as the training progresses. This shows that the eigenspaces of the NPK matrix align themselves in accordance to the labelling function. These experiments supports the fact that NPF and the NPK are indeed learnt during training in a manner to improve the generalisation performance. 
%sWe develop preliminary theory to throw light on feature learning, and argue that, at time $t$, and input example $x_s\in \R^{d_{in}},s\in[n]$, there are two sub-networks, namely i) an \emph{active} sub-network (for whose paths $\partial A_t(x_s,p)$ is close to zero) which holds the memory corresponding to that input example, and (ii) a \emph{sensitive} sub-network (for whose paths $\partial A_t(x_s,p)$ is significant), where the feature learning happens.


%In a DNN with ReLU activations, each input example can be associated with a gating pattern, i.e., the set of gates that are \emph{on} for that example. Note that, only the sub-network of those weights corresponding to these \emph{on} activations/gates are responsible for predicting the output. In this paper, we obtain several new insights related to optimisation and generalisation in DNNs via a formal study of the paths and gates. In what follows, we discuss the organisation of the paper and key contributions.

\begin{comment}
\textbf{Path-View:} Central to the contributions in the paper is the concept of \emph{path-view}, wherein, paths are regarded as basic building blocks of DNNs. A \emph{path} starts from an input node $i\in[d_{in}]$, passes through exactly one weight and one activation in each layer, and finally ends at the output node. Using shorthand notation, say a path $p$ pass through input node $\I_0(p)\in [d_{in}]$, and weights $\Theta(l,\I_{l-1}(p),\I_l(p)),l\in[d-1]$, and gates $G(l,I_l(p))$

For an FC-DNN with $d$ layers (depth), and $w$ hidden units per-layer, the total number of paths (denoted by $P$) starting from a given input node $i\in[d_{in}]$ is $w^{(d-1)}$.

$\bullet$ At time $t$, for an input $x\in \R^{d_{in}}$, a path $p$ is associated with two quantities namely: (i) the path value denoted by $v_t(p)$, which is the product of the weights in the path, and (ii) the path activation level denoted by $A_t(x,p)$ which is the product of the gates\footnote{In the case of DNN with ReLU activations/gates, $A_t(x,p) \in \{0,1\}$.} in the path. Note that the activation level of a path is dependent on the input and the value of a path is not dependent on the input.
\end{comment}


\begin{comment}
In a DNN with ReLU activations, each input example can be associated with a gating pattern, i.e., the set of activations that are \emph{on} for that example. Note that, only the sub-network of those weights corresponding to these \emph{on} activations/gates are responsible for predicting the output. In this paper, we obtain several new insights related to Questions I and II by a formal study of the paths and gates.
\end{comment}

\section{Path-View: Neural Path Features and Information Flow}\label{sec:pathgate}
\textbf{Notation:} We consider fully connected deep networks with $d$ layers, and $w$ hidden units per layer, which accepts an input $x\in\R^{d_{in}}$, and produces an output $\hat{y}_{\Theta}\in \R$ where $\Theta\in\R^{d_{net}}$ ($d_{net}=w^2(d-2)+w(d_{in}+1)$). We denote by $\Theta(l,i,j)$ the weight connecting the $i^{th}$ hidden unit of layer $l-1$ to the $j^{th}$ hidden unit of layer $l$, and we use $G_{x,\Theta}(l,i)$ be the gating value ($0/1$) of the $i^{th}$ hidden unit in layer $l$ for an input $x\in \R^{d_{in}}$.\\
\textbf{Gating:} In what follows, we consider two kinds of gates namely, i) hard gates, taking values in $\{0,1\}$, and ii) soft gates, taking values in $(0,1)$. For a pre-activation input $q\in\R$, the  hard and soft gates are given by $G(q)=\mathbbm{1}_{\{\q>0\}}$ and $G(q)=\frac{1}{1+\exp(-\beta q)}$, where $\beta>0$ is a positive constant. Using the hard and soft gates, we can specify the ReLU and `soft-ReLU' activations as $\chi(q)=q\cdot\mathbbm{1}_{\{q>0\}}$ and $\chi(q)=q\cdot\left(\frac{1}{1+\exp(-\beta q)}\right)$.\\
\textbf{Paths:}  We have a total of $P=d_{in}w^{(d-1)}$ paths. Let us say that an enumeration of the paths is given by $[P]=\{1,\ldots,P\}$. Let $\I_{l}\colon [P]\ra [w],l=0,\ldots,d-1$ provide the index of the hidden unit through which a path $p$ passes in layer $l$ (with the convention that $\I_d(p)=1,\forall p\in [P]$). The activity of a path $p$ for an input $x\in \R^{d_{in}}$ by $A_{\Theta}(x,p)\stackrel{def}{=}\Pi_{l=1}^{d-1} G_{x,\Theta}(l,\I_l(p))$.\\
We define $\phi_{x,\Theta}\stackrel{def}=(x_s(\I_0(p))A_{\Theta}(x_s,p) ,p\in[P])\in\R^P$ to be the \textbf{neural path feature} (NPF) of an input $x\in\R^{d_{in}}$, where, for a path $p$, $\I_0(p)$ is the input node at which the path starts, and $A_{\Theta}(x_s,p)$ is its activity. %By arranging the NPF of the $n$ input examples in a matrix $\Phi_t=(\phi_{x_s,\G_t},s\in[n])\in\R^{P\times n}$, we can express 
We define $v_{\Theta}\stackrel{def}=(\Pi_{l=1}^d \Theta(l,\I_{l-1}(p),\I_l(p)),p\in[P])\in\R^P$ to be the \textbf{neural path value} (NPV). The zeroth and first-order terms of a DNN can be written as:
\begin{align}
\label{eq:zero}\text{(Zeroth-Order)}\quad: \quad \hat{y}_{\Theta}(x)&=\ip{\phi_{x,\Theta},v_{\Theta}}=\sum_{p\in [P]}x(\I(p))A_{\Theta}(x,p)v_{\Theta}(p)\\
\label{eq:first}\text{(First-Order)}\quad: \quad \partial \hat{y}_{\Theta}(x)&=\underbrace{\ip{\phi_{x,\Theta},\partial v_t}}_{\text{value gradient}}+ \underbrace{\ip{\partial\phi_{x,\Theta},v_{\Theta}}}_{\text{feature gradient}}\\&=\sum_{p\in [P]}x(\I(p)) A_{\Theta}(x,p) \partial(v_{\Theta}(p))+\sum_{p\in [P]}x(\I(p)) \partial(A_{\Theta}(x,p)) v_{\Theta}(p)\nn
\end{align}
\subsection{Information Flow: Zeroth-Order}
$1.$ \textbf{Representational Power Vs Scale Invariance:} The ability of DNNs to fit data has been demonstrated in the past. \cite{ben} showed that DNNs can fit even random labels, and random pixels of standard datasets such as MNIST. However, we note that, for DNNs with ReLU, with no bias parameters, a dataset with $n=2$ points namely $(x,1)$ and $(x/2,-1)$ for some $x\in \R^{d_{in}}$ cannot be memorised. The reason is that the gating values are the same for both $x$ and $x/2$ (for that matter any positive scaling of $x$), and hence $\phi_{x/2,\Theta}= \phi_{x,\Theta}/2$, and thus it not possible to fit arbitrary values for $\hat{y}_t(x/2)$ and $\hat{y}_t(x)$, since $\hat{y}_t(x/2)= \hat{y}_t(x)/2$. Cartoons $(a)$ and $(b)$ in \Cref{fig:cartoon} illustrate this scale invariance for inputs $x=(1,-1,2)\in\R^3$ and $x/2=(0.5,-0.5,1)\in\R^3$:  when compared to $(a)$, pre-activation values and output of $(b)$ are scaled down by a factor $2$, and the gating values of $(a)$ and $(b)$ are identical.\\
$2.$ \textbf{Translation Invariance:} Consider a convolution network using $l$ layers of circular convolutions\footnote{Here, instead of zero-padding in the corners, we follow the convention that index $d_{in}+k$ will be interpreted as $k$, for $k>0$, and $-k$ will be interpreted as $d_{in}-k$.} with filter size $k'<d_{in}$ and unit stride, and let $x^l(i)$ be the output of the $i^{th}$ channel after either $\max$-/global-average-pooling. Looking at the third and fourth (from left) illustrations in \Cref{fig:cartoon}, it is easy to check the translation invariance property: each of the red, blue, green lines have the same path values due to weight sharing in the convolutional layers, this leads to a circular symmetry in the path values, due to which, a translation in the input will cause all the internal variable to translate, and final invariance results from the invariance of the $\max$/average operation.\\
$3.$ \textbf{Active Sub-Network:} For each input example, a set of gates are \emph{on} in each layer, and this gives rise to the sub-network of active paths for that input. This active sub-network can be said to hold the memory for a given input (see cartoon $(b)$ in \Cref{fig:cartoon}).
\begin{figure*}[t]
%\begin{minipage}{0.78\columnwidth}
\resizebox{\columnwidth}{!}{
\begin{tabular}{ccc}
\includegraphics[scale=0.5]{figs/nn-subnet.png}
\includegraphics[scale=0.5]{figs/nnsoft.png}
\includegraphics[scale=0.5]{figs/nnconv.png}
%\includegraphics[scale=0.5]{figs/nntwin-blck.png}
%\includegraphics[scale=0.5]{figs/nn.png}
%\includegraphics[scale=0.5]{figs/nn.png}
\end{tabular}
}
%\end{minipage}
%\begin{minipage}{0.18\columnwidth}
%\resizebox{\columnwidth}{!}{
%\includegraphics[scale=0.5]{figs/nnconv.png}
%}
%\end{minipage}
\caption{Cartoon illustration of usefulness of path-view and DGN framework.}
\label{fig:cartoon}
\end{figure*}
\subsection{Information Flow: First-Order}
$1.$ \textbf{Value Gradient} $\ip{\phi_{x,\Theta},\partial v_\Theta}$ (see \eqref{eq:zero}) flows through the active sub-network. To see this, $\ip{\phi_{x,\Theta},\partial v_{\Theta}}=\sum_{p\in[P]}x(\I_0(p))A_{\Theta}(x,p)\partial v$, i.e., only paths $p$ with $A_{\Theta}(x,p)=1$ contribute to the summation and $\partial v_{\Theta}(p)$ is non-zero only for those weights through which the path $p$ passes.\\
$2.$ \textbf{Feature Gradient}  $\ip{\partial\phi_{x,\Theta},v_{\Theta}}=\sum_{p\in [P]}x(\I(p)) \partial(A_{\Theta}(x,p)) v_{\Theta}(p)$. Note that, in the case of ReLU activations the gating values are either $0$ or $1$, and hence $\partial(A_{\Theta}(x,p))=0$. However, the gating values and hence the path activities $A_{\Theta_t}(\cdot,\cdot)$ changes during training. This artefact arising due to non-differentiability can be fixed by considering a \emph{soft-ReLU} activation. 
%where, for a pre-activation input $q\in\R$ the output is given by $q\frac{1}{1+\exp(-\beta q)}$ as opposed to $q\mathbbm{1}_{\{q>0\}}$ of the ReLU. 
Soft-ReLU `trick' enables us to capture the terms related to feature learning in our analysis. The difference between hard and soft gates can be seen cartoons $(a)/(b)$ and $(c)$ in \Cref{fig:cartoon}, where negative/positive values lead to $0/1$ in the case of hard gating, and close to $0/1$ in the case of soft-gating.\\
$3.$ \textbf{Sensitive Sub-Network:} In a DNN with soft-ReLU, $\partial A_{\Theta}(x,p)=\sum_{l=1}^d \partial G_{x,\Theta}(l)\Pi_{l'\neq l}G_{x,\Theta}(l')$ is significant for those paths $p$ for which one of the $d-1$ gating values is close to $0.5$ (say such a gate is in layer $l$, and for such a gate $\partial G_{x,\Theta}$ is significant) and rest of the $d-2$ gates are close to $1$, so that $\Pi_{l'\neq l}G_{x,\Theta}(l')$ is significant. The set of paths for which $\partial A_{\Theta}(x,p)$ is significant, form the sensitive sub-network for that input. The sensitive paths are shown in cartoon $(c)$ of \Cref{fig:cartoon}. In the case of, standard ReLU, sensitive paths are those which contain one of the gates with pre-activation close to $0$, and rest of the $d-2$ gates are $1$.
\begin{comment}
\subsection{Deep Gated Network (DGN)}
The NPFs are zeroth-order features stored solely in the gates of a DNN. In this paper, we separate out the NPFs from the NPVs. To this end, we introduce the deep gated network (DGN) framework, wherein, the output of a hidden unit is obtained as a product of its pre-activation and a gating value. A DGN has two network namely i) the gating network which holds the gating values and hence the NPF, and ii) a value network which holds the NPVs. %We denote a DNG by $\N(\Theta_t;\G(\Tg_t,\beta)$ or simply $\N(\Theta_t;\Tg_t,\beta)$
\begin{table}[h]
\begin{minipage}{0.5\columnwidth}
\resizebox{\columnwidth}{!}{
\begin{tabular}{|c|c|}\hline
Gating Network: $\G(\Tg_t,\beta)$\\\hline
$z_{x,\Tg_t}(0)=x$  \\\hline
$q_{x,\Tg_t}(l)={\Tg_t(l)}^\top z_{x,\Tg_t}(l-1)$ \\\hline
$z_{x,\Tg_t}(l)=q_{x,\Tg_t}(l)\odot G_{x,\Tg_t}(l)$ \\\hline
{$\begin{aligned}\beta >0: G_{x,\Tg_t}(l,i)&=\frac{1+\epsilon}{1+\exp(-\beta q_{x,\Tg_t}(l,i))} \\ \beta=\infty: G_{x,\Tg_t}(l,i)&=\mathbbm{1}_{\{q_{x,\Tg_t}(l,i)>0\}}\end{aligned}$}\\\hline 
\end{tabular}
}
\end{minipage}
\begin{minipage}{0.5\columnwidth}
\resizebox{\columnwidth}{!}{
\begin{tabular}{|l|l|}\hline
\multicolumn{2}{|c|}{Value Network: $\N(\Theta_t;\G_t)$}\\\hline 
Input layer & $z_{x,\Theta_t}(0)=x$ \\\hline
Pre-activation & $q_{x,\Theta_t}(l)={\Theta_t(l)}^\top z_{x,\Theta_t}(l-1)$\\\hline
Layer output & $z_{x,\Theta_t}(l)=q_{x,\Theta_t}(l)\odot G_{x,t}(l)$ \\\hline
Final output & $\hat{y}_t(x)={\Theta_t(d)}^\top z_{x,\Theta_t}(d-1)$\\\hline
Gating Values& $\begin{aligned}\G_t\stackrel{def}=\{G_{x_{s},t}(l,i), \forall s\in[n],\\l\in[d-1],i\in[w]\}\end{aligned}$\\\hline
\end{tabular}
}
\end{minipage}
\caption{$q(l),z(l)$ and $G(l)$ are $w$-dimensional quantities}
\label{tb:dgn}
\end{table}
\end{comment}
\section{Information Flow in DGNs}\label{sec:expressivity}
\textbf{Paths:} In a fully connected DGN, $d$ layers and $w$ hidden units per layer, we have a total of $P=d_{in}w^{(d-1)}$ paths. Let us say that an enumeration of the paths is given by $[P]=\{1,\ldots,P\}$. Let $\I_{l},l=0,\ldots,d-1$ provide the index of the hidden unit through which a path $p$ passes in layer $l$ (with the convention that $\I_d(p)=1,\forall p\in [P]$). We then define:\\
$1.$ The value of a path $p$ by $v_t(p)\stackrel{def}=\Pi_{l=1}^d \Theta_t(l,\I_{l-1}(p),\I_l(p))$.\\
$2.$ The activity of a path $p$ for an input $x\in \R^{d_{in}}$ by $A_{t}(x,p)\stackrel{def}{=}\Pi_{l=1}^{d-1} G_{x,t}(l,\I_l(p))$.\\
\textbf{Value Derivative:} For a path $p$, the derivative of its value with respect to any weight in the path is: 
\begin{align}\label{eq:vft}{\partial v_t(p)}/{\partial \Theta\left(l,\I_{l'-1}(p),\I_{l'}(p)\right)}|_{\Theta=\Theta_t}= \underset{l=1}{\underset{l\neq l'}{\overset{d}{\Pi}}} \Theta_t\left(l,\I_{l-1}(p),\I_{l}(p)\right)
\end{align}
If a path $p$ does not pass through a $\theta\in\Theta$, then ${\partial v_t(p)}/{\partial \theta}=0$.\\
\textbf{Activity Derivative:} For a path $p$, and an input $x\inrdin$ the derivative of its activity with respect to any weight is given by 
\begin{align}
\frac{\partial A_{t}(x,p)}{\partial \tg}= \sum_{l=1}^{d-1} \Big(\frac{\partial G_{x,\Tg_t}(l,\I_l(p))}{\partial \tg} \Big)\Big(\Pi_{l'\neq l} G_{x,\Tg_t}(l',I_{l'}(p))\Big)
\end{align}
\begin{comment}
\begin{table}
\begin{minipage}{0.5\columnwidth}
%\resizebox{\columnwidth}{!}{
\begin{tabular}{|c|l|}\hline								 								 													
NPF		&$\phi_{x,t}=(x(\I_0(p))A_t(x,p) ,p\in[P])\in \R^P$\\\hline	
NPK		&$H_t(x,x')=\ip{\phi_{x,t},\phi_{x',t}}$\\\hline		
OAP		&$\lambda_t(x,x')=\sum_{p\rsa i} A_t(x,p) A_t(x',p)$\\\hline
VTP		&$\varphi^v_{p,t}=(\partial_{\theta}v_t(p),\theta\in\Theta)\inrdnet$ \\\hline	
ATP		&$\varphi^a_{x,p,t}=(\partial_{\tg}A_t(x,p),\tg\in\Tg)\inrdnet$ \\\hline	
\end{tabular}
%}
\end{minipage}
\hspace{15pt}
\begin{minipage}{0.5\columnwidth}
%\resizebox{\columnwidth}{!}{
\begin{tabular}{|c|l|}\hline								 								 													
OSP 	&$\delta_t(x,x')=\sum_{p\rsa i} \ip{\varphi^a_{x,p,t},\varphi^a_{x',p,t}}$\\\hline
VG		&$\psi^v_{x,t}=\nabla_{\Theta} \hat{y}_t(x)\in\R^{d_{net}}$\\\hline
AG		&$\psi^{\phi}_{x,t}=\nabla_{\Tg} \hat{y}_t(x)\in\R^{d_{net}}$\\\hline
NTF		&$\psi_{x,t}=(\psi^v_{x,t},\psi^{\phi}_{x,t})\in\R^{2d_{net}}$\\\hline
NTK 		&$K_t(x,x')=\psi^\top_{x,t}\psi_{x',t}$\\\hline
\end{tabular}
%}
\end{minipage}
\caption{Shows all zeroth-order and first-order quantities related to information flow in a DGN.}
\label{tb:terms}
\end{table}
\end{comment}
\subsection{Neural Path Feature and Kernel}
The \emph{neural path feature} of an input example $x_s\in \R^{d_{in}}$ is given by $\phi_{x_s,\G_t}=(x_s(\I_0(p))A_t(x_s,p) ,p\in[P])\in\R^P$. Here, for a path $p$, $\I_0(p)$ is the input node at which the path starts and $A_t(x_s,p)$ is its activation level. By arranging the NPF of the $n$ input examples in a matrix $\Phi_t=\left[\phi_{x_1,t},\ldots, \Phi_{x_n,t}\right]$, we can express the predicted output of a DGN as: 
\begin{align}\label{eq:npfbasic}
\hat{y}_t=\Phi_t^\top v_t,
\end{align}
where, the value of the path $v_t$ is the equivalent of the so called \emph{weight-vector} in a standard linear approximation. 
The significance of the NPF are:\hfill\\
$1.$ \textbf{Signal-Wire Separation}: Note that $\Phi$ encodes the signal: say for a DNN with ReLU activations, the co-ordinate corresponding to path $p$ is either $x(\I_0(p))$ if the path is active for that input (i.e., $A_t(x,p)=1$) or $0$ if the path is inactive for that input  (i.e., $A_t(x,p)=0$). The value of the path thus encodes the \emph{wire}, i.e., the information contained in the weights of the network. \hfill\\
$2.$ \textbf{Deep Information Propagation:} The path view provides a novel way of looking at information propagation in DNNs, eschewing the conventional `layer-by-layer' expression for information flow.\hfill\\
As associated with the NPF matrix is the \emph{neural path kernel} matrix, given by $H_t=\Phi^\top_t\Phi_t$. 
\begin{definition}\label{def:lambda}
$\lambda_t(s,s')\stackrel{def}{=}\sum_{p\rsa i} A_t(x,p) A_t(x',p)$, $\forall s,s'\in[n]$, any $i\in [d_{in}]$,  
 \end{definition} 
\begin{lemma}\label{lm:npk}[Neural Path Kernel] 
Let $x=(x_s,s\in [n])\in\R^{d_{in}\times n}$ be the data matrix and let the neural path kernel matrix be defined as $H_t\stackrel{def}=\Phi^\top_t\Phi_t$. It follows that $H_t= (x^\top x)\odot(\lambda_t)$. 
\end{lemma}
\textbf{NPFs and Optimisation:} The ability of DNNs to fit data has been demonstrated in the past \cite{ben}, i.e., they can fit even random labels, and random pixels of standard datasets such as MNIST. However, for standard DNNs with ReLU gates, with no bias parameters, a dataset with $n=2$ points namely $(x,1)$ and $(2x,-1)$ for some $x\in \R^{d_{in}}$ cannot be memorised. The reason is that the gating values are the same for both $x$ and $2x$ (for that matter any positive scaling of $x$), and hence $\phi_{2x,\G_t }= 2\phi_{x,\G_t }$, and thus it not possible to fit arbitrary values for $\hat{y}_t(x)$ and $\hat{y}_t(2x)$.\\
\textbf{NPFs and Generalisation:} We trained DGN on standard datasets namely MNIST and CIFAR-10, under the following conditions: i) the gates are frozen $\G_t=\G_0,\forall t\geq 0$ and ii) the gating values are obtained from a ReLU network, which acts as the gating network (see \Cref{tb:dgn}). Since, the gates are frozen, the NPFs are fixed and the SGD learns only the path values. We compare the performance of $4$ different NPFs, wherein, the gates are copied from i) from a randomly initialised ReLU network (untrained), ii) from a ReLU network trained with good dataset iii) ReLU network trained on random labels and iv) ReLU network trained on random pixels.
\FloatBarrier
\begin{table}[h]
\begin{tabular}{|c|c|c|c|c|c|c|}\hline
&&&&\multicolumn{3}{c|}{NPF (trained)}\\\cline{5-7}
$(w,d)$	&Dataset		&ReLU		&NPF(untrained) 		&Good 		&Random Labels 	&Random Pixel\\\hline
$(128,6)$	& MNIST 		& $98.15$ 		&$96$ 		&$98.3$		&$92.6$			&$94.3$\\\hline
$(256,6)$	& MNIST 		& $98.5$ 		&$96.6$ 		&$98.4$		&$92.0$			&$81.1$\\\hline
\end{tabular}
\caption{Shows the training and generalisation performance of various GaLU network. Here, non-learned stands for gates from a randomly initialised ReLU network.}
\label{tb:npfs}
\end{table}
\textbf{NPF dynamics:} 
We consider ``Binary''-MNIST data set with two classes namely digits $4$ and $7$, with the labels taking values in $\{-1,+1\}$ and squared loss. We trained a standard DNN with ReLU activation ($w=100$, $d=5$). Recall that $H_t=\Phi^\top_t\Phi_t$  (the Gram matrix of the features) and let $\widehat{H}_t=\frac{1}{trace(H_t)}H_t$ be its normalised counterpart. For a subset size, $n'=200$ ($100$ examples per class) we plot $\nu_t=y^\top (\widehat{H}_t)^{-1} y$, (where $y\in\{-1,1\}^{200}$ is the labeling function), and observe that $\nu_t$ reduces as training proceeds (see first plot in \Cref{fig:gen}). Note that $\nu_t=\sum_{i=1}^{n'}(u_{i,t}^\top y)^2 (\hat{\rho}_{i,t})^{-1}$, where $u_{i,t}\in \R^{n'}$ are the orthonormal eigenvectors of $\widehat{H}_t$ and $\hat{\rho}_{i,t},i\in[n']$ are the corresponding eigenvalues. Since $\sum_{i=1}^{n'}\hat{\rho}_{i,t}=1$, the only way $\nu_t$ reduces is when more and more energy gets concentrated on $\hat{\rho}_{i,t}$s for which $(u_{i,t}^\top y)^2$s are also high. However, in $H_t=(x^\top x)\odot \lambda_t$, only $\lambda_t$ changes with time. Thus, $\lambda_t(s,s')$ which is a measure of overlap of sub-networks active for input examples $s,s'\in[n]$, changes in a manner to reduce $\nu_t$. We can thus infer that the \emph{right} active sub-networks are learned over the course of training. We now summarise the insights obtained from these experiments in the following remarks:\hfill\\
\begin{wrapfigure}{R}{0.3\textwidth}
\centering
\includegraphics[scale=0.25]{figs/path-gram.png}
\caption{\label{fig:frog1}This is a figure caption.}
\end{wrapfigure}

\begin{comment}
\begin{figure}[h]
\centering
\begin{tabular}{cc}
\includegraphics[scale=0.25]{figs/path-gram.png}
&
\includegraphics[scale=0.25]{figs/path-gram.png}
\end{tabular}
\caption{First two plots from the left show optimisation and generalisation in ReLU and GaLU networks for standard MNIST. The right most plot shows $\nu_t=y^\top (\widehat{H}_t)^{-1}y$, where $H_t=\Phi_t^\top \Phi_t$.}
\label{fig:gen}
\end{figure}
\end{comment}

\section{Prior Art and Gaps: Neural Tangent Feature and Kernel }
\textbf{Neural tangent feature} (NTF) is the collection of the output gradients with respect to the network parameters, and is given by $\psi_{x_s,t}=(\nabla_{\Theta|_{\Theta=\Theta_t}} \hat{y}_{\Theta}(x_s))\in\R^{d_{net}}$.  The NTF matrix can be used to linearise the output of a DNN about the point $\Theta_t\inrdnet$. To see this, let the NTF matrix be $\Psi_t=[\psi_{x_1,t},\ldots, \psi_{x_n,t}]\in\R^{d_{net}\times n}$, then a linearisation of the DNN output about $\Theta_t$ is given by: $\hat{y}_{\Theta_t+\Theta}=\hat{y}_{\Theta_t} + \Psi^\top_t (\Theta-\Theta_t)$. An associated \emph{neural tangent kernel} is then given by $K_t=\Psi^\top_t\Psi_t$.\\
\textbf{Trajectory Method:} Recent works have made use of the trajectory method to show that gradient descent achieves zero training error in over-parameterised DNNs. In the \emph{trajectory} based analysis, one looks at the dynamics of error $e_t$  (i.e., difference between predicted and true values) at time $t$. For a small step-size $\alpha_t>0$, the error dynamics follows a linear recursion given by: $e_{t+1}=e_t-\alpha_tK_te_t$, where $K_t$ is the NTK matrix obtained on the the dataset. Thus, the spectral properties of $K_t$ is key to achieve zero training error. \hfill\\
\textbf{Optimsation:} \cite{ntk} were the first to point out the role of NTK in DNNs. Using the trajectory based analysis, \cite{dudnn} show that in fully connected DNNs with $w=\Omega(poly(n)2^{O(d)})$, and in residual neural networks (ResNets) with $w=\Omega(poly(n,d))$ gradient descent converges to zero training loss.\\
\textbf{Research Gap I:} The result by \cite{dudnn} shows that ResNets are better than fully connected DNNs, based on the fact that the dependence on the number of layers improves exponentially for ResNets. However, Question I (see \Cref{sec:intro}), i.e., `why depth helps in training?' is unresolved.\\
\textbf{Generalisation:} Prior works by \cite{arora2019exact,cao2019generalization} suggest that, for randomised initialisation, DNNs can be thought of as learning with the linear features given by the random NTFs, and provide generalisation bounds with the corresponding NTK. They use NTK at initialisation to provide generalisation bounds as well as propose pure-kernel methods. However, couple of issues remain unresolved: firstly, if the DNNs are only linear learners with random NTFs, then it suggests that no feature learning happens in DNNs, and secondly, it was observed in prior experiments that the DNNs perform better than their corresponding NTK counterparts \cite{arora2019exact,lee2017deep}. \hfill\\
\textbf{Research Gap II:} However, couple of issues remain unresolved: firstly, if the DNNs are only linear learners with random NTFs, then it suggests that no feature learning happens in DNNs, and secondly, it was observed in prior experiments that the DNNs perform better than their corresponding NTK counterparts \cite{arora2019exact,lee2017deep}.
\section{Bridging the Gap: NTK equals NPK for random initialisation}
In this section, we assume that the NPKs are given to us and fixed (i.e, $G_t=\G_0,\forall t\geq 0$ is given to us). We show that, at randomised weight initialisation, the NTK is equal to (but for a scaling factor) to the NPK. We then use the \emph{Hadamard} structure of the NPK to comment about optimisation and generalisation. 
\begin{assumption}\label{assmp:main}
(i) $\Theta_0\inrdnet$  is statistically independent of $\G_0$ and (ii) The weights $\Theta_0$ are sampled i.i.d from a distribution such that for any $\theta_0\in\{\Theta_0\cup \Tg_0\}$,  we have $\E{\theta_0}=0$, and  $\E{\theta^2_0}=\sigma^2$, and $\E{\theta^4_0}={\sigma'}^2$.
\end{assumption}
\begin{lemma}\label{lm:disentangle}[Disentanglement] Let $\varphi_{p,t}=(\nabla_{\Theta} v_t(p))\inrdnet$, 
under \Cref{assmp:main}-(ii), $\forall\,\theta\in\Theta$, for paths $p,p'\in [P], p\neq p'$:  i) $\E{\ip{\varphi_{p,0},\varphi_{p',0}}}= 0$ and ii)$\E{\ip{\varphi_{p,0},\varphi_{p,0}}}= \sigma^{2(d-1)}$, iii) $\E{v_0(p)v_0(p')}=0$, iv) $\E{v^2_0(p)}=\sigma^{2d}$.
\end{lemma}
\begin{theorem}\label{th:main} Under \Cref{assmp:main}, we have:\\
(i) $\E{K_0}=\sigma^{2(d-1)}H_0=\sigma^{2(d-1)}(x^\top x)\odot(\lambda_0)$,\\
(ii) In addition, if ${4d}/{w^2}<1$, then $Var\left[K^v_0\right]\leq O\left(d^2_{in}\sigma^{4(d-1)}\max\{d^2w^{2(d-2)+1}, d^3w^{2(d-2)}\}\right)$,\\
\end{theorem}
\textbf{Proof of \Cref{th:main}-(i):} Let $\varphi_t=(\varphi_{p,t},p\in[P])\in \R^{d_{net}\times P}$ matrix, then since $K_t=\Psi^\top_t\Psi$, where $\Psi_t=\varphi_t \Phi_t$, we have $\E{K_t}=\E{\Phi^\top_t \varphi^\top_t \varphi_t \Phi_t}$. At initialisation, using the \Cref{assmp:main}-(i), we can pull out $\Phi^\top_t$ and $\Phi_t$ outside of the expectation to have \begin{align}\label{eq:pullout}\E{K_0}=\Phi^\top_0\E{ \varphi^\top_t \varphi_t }\Phi_0,\end{align} and from \Cref{lm:disentangle}, it follows that $\E{ \varphi^\top_t \varphi_t }=d\sigma^{2(d-1)}I$, and hence $\E{K_0}=d\sigma^{2(d-1)}\Phi^\top_0\Phi_0=d\sigma^{2(d-1)}H_0$.\\
\textbf{Discussion:}\\
$1.$ \emph{Active Sub-Network and Gradient Flow:}  Each input example has its own associated set of active sub-network, and while training a particular example, the gradient flows through the weights of the corresponding active sub-network. Now, the active sub-networks corresponding to different examples have some overlap, and hence there is bound to be \emph{cross-talk} of the gradients flowing through them. This overlap is captured by $\lambda_t(s,s')$ which is the measure of overlap of the sub-networks that are active for both the inputs $x,x'\in\R^{d_{in}}$. Under. \Cref{assmp:main}, the inter-path interaction $\varphi^\top_t\varphi_t$ gets disentangled, result in the claim \Cref{th:main}-(i).\\
$2.$ As seen from \Cref{th:main}, $\lambda_0$ directly controls the spectral properties of the NTK matrix $K_0$. Characterising $\lambda_0$ for the general case is left as future work. In what follows, we present an informal reasoning that increasing depth causes whitening of $\lambda_0$. However, we consider a special case in \Cref{sec:spectrum}, where, we give an explicit characterisation of the spectrum of $\E{K_0}$.\\
$3.$ \emph{Why increasing depth till a point helps in training? } From \Cref{th:main}-(ii) it follows that for $w\ra\infty$, $K_0\ra\E{K_0}$. We now argue that when $\sigma=\sqrt{\frac{2}{w}}$, increasing depth causes whitening of $\lambda_0$, and hence $K_0$ .\hfill\\
$\bullet$ Let us first look at the diagonal terms of $\lambda_0$. It is reasonable to assume that, owing to the symmetric nature of the weights, roughly $\mu=\frac{1}{2}$ fraction of the gates are \emph{on} every layer. Thus $\lambda_0(s,s)\approx (w/2)^{d-1}$. Now, due our choice of $\sigma=\sqrt{\frac{2}{w}}$, the diagonal entries will be close to $1$.\hfill\\
$\bullet$ We now turn our attention towards the non-diagonal entries of $\lambda_0$. Define $\tau(s,s',l)\stackrel{def}=\sum_{i=1}^w G_{x_s,t}(l,i)G_{x_{s'},t}(l,i)$ be the overlap of the active gates in layer $l$ for input examples $s,s'\in[n]$, and  let $\eta\stackrel{def}=\max_s\left(\max_{s',l} \frac{\tau(s,s',l)}{\tau(s,s,l)}\right)$ be the maximum overlap between gates of a layer (maximum taken over over input pairs $s,s'\in[n]$ and layers $l\in [d]$).  Then it follows that $\max_{s,s'\in [n]} \frac{\bar{\lambda}_{cross}(s,s')}{\bar{\lambda}_{self}(s)}\leq \eta^{d-1}$. Thus, the non-diagonal entries decay an exponential rate in comparison to the diagonal entries.\hfill\\
$4.$ \emph{Why increasing the depth beyond hurts training?} Note that for $\sigma=O\left(\sqrt{\frac{1}{w}}\right)$, for a fixed depth $d$, as width $w$ increases, $K_0\ra\E{K_0}$. However, the variance expression in \Cref{th:main}-$(ii)$ involves $d^2$ and $d^3$ terms, and hence for a fixed width as depth increases, the entries of $K_0$ deviates from $\E{K_0}$, and as a result the spectrum of $K_0$ degrades, thereby hurting training performance.\\
$5.$ \emph{Generalisation and Feature Learning:} As seen in \Cref{tb:npfs}, we know that, different NPFs give different generalisation performance. In this light, \Cref{th:main} complements the results by \cite{arora2019exact,cao2019generalization}, in that, one can plug-in the NPK in the place of NTK in their generalisation bounds. Further, the gap in the performance of the DNN and the NTK counterparts can be explained by the fact that the NPK keeps changing during training (see \Cref{fig:gen}). We will discuss learning of NPF in \Cref{sec:featlearn}.\\
$6.$ \Cref{assmp:main} is not satisfied by ReLU activations, i.e., conditioned on the fact that a ReLU is \emph{on}, the incoming weights cannot all be simultaneously negative. This implies that the $\Phi^\top_t$ and $\Phi_t$ terms cannot be pulled out of the expectation as in \eqref{eq:pullout}.
\subsection{Fixed random gating: Explicit spectral characterisation} 
\FloatBarrier
\begin{wrapfigure}{h}{0.25\textwidth}
\includegraphics[scale=0.22]{figs/dgn-fra-ecdf-ideal.png}
\includegraphics[scale=0.22]{figs/dgn-fra-ecdfbyd-w25.png}
\includegraphics[scale=0.22]{figs/dgn-fra-ecdfbyd-w500.png}
\includegraphics[scale=0.21]{figs/dgn-fra-conv-w25.png}
\includegraphics[scale=0.21]{figs/dgn-fra-conv-w500.png}
\caption{}
\label{fig:dgn-frg-gram-ecdf}
\end{wrapfigure}
In this section, we consider the a case where the spectrum of $\lambda_0$ (hence $H_0$ and $K_0$) can be explicitly characterised. Here, for each input example in the dataset, we sample gating values from $Ber(\mu)$ taking values in $\{0,1\}$, and collect it in $\G_0$. In this case, it is easy to check that $\mathbb{E}_{\mu}\left[\lambda_0(s,s)\right]=(\mu w)^{(d-1)},\forall s\in[n]$ and $\mathbb{E}_{\mu}\left[\lambda_0(s,s')\right]=(\mu^2 w)^{(d-1)},\forall s,s'\in[n]$.\\
\textbf{Dataset:} $(x_s,y_s)_{s=1}^n\in \R\times \R$, where $x_s=1,\forall s\in [n]$, and $y_s\sim unif([-1,1])$, $n=200$. \\
\textbf{NPK:} The input Gram matrix $x^\top x$ is a $n\times n$ matrix with all entries equal to $1$ and its rank is equal to 1, and hence $H_0=\lambda_0$.\\
\textbf{Spectral Properties:} For $\sigma=\sqrt{\frac{1}{\mu w}}$, and by further averaging $\mathbb{E}_{\mu}\left[K_0(s,s)/d\right]=1$, and $\mathbb{E}_{\mu}\left[K_0(s,s')/d\right]=\mu^{(d-1)}$. Now, let $\rho_i\geq 0,i \in [n]$ be the eigenvalues of $\frac{\E{K_0}}{d}$, and let $\rho_{\max}$ and $\rho_{\min}$ be the largest and smallest eigenvalues. One can easily show that $\rho_{\max}=1+(n-1)\mu^{d-1}$ and corresponds to the eigenvector with all entries as $1$, and $\rho_{\min}=(1-\mu^{d-1})$ repeats $(n-1)$ times, which corresponds to eigenvectors given by $[0, 0, \ldots, \underbrace{1, -1}_{\text{$i$ and $i+1$}}, 0,0,\ldots, 0]^\top \in \R^n$ for $i=1,\ldots,n-1$.\\
\begin{comment}
\begin{figure}[h]
\resizebox{\textwidth}{!}{
\begin{tabular}{ccc}
\includegraphics[scale=0.4]{figs/dgn-fra-ecdf-ideal.png}
&
\includegraphics[scale=0.4]{figs/dgn-fra-ecdfbyd-w500.png}
&
\includegraphics[scale=0.4]{figs/dgn-fra-conv-w500.png}
\end{tabular}
}
\caption{Shows the plots for fixed random gating with $\mu=\frac{1}{2}$ and $\sigma=\sqrt{\frac{2}{w}}$. The first plot in the left shows the ideal cumulative eigenvalue (e.c.d.f) for various depths $d=2,4,6,8,12,16,20$. Note that the ideal plot converges to identity matrix as $d$ increases. The second plot from the left shows the cumulative eigenvalues (e.c.d.f) for $w=500$. }
\label{fig:dgn-frg-gram-ecdf}

\end{figure}
\end{comment}
\textbf{Numerical Evidence:} We look at the cumulative eigenvalue (e.c.d.f) obtained by first sorting the eigenvalues in ascending order then looking at their cumulative sum. The ideal behaviour (middle plot of \Cref{fig:dgn-frg-gram-ecdf}) as predicted from theory is that for indices $k\in[n-1]$, the e.c.d.f should increase at a linear rate, i.e., the cumulative sum of the first $k$ indices is equal to $k(1-\mu^{d-1})$, and the difference between the last two indices is $1+(n-1)\mu^{d-1}$. In \Cref{fig:dgn-frg-gram-ecdf}, we plot the actual e.c.d.f for various depths $d=2,4,6,8,12,16,20$ and $w=500$. \hfill\\
In order to compare how the rate of convergence varies with the depth, we set the step-size $\alpha=\frac{0.1}{\rho_{\max}}$, $w=100$. We use the vanilla SGD-optimiser. Note the$ \frac{1}{\rho_{\max}}$ in the stepsize, ensures that the uniformity of maximum eigenvalue across all the instances, and the convergence should be limited by the smaller eigenvalues. We also look at the convergence rate of the ratio $\frac{\norm{e_t}^2_2}{\norm{e_0}^2_2}$, and we observe that the convergence rate gets better with depth as predicted by theory.
\section{Feature Learning}
We saw in experiments (see \Cref{tb:npfs,fig:gen}) that the NFPs are learned during training. 
Consider a DGN, wherein, the gating network is parameterised by $\Tg\inrdnet$, and let $\Phi_{\Theta}=(\phi_{x_s,\Theta},s\in[n] )\in \R^{P\times n}$ be the NPF matrix $y=(y_s,\in[n])\in\R^n$ be the vector of target values to be learnt, and say we are interested in minimising the square loss given by:
\begin{align}\label{eq:sqloss}
\min_{\Theta\in \R^{d_{net}}, \Tg\in \R^{d_{net}}}\norm{\Phi_{\Tg}v_{\Theta}-y}^2_2
\end{align}
A gradient descent update for the above objective in \eqref{eq:sqloss} involves two gradient components, namely: \\
(i) The value gradient $\psi^v_{x,t}=\nabla_{\Theta} \hat{y}_t(x)\in\R^{d_{net}}$, which learns the values keeping the NPF fixed.\WFclear
(ii) The feature gradient $\psi^{\phi}_{x,t}=\nabla_{\Tg} \hat{y}_t(x)\in\R^{d_{net}}$, which learns the NPFs keeping the values fixed.
An interesting feature to note is that in the case when gating values belong to $\{0,1\}$ (case of $\beta=\infty$ in \Cref{tb:dgn}), the feature gradient is $0$. Further, when $\Tg=\Theta_t$, the NTF is given by $\psi_{x,t}=\psi^v_{x,t}+\psi^{\phi}_{x,t}$, which is a harder case to analysis. The choice of distinct parameters $\Tg\inrdnet$ and $\Theta\inrdnet$ decouples the gradients, in that, the NTF is given by $\psi_{x,t}=(\psi^v_{x,t},\psi^{\phi}_{x,t})\in\R^{2d_{net}}$. A desirable consequence is that the NTK matrix $K_t=K^v_t+K^{\phi}_t$ is an addition of two different Gram matrices, one corresponding to value learning and the other corresponding to feature learning. 
To start with, we gain insights into the joint optimisation and feature learning in \eqref{eq:sqloss}, we consider the following toy experiment based on simple modification of logistic regression.\\
\textbf{A Toy Experiment:} We look at a simple neural network with $d_{in}=2$, $w=2$ and $1$-hidden layer (see first diagram on the left in \Cref{fig:feat}). The first layer weights is an identity matrix, and for input $x\in\R^2$ to the network, the first layer output is given by $z_{t}(i)=x(i)G_t(i),i=1,2$, where $G_t(i)=\frac{1}{1+\exp(-\tg(i))},i=1,2$. In this network, there are $2$ paths, and the NPF is given by $\phi_{x,t}=(x(1)G_t(1),x(2)G_t(2))\in \R^2$ (note that this mimics the general structure of the NPF as presented in \Cref{sec:expressivity}).
We check the performance of frozen-gates versus adaptable gates in this network. We consider a binary classification task, with $n=500$ example for each class, with class $+1$ and $-1$ examples sampled from $U([0.1,1]\times [-a,a])$ and $U([-1,-0.1]\times[-a,a])$ respectively. We considered two cases $a=1$ (first two plots from the left in \Cref{fig:feat}), and $a=100$ (right most plot in \Cref{fig:feat}). 
We trained the simple neural network using gradient descent with step-size of $0.1$, initialisation $\Tg_0=\Theta_0=(0,0)\in\R^2$, and binary cross entropy loss for the following two cases: i) frozen-gates, wherein, we set $\Tg_0=\Tg_t=\Tg_0,\forall t\geq 0$, and train only $\Theta_t$ ii) adaptable gates, wherein, we train both $\Tg_t$ and $\Theta_t$. In all the plots, green bold and black dotted lines corresponds to classifier learned  by adaptable and frozen gates respectively, and the points are represented using the NPF.
%While both cases train for $T=10^4$ epochs, for $T=10^3$ epochs only the model with adaptable gates trains successfully. The results are shown in \Cref{fig:feat}, notice that the right most plot shows that in the case when the gates are adapting, they learn to suppress the second co-ordinate (the scale of $\phi_{x_s,T}$ is from $-15$ to $15$ as opposed to $-100$ to $100$ in $x_s$).
\FloatBarrier
\begin{figure}[h]
\begin{minipage}{0.15\columnwidth}
\includegraphics[scale=0.15]{figs/featlearn.png}
\end{minipage}
\hspace{50pt}
\begin{minipage}{0.8\columnwidth}
\resizebox{1\columnwidth}{!}{
\begin{tabular}{cccc}
\includegraphics[scale=0.2]{figs/simple-1e2.png}
&
\includegraphics[scale=0.2]{figs/simple-1e3.png}
&
\includegraphics[scale=0.2]{figs/simple.png}
&
%\includegraphics[scale=0.2]{figs/simple-1e4.png}
\\
\includegraphics[scale=0.2]{figs/adapt-1e2.png}
&
\includegraphics[scale=0.2]{figs/adapt-1e3.png}
&
\includegraphics[scale=0.2]{figs/adapt.png}
&
%\includegraphics[scale=0.2]{figs/adapt-1e4.png}
\end{tabular}
}
\end{minipage}

\caption{From left: second and third plots shows the training performance of frozen and adaptable gates respectively. In both plots, the bold line in green is the classifier learnt in the case of adaptable gates, and the dotted black line is the classifier learnt in the case of frozen gates. Notice the transformation of the feature space in the case of adaptable gates.}
\label{fig:feat}
\end{figure}
\textbf{Observations:}\\
$1.$ Initially for both networks $\tg_0=(0,0)^\top$, which means the horizontal and vertical co-ordinates in the NPF are both scaled by a factor of $0.5$. For the case of $a=1$, both networks train successfully in $T=10^2$ epochs, and we do not see any change in the NPFs. However, when trained for $T=10^3$ epochs, the network with adaptable gates learns to scale the horizontal co-ordinate in the NPF by factor $1.0$, while it retains a value of $0.5$ for the vertical co-ordinate (see second plot from left of \Cref{fig:feat}).\\
$2.$ In the case of $a=100$ (right-most plot), the adaptable network successfully trains in $T=10^3$ epochs by learning to scale the vertical co-ordinate down by a factor close to $0.15$. However, the network with frozen-gates does not train. We also observed that both networks trained successfully for $T=10^4$ epochs.
%\textbf{An open question:} Informally speaking, in the above example, even though the gating parameters had $2$-degrees of freedom, it was nonetheless sufficient to adapt the features. Thus, perhaps we can hypothesise that subject to the `well-conditioned' ness of $K^a_t$, such margin increase can be perhaps achieved for all the $n$ examples.  However, in practice, both $\Tg_t$ as well as $\Theta_t$ change, and an open question is to understand how the joint optimisation and feature learning happens. 

\begin{comment}
\textbf{Zeroth-order feature and kernel:} We now introduce the \emph{neural path feature} (NPF) and the \emph{neural path kernel} (NPK), which are zeroth-order quantities defined using the gating information $\G_t$. These quantities hold for all kinds of networks namely ReLU, GaLU, soft-ReLU and soft-GaLU. \\
The NPF of an input $x\in \R^{d_{in}}$ is given by $\phi_{x,t}=(x(\I_0(p))A_t(x,p) ,p\in[P])\in\R^P$. By arranging the NPF of the $n$ input examples in a matrix $\Phi_t=\left[\phi_{x_1,t},\ldots, \Phi_{x_n,t}\right]$, we can express the predicted output of a DGN as: \begin{align}\label{eq:npfbasic}\hat{y}_t=\Phi_t^\top v_t,\end{align}
where, the value of the path $v_t$ is the equivalent of the so called \emph{weight-vector} in a standard linear approximation. 

The significance of the NPF are:\hfill\\
$1.$ \textbf{Signal-Wire Separation}: Note that $\Phi$ encodes the signal: say for a DNN with ReLU activations, the co-ordinate corresponding to path $p$ is either $x(\I_0(p))$ if the path is active for that input (i.e., $A_t(x,p)=1$) or $0$ if the path is inactive for that input  (i.e., $A_t(x,p)=0$). The value of the path thus encodes the \emph{wire}, i.e., the information contained in the weights of the network. \hfill\\
$2.$ \textbf{Deep Information Propagation:} The path view provides a novel way of looking at information propagation in DNNs, eschewing the conventional `layer-by-layer' expression for information flow.\hfill\\
$3.$ \textbf{Representation:} As a parallel to the standard linear approximation, the NPF can be regarded as the \emph{hidden feature} and $v_t$ as the weight vector.
\end{comment}
\begin{comment}
$\lambda_t(s,s')$ is to be understood as the measure of overlap of the sub-networks that are active for both the inputs $x,x'\in\R^{d_{in}}$. Note that the definition of $\lambda_t$ is independent of $i\in [d_{in}]$: owing to symmetry the same number of paths start from any given input node, and looking forward, the paths see exactly the same gates and gating values in the subsequent layers.

%\subsection{First-order feature and kernel}
\textbf{Activity Derivative:} For a path $p$, and an input $x\inrdin$ the derivative of its activity with respect to any weight is given by 
\begin{align}
\frac{\partial A_{t}(x,p)}{\partial \tg}= \sum_{l=1}^{d-1} \Big(\frac{\partial G_{x,\Tg_t}(l,\I_l(p))}{\partial \tg} \Big)\Big(\Pi_{l'\neq l} G_{x,\Tg_t}(l',I_{l'}(p))\Big)
\end{align}
\end{comment}

\begin{comment}
Given that the output of a DGN is expressed as $\hat{y}_t(x)=\phi_{x,t}^\top v_t$, we define the following:\\
$1.$ \textbf{Value gradient:} $\psi_{x,t}^v=(\nabla_{\Theta}v_t)^\top  \phi_{x,t}$, which learns the value of the paths keeping the NPF fixed. \\
$2.$ \textbf{Feature gradient:}  $\psi_{x,t}^{\phi}=(\nabla_{\Tg}\phi_{x,t})^\top v_t $, which learns the NPFs keeping the value of the paths fixed. Note that the feature gradient is $0$ in the case of hard-gates.
To the best of our knowledge, for the first time in the literature, we note that the gradient flow has two components. Note that, in the value and feature gradients, quantities $\nabla_{\Theta}v_t$ and $\nabla_{\Tg}\phi_{x,t}$ are $P\times d_{net}$ matrices, which are obtained by computing the derivative of $v_t(p)$ and $\phi_{x,t}(p)$ for all paths $p\in[P]$ with respect to the respective $d_{net}$ parameters.\\
\end{comment}

\begin{comment}
\textbf{Neural Tangent Feature (NTF):} $\psi_{x,t}=[\psi^v_{x,t},\psi^{\phi}_{x,t}]\in \R^{2d_{net}}$. \\
\textbf{Neural Tangent Kernel (NTK):} $K_t(x,x')=K^v_t(x,x')+K^{\phi}_t(x,x')$, where $K^v_t(x,x')={\psi^v}^\top_{x,t}\psi^v_{x',t}$ and $K^{\phi}_t(x,x')={\psi^{\phi}}^\top_{x,t}\psi^{\phi}_{x',t}$. 
\begin{assumption}\label{assmp:decouple}
$\G_0$ is statistically independent of $\Theta_0\inrdnet$ or in the case of parameterised gating, $\Tg_0\inrdnet$ is statistically independent of $\Theta_t\inrdnet$.
\end{assumption}
\end{comment}

\begin{definition}\label{def:delta}
$\delta_t(x,x')\stackrel{def}= \underset{{p\rsa i}}{\sum} \sum_{\tg\in\Tg}\frac{\partial A_{t}(x,p)}{\partial \tg} \frac{\partial A_{t}(x',p)}{\partial \tg}$, for $x,x'\in\R^{d_{in}}$, using any $i\in[d_{in}]$.
\end{definition}
%Note that in \Cref{def:delta}, $\delta_t$ contains $\partial A_t(x,p)$ terms as opposed to $A_t(x,p)$ term in $\lambda_t$ defined in \Cref{def:lambda}.
\begin{theorem}\label{th:main} Let $x=(x_s,s\in[n])\in \R^{d_{in}\times n}$ be the data matrix. Then, under \Cref{assmp:main}, we have:\\
(i) $\E{K_0}=\mathbb{E}\big[K^v_0\big]+\E{K^{\phi}_0}$, where $\E{K^{v}_0}=\sigma^{2(d-1)} (x^\top x)\odot \lambda_0$, and $\E{K^{\phi}_0}=\sigma^{2d}  (x^\top x)\odot \delta_0$.
\end{theorem}
\textbf{Discussion:}\\
$1.$ \emph{Sensitive Sub-Network and Gradient Flow:} For input $x\in\R^{d_{in}}$, let $\P^{\A}_{x,t}(\tau_{\A})=\{p\in[P]:A_t(x,p)>\tau_{\A}\}$ be the set of paths whose activity is greater than some threshold value $\tau_{\A}>0$. In the case of ReLU networks, $\P^{\A}_{x,t}(0)$ completely determines the NPF $\phi_{x,t}$. For an input $x\in\R^{d_{in}}$, let $\P^{\S}_{x,t}(\tau_{\S})=\{p\in[P]: |\partial_{\tg} A_t(x,p)|\tau_{\S}\}$ be the set of paths, which have activations whose gradient to any of $\Tg\inrdnet$ is greater than some threshold $\tau>0$. Using the property that the slope of the sigmoid diminishes in the extremities, we note that for appropriate choices of $\tau_{\A}$ (sufficiently close to $1$) and $\tau_{\S}$ (large enough), $\P^{\S}(\tau_{S})\cup \P^{\A}(\tau_{\A})$.\\
$2.$ \emph{Open Questions:} We hypothesise the following:\\
(i) Feature learning helps in increasing margin: Considering the fact that $\hat{y}_t=\Phi_{\Tg_t}v_{\Theta_t}$, the gradient with respect to $\Tg$ will change the NPF matrix $\Phi_{\Tg_t}$ in such a manner to reduce the loss, i.e., increase the margin of each of the classified examples. \\
(ii) Subject to the `well-conditioned' ness of $K^{\phi}_t$, such margin increase can be perhaps achieved for all the $n$ examples: In the above example, even though the gating parameters had $2$-degrees of freedom, it was nonetheless sufficient to adapt the features. For $n$ examples, we might require $n$ degrees of freedom to learn the features, which is captured by the conditioning of $K^{\phi}_t$.\\

%\begin{abstract}
We study optimisation and generalisation in deep gates networks (DGNs), wherein, the output of a single hidden neuron is obtained as a product of its pre-activation input and a gating value. DNNs with ReLU activations are special cases of DGNs. Central idea in this paper is to regard paths and gates as basic building blocks of a DGN. At time $t$, for an input $x\in\R^{d_{in}}$ to the DGN, we define a novel \emph{neural path feature} (NPF) whose dimension is equal to total the number of paths, and whose co-ordinate corresponding to a path is the product of the signal at its input node  and the activity\footnote{A path is active if all the gates in the path are active.} of the path. We show that the associated \emph{neural path kernel} (NPK) plays an important role in optimisation and generalisation of DGNs. %In particular, $H_t(x,x')=(x^\top x')\lambda_t(x,x')$, where $\lambda_t(x,x')$ is a count of total number of paths that are active simultaneously for both inputs $x,x'\in \R^{d_{in}}$. \hfill\\
For randomly initialised DGNs optimising with fixed NPFs, we show that i) increasing depth till a point helps in training and ii) increasing depth beyond hurts training. We verify via experiments that the NPFs are learned during training by showing that the norm of the labelling function measured with respect to the inverse of the trace normalised NPK reduces with time.
\end{abstract}

%\section{Introduction}
Understanding optimisation and generalisation in deep neural networks (DNNs) trained using first-order method such as gradient descent (GD) is an important problem in machine learning. Despite having a non-convex loss surface, GD achieves zero training error in over-parameterised DNNs where the number of parameters exceeds the size of the dataset. Interestingly, \cite{ben} demonstrated that practical DNNs have enough capacity to achieve zero training loss with even random labelling of standard datasets such as MNIST and CIFAR. However, when trained with true labels, such networks achieve zero training error and also exhibit good performance on test data. 

In this paper, we consider fully connected DNNs with ReLU activations, with $d$ layers and $w$ hidden units per layer. In what follows ,we denote the dataset by $(x_s,y_s)_{s=1}^n\in \R^{d_{in}}\times \R$, and the parameters of the network by $\Theta\in \R^{d_{net}}$, the network output for an input $x\in \R^{d_{in}}$ by $\hat{y}_{\Theta}(x)$.

Some of the recent works (\citenum{dudnn,du2018,arora2019exact,dudln}) on this problem have made use of the \emph{neural tangent features} (NTFs) and the \emph{trajectory based analysis}, which we describe in brief. The NTF of an input $x\in \R^{d_{in}}$ is defined as $\psi_{x,\Theta}=\left(\partial_{\theta}\hat{y}_{\Theta}(x),\theta\in\Theta\right)\in\R^{d_{net}}$\footnote{Here $\partial_{\theta}(\cdot)$ stands for $\frac{\partial (\cdot)}{\partial \theta}$}, i.e., the gradient of the network output with respect to the weights. By collecting the NTFs of all the inputs examples in the dataset, we can form the NTF matrix $\Psi=(\psi_{x_s,\Theta},s\in[n])\in\R^{d_{net}\times n}$\footnote{$[n]=\{1,\ldots,n\}$}. The trajectory based analysis looks at the dynamics of the error $e_t=(\hat{y}_{\Theta_t}-y_s,s\in[n])\in\R^n$. For a small enough step-size $\alpha_t>0$ of the GD procedure, the error dynamics is given by:
\begin{align}
e_{t+1}=e_t-\alpha_t K_{\Theta_t} e_t,
\end{align}
where $K_{\Theta_t}\in\R^{n\times n}$ is the \emph{neural tangent kernel} (NTK) matrix give by $K_{\Theta}=\Psi^\top_{\Theta} \Psi_{\Theta}$. Thus, the spectral properties, and in particular, $\rho_{\min}(K_{\Theta_t})$ the minimum eigenvalue of $K_{\Theta_t}$ dictates the rate of convergence. Under randomised initialisation, and in the limit of infinite width an interesting property emerges: the parameters of the DNN deviate very little during training, i.e. $\Theta_t\approx \Theta_0$. In particular, $K_{\Theta_t}\approx K_{\Theta_0}$, $K_{\Theta_0}\ra K^{(d)}$, i.e., the NTK stays almost constant through training and the NTK matrix at initialisation $K_{\Theta_0}$ converges to a deterministic matrix $K^{(d)}$ (see \Cref{sec:exact} for exact expression of $K^{(d)}$. Thus in the `large-width' regime, zero training error can be achieved if $\rho_{\min}(K^{(d)})>0$ which holds as long as the training data is not degenerate (\citenum{dudnn,arora2019exact}). In the `large-width' regime, \cite{arora2019exact} show that the fully trained DNN is equivalent to kernel regression with $K^{(d)}$. Hence, a trained DNN enjoys the generalisation ability of its corresponding $K^{(d)}$ matrix in the `large-width' regime. \cite{cao2019generalization} show that in the `large-width' regime, the DNN is almost a linear learner with the random NTFs, and showed a generalisation bound in the form of $\tilde{\mathcal{O}}\left(d\cdot\sqrt{y^\top K^{(d)} y/n}\right)$\footnote{$a_t=\mathcal{O}(b_t)$ if $\lim\sup_{t\ra\infty}|a_t/b_t|<\infty$, and $\tilde{\mathcal{O}}(\cdot)$ is used to hide logarithmic factors in $\mathcal{O}(\cdot)$.}, where $y=(y_s,s\in[n])\in\R^n$ is the labelling function.

\textbf{Research Gap I (Feature Learning):} In the `large-width' i.e., fixed NTF/NTK regime, the DNNs are linear learner using the random NTFs at random initialisation. This implies that there is little or no feature learning. Is this true?\\
\textbf{Research Gap II (Finite vs Infinite):} \cite{arora2019exact} note that, while pure-kernel methods based on the limiting NTK (i.e., $K^{(d)}$) outperform other state-of-the-art kernel methods, the finite width DNNs (CNNs) still outperform their NTK (CNTK)\footnote{CNTK: Convolutional Neural Tangent Kernel, the NTK for CNNs} counterpart. Can we explain this gap?
\subsection{Contributions}
$\bullet$ (\Cref{sec:pathgate}) Central idea in this paper is the 'path-view', wherein, we decompose the computation in such DNNs into paths and gate. Intuitively speaking, when a particular example is presented to the network only a sub-set of the activations \emph{on} and the output is obtained as the sum of the contribution of the various paths in the sub-network formed by such \emph{on} gates. We explicitly encode the state of the gates in a novel \emph{neural path feature} (NPFs), which is zeroth-order feature. Using the `path-view' we characterise the information flow via two different sub-networks, i) of active gates which are on and ii) of sensitive gates which are on the verge of turning on or off. We also show the NTK can be decomposed into $K_{\Theta}=K^v_{\Theta]+K^{\phi}_{\Theta}+M$, where $K^v_{\Theta}$ is the NTK of values responsible for optimisation with fixed features, and $K^{\phi}_{\Theta}$ is the NTK of features responsible for feature learning, and $M$ is a symmetric matrix. \\
$\bullet$ We study a novel fixed NPF regime. Here, we show that as width increases to infinity the optimisation and generalisation properties are captured by the \emph{neural path kernel} (NPK), the kernel associated with the NPFs. In the case of finite width,  we show that when optimising  using GD i) increasing depth till a point helps in training and ii) increasing depth beyond a point hurts training.\\
$\bullet$ We show via experiments, that NPFs are learnt during training, and such learning improves generalisation performance.



\textbf{Organisation:} In \Cref{sec:pathgate}, we introduce the `path-view', wherein, we decompose the computation in a DNN in terms of the paths and the gates. Using the `path-view', we define a novel \emph{neural path feature} (NPF) and a novel \emph{neural path value} (NPV), and express the output as an inner product of the NPV and NPF. In \Cref{sec:our} we  discuss how `path-view' helps in capturing feature learning, and in particular, show the NTK can be decomposed into $K_t=K^v_t+K^{\phi}_t+M$, where $K^v_t$ is the NTK of values responsible for optimisation with fixed features, and $K^{\phi}_t$ is the NTK of features responsible for feature learning, and $M$ is a symmetric matrix. 
In \Cref{sec:optimisation}-\Cref{th:main}, we consider optimisation and generalisation with fixed NPFs. In \Cref{sec:generalisation},  we show that NPFs obtained from a pre-trained network generalises better than NPFs from a randomly initialised network. This implies that the NPFs are learned during training. Since \Cref{th:main} allows us to use $H_0$ in the place of $K_0$ in the generalisation bounds \cite{cao2019generalization}, we verify that the norm of the labelling function measured with respect to the inverse of the trace normalised NPK reduces with time.\\
\begin{comment}\textbf{Lesson Learnt:} The `path-view' enables us to model and study the dynamics of the gates, and hence the learning of NPFs. Based on the theory and experiments, we can say that \emph{``understanding deep learning requires understanding the dynamics of the gates".}\end{comment}
\begin{comment}
\textbf{Lesson Learnt:} Gradient is a first-order information, and learning with GD is essentially a process of integration, and naturally its outcome is a zeroth-order term. It turns out that splitting this zeroth-order term into NPFs and NPVs is quite useful, in particular, this enables us to identify the role of NPFs. \emph{Understanding deep learning requires understanding the dynamics of the gates.}
\end{comment}
\begin{comment}
$1.$ \textbf{Deep Gated Networks (DGNs):} Here, the output of a single hidden neuron is obtained as a product of its pre-activation input ($\in \R$) and a gating value ($\in [0,1]$ ). The DGNs we consider consists of two networks i) a value network which holds the pre-activation input, layer output and layer weights $\Theta\in \R^{d_{net}}$, and the final output, and ii) a separate gating network (with its own internal pre-activation inputs, layer output, and layer weights $\Tg\in \R^{d_{net}}$), which provides the gating values. 
DNN with ReLU activation is a special case, wherein, the value network and gating network are \emph{coupled} i.e., $\Theta=\Tg$.\\
$2.$ \textbf{Neural Path Feature and Kernel (NPF and NPK):} We express the output as the sum total of the contributions from various paths. A \emph{path} starts from an input node, passes through exactly one weight and one activation in each layer, and finally ends at the output node. Let $[P]=\{1,\ldots,P\}$ be an enumeration of all the paths, then the contribution of a path in a DGN is the product of the signal at the input node, all the gating values of the hidden neurons and all the weight through which it passes. At time $t$, and for an input $x\in\R^{d_{in}}$,  the output of a DGN is recovered as $\hat{y}_t(x)=\ip{\phi_{x,t},v_t}$, where $\phi_{x,t}\in \R^P$ is the \emph{neural path feature} (NPF). Here, for a path $p\in[P]$, $\phi_{x,t}(p)$ holds the product of the signal at its input node and all its gating values, and  $v_t=(v_t(p),p\in[P])\in\R^P$, with $v_t(p)$ holding the product of the weights in path $p$. The associated \emph{neural path kernel} (NPK) matrix $H_t$ has a special \emph{Hadamard} structure, i.e., $H_t=\Sigma \odot \lambda_t$, where $\Sigma$ is the data \emph{Gram matrix}, $\lambda_t(x,x')$ measure the overlap of active sub-networks for inputs $x,x'\in\R^{d_{in}}$, and $\odot$ is the Hadamard product.\\
$3.$ \textbf{Gate Copying:} Contrary to NTK based works \cite{jacot18,jacot19,cao2019generalization,arora2019exact}, wherein, the DNNs are linear learners using fixed NTFs at random initialisation, the DGN framework enables the study the optimisation and generalisation with different NPFs. Two interesting cases arise when NPFs are obtained from a gating network which is a i)  pre-trained DNN, and ii) randomly initialised DNN.\\
$4.$ \textbf{Optimisation:} We consider DGN training in the fixed gating regime, where, the NPFs in the gating network are kept fixed through training, and only the path values are trained. When the weights are initialised statistically independent of the NPFs, we show, in \Cref{th:main}, that as width approaches $\infty$, $K_0\ra C H_0$ (for a constant $C>0$). This result implies that in the limit of infinite width, increasing depth leads to whitening of $\lambda_0$ and thus helps in training, and for finite width increasing depth beyond a point hurts training variance in entries of $K_0$ which depend on $d$. \\
$5.$ \textbf{Generalisation:}  We show (on MNIST and CIFAR datasets) that DGN trained with fixed NPFs obtained from a pre-trained gating network generalise better than DGNs trained with fixed NPFs obtained from a randomly initialised gating network. We verify that (experimentally with a `Binary'-MNIST dataset) that the norm of the label function measured with respect to the normalised NPK matrix reduces during training. We note that using \Cref{th:main} to replace $K_0$ with $H_0$ in the generalisation bounds of \cite{cao2019generalization}, would provide the theoretical justification of why NPFs from pre-trained gating networks generalise better.\\
$6.$ \textbf{Feature Learning:} The gradient $\partial \hat{y}$ has two components i) a \emph{value gradient} term $(\partial v)^\top \phi$ which learns path values with fixed NPFs and ii) a \emph{feature gradient} term $(\partial \phi)^\top v$ which learns the features themselves. The value gradient flows through the sub-network of \emph{active} paths, and the feature gradient flows the sub-network of \emph{sensitive} paths which contain gates that are in the verge of switching \emph{on/off}. In particular, when $\Tg$ and $\Theta$ are initialised statistically independent of each other, we show that $\E{K_0}=(C\Sigma\odot\lambda_0+C'\Sigma\odot\delta_0)$, where $\delta_0(x,x')$ is a measure of overlap of sensitive sub-networks for inputs $x,x'\in\R^{d_{in}}$.
\end{comment}
\begin{comment}
Say $[P]=\{1,\ldots,P\}$ be an enumeration of the paths, 
$1.$ At time $t$, for an input $x\in\R^{d_{in}}$, we define a novel (zeroth-order) \emph{neural path feature} (NPF) $\phi_{x,t}\in \R^P$, where $P$ is the total number of paths, and for a path $p$, $\phi_{x,t}(p),\p\in[P]$\footnote{$[P]=\{1,\ldots,P\}$} is equal to the product of the signal at its input node and its activity. The output can be written as $\hat{y}_t(x)={\phi^\top_{x,t}} v_t$, where $v_t=(v_t(p),p\in[P])\in\R^P$ is the vector of path values with $v_t(p)$ given by the product of the weights in the path.\\
%\text{(First Order)}\quad: \quad \partial \hat{y}_t(x)&= \sum_{p\in [P]}x(\I(p)) A_t(x,p) \partial((v_t(p))+\sum_{p\in [P]}x(\I(p)) \partial(A_t(x,p)) v_t(p)\nn\\
%\label{eq:first}&\text{(First- Order/Neural Tangent Feature)}&\quad:&\quad   \nabla \hat{y}_t(x)=\psi_{x,t}=(\nabla v_t )^\top \phi_{x,t} + (\nabla \phi_{x,t})^\top v_t,
$2.$ The associated \emph{neural path kernel} (NPK) matrix $H_t$ has a special \emph{Hadamard} structure, i.e., $H_t=\Sigma \odot \lambda_t$, where $\Sigma$ is the data \emph{Gram matrix}, $\lambda_t(x,x')$ measure the overlap of active sub-networks for inputs $x,x'\in\R^{d_{in}}$, and $\odot$ is the Hadamard product.\\
$3.$ Say $\Theta$ are the network weights, now, since the NPFs are solely based on the gating information, the NPFs can be obtained from a separate gating network, whose weights are $\Tg\neq\Theta$. The NPFs could be derived from i) a randomly initialised gating network or ii) a pre-trained gating network. This way we can study optimisation and generalisation of different kinds of NPFs, which is in sharp contrast with the NTK based works \cite{jacot18,jacot19,cao2019generalization,arora2019exact}, wherein, the DNNs are linear learners using fixed NTFs at random initialisation.\\
$4$. The gradient $\partial \hat{y}$ has two components i) a $(\partial v)^\top \phi$ which learns path values with fixed NPFs and ii) a $(\partial \phi)^\top v$ which learns the features themselves.\\
\textbf{Main Results:} We present the analysis and experiments in two regimes namely the fixed gating regime, wherein, the NPFs are fixed and the adaptable gating regime, wherein, the NPFs are learnt.\\
$\bullet$ \textbf{Optimisation:} When the gates are fixed, under appropriate statistical assumptions, we show that (\Cref{th:main}) as width approaches $\infty$, $K_0\ra C H_0$ (for a constant $C>0$). This result implies that in the limit of infinite width, increasing depth leads to whitening of $\lambda_0$ and thus helps in training, and for finite width increasing depth beyond a point hurts training variance in entries of $K_0$ which depend on $d$. \\
$\bullet$ \textbf{Generalisation:} We show via experiments (on standard MNIST and CIFAR datasets) that DGNs with fixed NPFs derived from a pre-trained gating network generalise better than fixed NPFs obtained from a randomly initialised gating network. \\
We show on an experiment in `Binary'-MNIST dataset that the norm of the label function measured with respect to the normalised NPK matrix reduces during training.
$\bullet$ 
\end{comment}
\begin{comment}
$1.$ \textbf{Information Flow:} We introduce a host of novel terms/expressions and concepts related to information flow in DGNs. This includes i) the zeroth-order (see \eqref{eq:zero}) \emph{neural path feature} (NPF) and the \emph{neural path kernel} (NPK) denoted by $H_t$, which are defined using the gating information, ii) the first-order (see \eqref{eq:first}) terms namely, the \emph{value gradient} given by $(\nabla v_t )^\top \phi_{x,t}$, and the \emph{feature gradient} given by $(\nabla \phi_{x,t})^\top v_t$, iii)  two sub-network of paths namely, an \emph{active} sub-network, which is the set of active paths that hold the memory for an input, through which the value gradient flows and a \emph{sensitive} sub-network, which is the set of sensitive paths, through which the feature gradient flows. We also show that for random initialisation, the paths exhibit a \emph{disentanglement} property, which is very key in our results.\\
%$1.$ \textbf{Representation and Generalisation:} By fixing the activation (i.e., $A_t=A_0,\forall t\geq 0$), we train DGNs with different fixed \emph{neural path features} (NPFs), and demonstrate that the NPFs are key for generalisation. Further, we show that the NPFs are learnt during training. This shows that NPFs can be regarded as the true hidden features in a deep network.\\
\end{comment}
\begin{comment}
$3.$ \textbf{Neural Path Features} are the information stored in the gates of a deep network. Our main result connects the NPK to the NTK, i.e., we show that, when the weights are randomly initialised (with an appropriate scale) in a manner statistically independent of the gates, i)  $\E{K_0}=H_0$, and ii) $Var\left[K_0(x,x')\right]\leq O(\max\{\frac{d^2}{w}, \frac{d^3}{w^2}\})$.\\
$\quad\bullet$ \textbf{Optimisation:}   We argue that  $H_0$ whitens with depth, and hence, increasing depth helps in training. Increasing depth further causes $K_0$ to deviate from $H_0$, and hence, degrades the spectrum of $K_0$, thereby affecting training performance.\\
$\quad\bullet$ \textbf{Generalisation:} By fixing the activation (i.e., $A_t=A_0,\forall t\geq 0$), we train DGNs with different fixed NPFs, and demonstrate that the NPFs are key for generalisation. Further, we show that the NPFs are learnt during training. This shows that NPFs can be regarded as the true hidden features in a deep network.\\
$3.$ \textbf{NPF Learning:} We present preliminary results connecting NPF learning and the trajectory method.
\end{comment}
\begin{comment}
Understanding optimisation and generalisation of deep neural networks (DNNs) trained using first-order method such as (stochastic) gradient descent is an important problem in machine learning. In this paper, we throw light on the following two questions:\hfill\\
\textbf{Question I (Optimisation):} \emph{What is the role of width and depth in training of DNNs? Why increasing depth till a point helps in training? Why increasing the depth beyond hurts training?}\\
\textbf{Question II (Generalisation):} \emph{What are the hidden features in a DNN? Are these features learned? and if so, how?} \hfill\\

\textbf {Background:} We consider a fully connected deep gated network (DGN) of depth $d$, and width $w$, weights $\Theta\in\R^{d_{net}}$, which, accepts an input $x\in \R^{d_{in}}$ and produces an output $\hat{y}_{\Theta_t}(x)\in \R$. In a DGN, the output of a hidden neuron is obtained as a product of its pre-activation input and a gating value. DNNs with ReLU activations are special cases DGNs. Some of the recent works to understand SGD in relation to the optimisation and generalisation in DNNs make use of the \emph{neural tangent features} (NTFs) and the \emph{trajectory} based analysis, which we describe in brief.\hfill\\
$1.$The NTF for an input $x\in \R^{d_{in}}$ is defined as $\psi_{x,t}=(\frac{\partial \hat{y}_{\Theta}}{\partial \theta}|_{\Theta=\Theta_t},\theta \in \Theta)\in \R^{d_{net}}$, i.e., collection of the gradients of the network output with respect to its weights. Since the NTF is a first-order term, it can be used to linearise the DNN output about the point $\Theta_t$. Associated with the NTF is a \emph{neural tangent kernel} (NTK), which is given by $K_t(x,x')=\psi^\top_{x,t}\psi_{x',t}$. 
$2.$ In the \emph{trajectory} based analysis, one looks at the dynamics of error $e_t$  (i.e., difference between predicted and true values) at time $t$. For a small step-size $\alpha_t>0$, the error dynamics follows a linear recursion given by: $e_{t+1}=e_t-\alpha_tK_te_t$, where $K_t$ is the NTK matrix obtained on the the dataset. Thus, the spectral properties of $K_t$ is key to achieve zero training error. \hfill\\
\textbf{Related Work and Gaps:} \hfill\\
\emph{Optimisation:} \cite{ntk} were the first to point out the role of NTK in DNNs. Using the trajectory based analysis, \cite{dudnn} show that in fully connected DNNs with $w=\Omega(poly(n)2^{O(d)})$, and in residual neural networks (ResNets) with $w=\Omega(poly(n,d))$ gradient descent converges to zero training loss.  However, Question I, i.e., `why depth helps in training?' is unresolved.\hfill\\%This shows that ResNets are better than fully connected DNNs, based on the fact that the dependence on the number of layers improves exponentially for ResNets. 
\emph{Generalisation:} \cite{arora2019exact,arora,cao2019generalization} use NTK to provide generalisation bounds as well as propose pure-kernel methods. However, couple of issues remain unresolved: firstly, if the DNNs are only linear learners with random NTFs, then it suggests that no feature learning happens in DNNs, and secondly, it was observed in prior experiments that the DNNs perform better than their corresponding NTK counterparts \cite{arora2019exact,lee2019wide}. \hfill\\

\textbf{Our Results:} We now highlight the contributions in this paper.\\
$1.$ \textbf{Neural Path Feature:} Central idea in this paper is the `path-view': to regard paths and gates as basic building blocks of a DGN.  A \emph{path} starts from an input node, passes through exactly one weight and one activation in each layer, and finally ends at the output node. Let $[P]=\{1,\ldots,P\}$ be an enumeration of all the paths, then the \emph{neural path feature} (NPF) is defined $\phi_{x,t}=(x(\I(p))A_t(x,p),p\in[P])\in\R^P$, wherein, for a path $p$, $\I(p)$ is the input node at which it starts, and  $A_t(x,p)$ is its activation level is equal to the product of the gating values in the path. Using the NPF, the output is expressed as $\hat{y}_t(x)=\phi^\top_{x,t}v_t$, where, $v_t=(v_t(p),p\in[P])\in \R^P$, with $v_t(p)$ is the value of a path $p$ and is equal to the product of its weights.\hfill\\
$2.$ \textbf{Neural Path Kernel (NPK)}  is given by $H_t(x,x')=\phi^\top_{x,t}\phi_{x',t}$, where $x,x'\in\R^{d_in}$ are the inputs to the DGN. We show that the NPK whitens as depth increases.\hfill\\
$3.$ \textbf{Optimisation:}  When weights are initialised at random (with variance $\sigma$) and statistically independent of the activations, we show that $\E{K_0}= CH_0$, (for some constant $C>0$) and  $Var\left[K_0(x,x')\right]\leq O(\max\{\frac{d^2}{w}, \frac{d^3}{w^2}\})$ (for $\sigma^2=O(\frac{1}w)$). Thus,\\
(i) For a fixed $d$, increasing $w$ ensures $K_0$ converges to $\E{K_0}$, and since NPK matrix whitens as depth increases, increasing depth till a point helps in training performance.\hfill\\
(ii) For a fixed $w$, increasing $d$ makes the entries of $K_0$ deviate from $\E{K_0}$, thus degrading the spectrum of $K_0$. Thus increasing depth beyond a point hurts training performance. \hfill\\
$4.$ \textbf{Feature Learning:} Since $\hat{y}_t(x)=\phi^\top_{x,t}v_t$, the gradient flow has two components: an $(\nabla v_t)^\top\phi_{x,t}$ term, which we call the \emph{value gradient}, learns the path values (which are akin to weight vector in a standard linear approximation) keeping the NPFs fixed, and a $(\nabla \phi_{x,t})^\top v_t $ term, which we call the \emph{feature gradient} learns the NPFs keeping the path values fixed. We present preliminary theory connecting the trajectory method and feature learning. \hfill\\
$5.$ \textbf{Generalisation:} We experiment with two gating regimes namely the static, wherein, $A_t=A_0,\forall t \geq 0$, and the dynamic, wherein $A_t$ changes during training. The experiments on MNIST and CIFAR show that, better generalisation happens when the activations $A_t$ change with time. In particular, for binary classification, we verify that the quantity $y^\top (\widehat{H_t})^{-1}y$\footnote{For a matrix $H$, $\hat{H}=\frac{1}{trace(H)}H$ .} reduces as the training progresses. This shows that the eigenspaces of the NPK matrix align themselves in accordance to the labelling function. These experiments supports the fact that NPF and the NPK are indeed learnt during training in a manner to improve the generalisation performance. \hfill\\
\end{comment}
\begin{comment}
\textbf{Paths:}  A \emph{path} starts from an input node, passes through exactly one weight and one activation in each layer, and finally ends at the output node. Let $[P]=\{1,\ldots,P\}$ be an enumeration of all the paths, the zeroth and first order quantities of a DNN can then be expressed as:
\begin{align}
\label{eq:zero}\text{(Zeroth Order)}\quad: \quad \hat{y}_t(x)&=\sum_{p\in [P]}x(\I(p))A_t(x,p)v_t(p)={\phi^\top_{x,t}} v_t\\
\text{(First Order)}\quad: \quad \partial \hat{y}_t(x)&= \sum_{p\in [P]}x(\I(p)) A_t(x,p) \partial((v_t(p))+\sum_{p\in [P]}x(\I(p)) \partial(A_t(x,p)) v_t(p)\nn\\
\label{eq:first}\psi_{x,t}=(\partial v_t)^\top\phi_{x,t}+ (\partial \phi_{x,t})^\top v_t,
\end{align}
where, for a path $p$, $\I(p)$ is the input node at which it starts, $v_t(p)$ is its value given by the product of its weights and for input $x\in\R^{d_{in}}$, $A_t(x,p)$ is its activation level given by the product of the gating values in the path. \hfill\\
\textbf{Independent Gates and Activations:} In order to understand the role $A_t$, we handle the gating values as independent variables\footnote{In DNN with ReLU, the gates are implicit: a gate is \emph{on} only if pre-activation input is positive.}. All the theoretical results are under the assumption that $A_0$ is statistically independent of $\Theta_0$. \hfill\\
\textbf{Neural Path Feature and Kernel:} The \emph{neural path feature} (NPF) is given by $\phi_{x,t}=(x(\I(p))A_t(x,p))\in \R^P$ (see \eqref{eq:zero}) , and an associated \emph{neural path kernel} (NPK) to $H_t(x,x')=\phi^\top_{x,t}\phi_{x',t}$. The NPK has special structure: $H_t(x,x')=(x^\top x')\lambda_t(x,x')$, where $\lambda_t(x,x')$ is a measure of similarity based on the path activation levels for inputs $x,x'\in\R^{d_{in}}$. \hfill\\
\textbf{NPK vs NTK: } Under symmetric Bernoulli initialisation, we have $\E{K_0} = C H_0$, ($C>0$ is a constant), where $K_0$ is the NTK matrix at $t=0$. \hfill\\
\textbf{Optimisation I:} We argue that $\frac{\lambda_0(x,x')}{\lambda_0(x,x)}$ decays at an exponential rate with depth, i.e., the Gram matrix $K_0$ whitens with depth. Thus increasing depth helps in training. \hfill\\
\textbf{Optimisation II:} We show that $Var\left[K_0(x,x')\right]\leq O(\max\{\frac{d^2}{w}, \frac{d^3}{w^2}\})$ (for $\sigma^2=O(\frac{1}w)$). For a fixed $d$, increasing $w$ ensures $K_0$ converges to $\E{K_0}$. However, for a fixed $w$, increasing $d$ makes the entries of $K_0$ deviate from $\E{K_0}$, thus degrading the spectrum of $K_0$. Thus increasing depth beyond a point hurts training performance. \hfill\\
\textbf{Generalisation:} We experiment with two gating regimes namely the static, wherein, $A_t=A_0,\forall t \geq 0$, and the dynamic, wherein $A_t$ changes during training. The experiments on MNIST and CIFAR show that, better generalisation happens when the activations $A_t$ change with time. In particular, for binary classification, we verify that the quantity $y^\top (\widehat{H_t})^{-1}y$\footnote{For a matrix $H$, $\hat{H}=\frac{1}{trace(H)}H$ .} reduces as the training progresses. This shows that the eigenspaces of the NPK matrix align themselves in accordance to the labelling function. These experiments supports the fact that NPF and the NPK are indeed learnt during training in a manner to improve the generalisation performance. \hfill\\
\end{comment}
\section{Optimisation: The Role of Depth}\label{sec:optimisation}
\begin{comment}
Consider a DNN with ReLU activation at time $t$. Now, by collecting the gating values for the $n$ examples in the dataset in $\G_t$, we can define the NPF matrix $\Phi_t=(\phi_{x_s,t},s\in[n] )\in \R^{d_{net}\times n}$. In the case of say \emph{squared-loss}, GD considers the following problem:
\begin{align}
\min_{\Theta\inrdnet}\norm{\Phi^\top_tv_{\Theta_t+\Theta}-y}_2^2,
\end{align}
\end{comment}
The error trajectory of the GD is then given by $e_{t+1}=e_t-\alpha_t K_te_t$, where $e_t=(\hat{y}_t(x_s)-y_s,s\in[s])\in\R^n$, and $K_t=\Psi^\top_t\Psi_t$, with $\Psi_t$ being the $d_{net}\times n$ NTF matrix. Now, $\Psi_t$ is given by $\Psi_t=\left(\nabla_{\Theta|_{\Theta=\Theta_t}v_t}\right)^\top\Phi_t$, where $\nabla_{\Theta}v_t=$ is a $P\times d_{net}$ matrix of $\partial_{\theta} v_t(p),\forall p\in [P], \theta\in \Theta$. Putting together, we have:
\begin{align*}
K_t=\Psi^\top_t\Psi_t=\Phi^\top_t\left(\nabla_{\Theta|_{\Theta=\Theta_t}v_t}\right)\left(\nabla_{\Theta|_{\Theta=\Theta_t}v_t}\right)^\top\Phi_t
\end{align*}
At $t=0$, using \Cref{assmp:init,assmp:decouple}, we can obtain the following simplified expression
\begin{align}\label{eq:opti-refer}
\E{K_0}=\Phi^\top_t\E{\left(\nabla_{\Theta|_{\Theta=\Theta_t}v_t}\right)\left(\nabla_{\Theta|_{\Theta=\Theta_t}v_t}\right)^\top}\Phi_t=d\sigma^{2(d-1)}(x^\top x)\odot \lambda_0
\end{align}
\textbf{Role of Randomisation:} Note that, in \Cref{assmp:decouple} helps us in pulling out the $\Phi^\top_t$ and $\Phi_t$ terms outside, and \Cref{assmp:init} disentangles the path as per \Cref{lm:disentangle}, due to which the inner expectation of $\nabla_{\Theta}v\nabla^\top_{\Theta}v$ reduces to a $P\times P$ identity matrix times a constant $d\sigma^{2(d-1)}$.\\
\textbf{Active Sub-Network and Gradient Flow:}  Each input example has its own associated set of active sub-network, and while training a particular example, the gradient flows through the weights of the corresponding active sub-network. Now, the active sub-networks corresponding to different examples have some overlap, and hence there is bound to be \emph{cross-talk} of the gradients flowing through them. This overlap is captured by $\lambda_t(s,s')$ which is the measure of overlap of the sub-networks that are active for both the inputs $x,x'\in\R^{d_{in}}$. As seen from \Cref{th:main}, $\lambda_0$ directly controls the spectral properties of the NTK matrix $K_0$.\\
\textbf{Why increasing depth till a point helps in training? } From \Cref{th:main}-(ii) it follows that for $w\ra\infty$, $K_0\ra\E{K_0}$. We now argue that when $\sigma=\sqrt{\frac{2}{w}}$, increasing depth causes whitening of $\lambda_0$, and hence $K_0$ .\hfill\\
$1.$ Let us first look at the diagonal terms of $\lambda_0$. It is reasonable to assume that, owing to the symmetric nature of the weights, roughly $\mu=\frac{1}{2}$ fraction of the gates are \emph{on} every layer. Thus $\lambda_0(s,s)\approx (w/2)^{d-1}$. Now, due our choice of $\sigma=\sqrt{\frac{2}{w}}$, the diagonal entries will be close to $1$.\hfill\\
$2.$ We now turn our attention towards the non-diagonal entries of $\lambda_0$. Define $\tau(s,s',l)\stackrel{def}=\sum_{i=1}^w G_{x_s,t}(l,i)G_{x_{s'},t}(l,i)$ be the overlap of the active gates in layer $l$ for input examples $s,s'\in[n]$, and  let $\eta\stackrel{def}=\max_s\left(\max_{s',l} \frac{\tau(s,s',l)}{\tau(s,s,l)}\right)$ be the maximum overlap between gates of a layer (maximum taken over over input pairs $s,s'\in[n]$ and layers $l\in [d]$).  Then it follows that $\max_{s,s'\in [n]} \frac{\bar{\lambda}_{cross}(s,s')}{\bar{\lambda}_{self}(s)}\leq \eta^{d-1}$. Thus, the non-diagonal entries decay an exponential rate in comparison to the diagonal entries.\hfill\\
\textbf{Why increasing the depth beyond hurts training?} Note that for $\sigma=O\left(\sqrt{\frac{1}{w}}\right)$, for a fixed depth $d$, as width $w$ increases, $K_0\ra\E{K_0}$. However, the variance expression in \Cref{th:main}-$(ii)$ involves $d^2$ and $d^3$ terms, and hence for a fixed width as depth increases, the entries of $K_0$ deviates from $\E{K_0}$, and as a result the spectrum of $K_0$ degrades, thereby hurting training performance.
\begin{center}
\textbf{Optimisation with fixed random gating} 
\end{center}
Consider the dataset $(x_s,y_s)_{s=1}^n\in \R\times \R$, where $x_s=1,\forall s\in [n]$, and $y_s\sim unif([-1,1])$, $n=200$. The input Gram matrix $x^\top x$ is a $n\times n$ matrix with all entries equal to $1$ and its rank is equal to 1. Since all the inputs are identical, this is the worst possible case for optimisation. In this experiment, since we are interested only in the optimisation question, we take full control of the gating. For each input example, we sample gating values from $Ber(\mu)$ taking values in $\{0,1\}$, and collect it in $\G_0$. In this case, it is easy to check that $\mathbf{E}_{\mu}\left[\lambda_0(s,s)\right]=(\mu w)^{(d-1)},\forall s\in[n]$ and $\mathbf{E}_{\mu}\left[\lambda_0(s,s')\right]=(\mu^2 w)^{(d-1)},\forall s,s'\in[n]$.\hfill\\
\textbf{Theoretical Prediction:} It follows that,  for $\sigma=\sqrt{\frac{1}{\mu w}}$, all the diagonal entries of $\E{K_0}/d$ are $1$ and non-diagonal entries are $\mu^{d-1}$. Now, let $\rho_i\geq 0,i \in [n]$ be the eigenvalues of $\frac{\E{K_0}}{d}$, and let $\rho_{\max}$ and $\rho_{\min}$ be the largest and smallest eigenvalues.  One can easily show that $\rho_{\max}=1+(n-1)\mu^{d-1}$ and corresponds to the eigenvector with all entries as $1$, and $\rho_{\min}=(1-\mu^{d-1})$ repeats $(n-1)$ times, which corresponds to eigenvectors given by $[0, 0, \ldots, \underbrace{1, -1}_{\text{$i$ and $i+1$}}, 0,0,\ldots, 0]^\top \in \R^n$ for $i=1,\ldots,n-1$.\hfill\\
\begin{figure*}
\resizebox{\textwidth}{!}{
\begin{tabular}{ccc}
\includegraphics[scale=0.4]{figs/dgn-fra-ecdf-ideal.png}
&
\includegraphics[scale=0.4]{figs/dgn-fra-ecdfbyd-w500.png}
&
\includegraphics[scale=0.4]{figs/dgn-fra-conv-w500.png}
\end{tabular}
}
\caption{Shows the plots for DGN-FRG with $\mu=\frac{1}{2}$ and $\sigma=\sqrt{\frac{2}{w}}$. The first plot in the left shows the ideal cumulative eigenvalue (e.c.d.f) for various depths $d=2,4,6,8,12,16,20$. Note that the ideal plot converges to identity matrix as $d$ increases. The second plot from the left shows the cumulative eigenvalues (e.c.d.f) for $w=500$. }
\label{fig:dgn-frg-gram-ecdf}
\end{figure*}
\textbf{Numerical Evidence:} We look at the cumulative eigenvalue (e.c.d.f) obtained by first sorting the eigenvalues in ascending order then looking at their cumulative sum. The ideal behaviour (middle plot of \Cref{fig:dgn-frg-gram-ecdf}) as predicted from theory is that for indices $k\in[n-1]$, the e.c.d.f should increase at a linear rate, i.e., the cumulative sum of the first $k$ indices is equal to $k(1-\mu^{d-1})$, and the difference between the last two indices is $1+(n-1)\mu^{d-1}$. In \Cref{fig:dgn-frg-gram-ecdf}, we plot the e.c.d.f for various depths $d=2,4,6,8,12,16,20$ and $w=500$. \hfill\\
In order to compare how the rate of convergence varies with the depth, we set the step-size $\alpha=\frac{0.1}{\rho_{\max}}$, $w=100$. We use the vanilla SGD-optimiser. Note that the convergence rate is determined by a linear recursion $e_{t+1}=e_t-\alpha_t K_te_t$, and choosing $\alpha=\frac{0.1}{\rho_{\max}}$ can be seen to be equivalent to having a constant step-size of $\alpha=0.1$ but dividing the Gram matrix by its maximum eigenvalue instead. Thus, after this rescaling, the maximum eigenvalue is $1$ uniformly across all the instances, and the convergence should be limited by the smaller eigenvalues. We also look at the convergence rate of the ratio $\frac{\norm{e_t}^2_2}{\norm{e_0}^2_2}$, and we observe that the convergence rate gets better with depth as predicted by theory.

\section{Generalisation}\label{sec:generalisation}
In this section, we report experiments on CIFAR-10 and MNIST  datasets . Our aim here is to study the role of activations in generalisation performance of DNNs. In particular, we compare the generalisation performance of DNNs with adaptable gates (i.e., $A_t(\cdot,\cdot)$ changes with time) and DNNs with frozen gates (i.e., $A_t(\cdot,\cdot)=A_0(\cdot,\cdot),\forall t\geq 0$).
Network Architecture and Hyper-parameters:\hfill\\
\begin{table}
\begin{tabular}{|c|c|c|c|c|c|c|}\hline
&&&&\multicolumn{3}{c|}{GaLU (learnt)}\\\cline{5-7}
$(w,d)$	&Dataset		&ReLU		&GaLU(not-learned) 		&Good 		&Random Labels 	&Random Pixel\\\hline
$(128,6)$	& MNIST 		& $98.15$ 		&$96$ 		&$98.3$		&$92.6$			&$94.3$\\\hline
$(256,6)$	& MNIST 		& $98$ 		&$96$ 		&$98$		&$-$			&$-$\\\hline
\end{tabular}
\end{table}
\textbf{Frozen Non-Learned gates generalise poorly:} We trained both ReLU and frozen-GaLU (in $\N_G(\Theta_t,\G(\N_R(\Tg_t)))$ we have $\Tg_t=\Tg_0,\forall t\geq 0$) networks on standard MNIST dataset to close to $100\%$ accuracy. We observed that the frozen-GaLU network trains a bit faster than the ReLU network. However, the test performances were around $96\%$ and  $98\%$ for frozen-GaLU and ReLU networks respectively\footnote{In the case of GaLU, in order to eliminate the effect of initialisation, we trained with identical as well as independent initialisation for the gating as well as the main network in the GaLU. }. We trained ReLU and GaLU networks on standard CIFAR-10 dataset close to $100\%$ training accuracy. In this case, the test performances were around $64\%$ and $72\%$ for GaLU and ReLU networks respectively.\hfill\\
\textbf{Frozen Learned gates generalise better:} Note that, in these experiments, the frozen-GaLU network gates are copied from a gating network which is pre-trained, however, the weights $\Theta_0$ are initialised and trained again.  In the case of MNIST, 
In the case of CIFAR-10, frozen-GaLU with learned gates achieved a test performance of $70\%$, which is better than frozen-GaLU with non-learned gates with a performance of  $64\%$, and is however less than ReLU with performance of $72\%$. \hfill\\
\begin{figure}
\resizebox{\textwidth}{!}{
\begin{tabular}{ccc}
\includegraphics[scale=0.18]{figs/path-gram.png}
&
\includegraphics[scale=0.1]{figs/allnet-train.png}
&
\includegraphics[scale=0.1]{figs/allnet-gen.png}
\end{tabular}
}
\caption{First two plots from the left show optimisation and generalisation in ReLU and GaLU networks for standard MNIST. The right most plot shows $\nu_t=y^\top (\widehat{H}_t)^{-1}y$, where $H_t=\Phi_t^\top \Phi_t$.}
\label{fig:gen}
\end{figure}
\textbf{NPK dynamics in ReLU:} We consider ``Binary''-MNIST data set with two classes namely digits $4$ and $7$, with the labels taking values in $\{-1,+1\}$ and squared loss. We trained a standard DNN with ReLU activation ($w=100$, $d=5$). Recall that $H_t=\Phi^\top_t\Phi_t$  (the Gram matrix of the features) and let $\widehat{H}_t=\frac{1}{trace(H_t)}H_t$ be its normalised counterpart. For a subset size, $n'=200$ ($100$ examples per class) we plot $\nu_t=y^\top (\widehat{H}_t)^{-1} y$, (where $y\in\{-1,1\}^{200}$ is the labeling function), and observe that $\nu_t$ reduces as training proceeds (see first plot in \Cref{fig:gen}). Note that $\nu_t=\sum_{i=1}^{n'}(u_{i,t}^\top y)^2 (\hat{\rho}_{i,t})^{-1}$, where $u_{i,t}\in \R^{n'}$ are the orthonormal eigenvectors of $\widehat{H}_t$ and $\hat{\rho}_{i,t},i\in[n']$ are the corresponding eigenvalues. Since $\sum_{i=1}^{n'}\hat{\rho}_{i,t}=1$, the only way $\nu_t$ reduces is when more and more energy gets concentrated on $\hat{\rho}_{i,t}$s for which $(u_{i,t}^\top y)^2$s are also high. However, in $H_t=(x^\top x)\odot \lambda_t$, only $\lambda_t$ changes with time. Thus, $\lambda_t(s,s')$ which is a measure of overlap of sub-networks active for input examples $s,s'\in[n]$, changes in a manner to reduce $\nu_t$. We can thus infer that the \emph{right} active sub-networks are learned over the course of training. We now summarise the insights obtained from these experiments in the following remarks:\hfill\\
\textbf{Remark} $(1)$ Neural path features encode information, and they are learned over the course of training. This is clear from the difference in the performances of frozen-GaLU networks with non-learned gates and learned gates.\hfill\\
\textbf{Remark} $(2)$ \emph{Adaptability} of the gates and hence activations is key for generalisation. \hfill \\
\textbf{Remark} $(3)$ Prior works \cite{arora,arora2019exact,cao2019generalization} suggest that, for randomised initialisation, DNNs can be thought of as learning with the linear features given by the random NTFs, and provide generalisation bounds with the corresponding NTK. \cite{arora2019exact} proposes a pure-kernel learning method with the NTK that  performs much better than the prior state-of-the-art kernel only based methods. However, couple of issues remain unresolved. Firstly, if the DNNs are only linear learners with random NTFs, then it suggests that no feature learning happens in DNNs. This issue is resolved by \textbf{Remark} $(1)$ above. Secondly, the DNNs perform slightly better than their corresponding NTK counterparts \cite{arora2019exact,lee2017deep}. This issue is resolved by \textbf{Remark} $(2)$.



%\section{Understanding the role of convolutions and pooling operations}\label{sec:conv}
In this section, we will use the frameworks of DGNs and ``path-view'' to obtain insights about (i) convolutional layers and (ii) pooling: global average pooling\footnote{The arguments can be extended to $\max$-pooling with technical modifications.}. In this section, we continue to be in the DGN setup, i.e., we will have separate parameterisations $\Tg$ and $\Tv$, and assume that Assumptions~\ref{assmp:mainone}, \ref{assmp:maintwo} hold. However, we impose additional restrictions to account for the presence of convolutional and pooling layers, which, we describe below.

\textbf{Circular Convolutional layers:}

$1.$ We assume that, the initial $0<L<d$ layers are convolutional layers. In particular, each layer uses a $1$-dimensional kernel of width $0<\hat{w}<d_{in}$, and the output of each layer is a $d_{in}$-dimensional vector.

$2.$ We consider circular convolutional operations instead of zero padding, i.e., during the convolution operation, say index $i$ exceeds $d_{in}$ then it will be considered as $i-d_{in}$, and in the case when a negative index is required, i.e., if index $i<0$ is needed, then $d_{in}+i$ will be used instead. We illustrate this circular convolution with the help of \Cref{fig:circconv}, wherein, $\hat{w}=2$, $d_{in}=3$. Here, $\theta^{(l)}(i),l=1,\ldots,L-1, i=1,2$ are the weights, and the final layer weight $\theta^{(L)}=\left[\frac{1}{d_{in}},\ldots, \frac{1}{d_{in}}\right]^\top\in \R^{d_{in}}$ in the case of global average pooling. Note that, in \Cref{fig:circconv}, we have used only one network, and we have also used a simpler and different notation for the weights: this is because, in DGN (with circular convolutions), both the gating network parameterised by $\Tg$ and the weights network parameterised by $\Tv$ will have identical architecture, and in order to explain just the circular convolution alone more clearly, we have used a simpler notation for the weights and have left the gating information unspecified in \Cref{fig:circconv}.

\begin{figure}
\centering
\includegraphics[scale=0.25]{figs/circconv.png}
\caption{Shows a circular convolutional network with $d_{in}=3$ and kernel size $\hat{w}=2$. Note that there are only $8$ unique path strengths in this example (in the case of global average pooling).}
\label{fig:circconv}
\end{figure}



\textbf{Path Sharing:} With the understanding of circular convolution in the background, we now investigate the similarity of two inputs $x_s\in \R^{d_{in}}$ and $x_{s'}\in \R^{d_{in}}$ after they pass through $L$ convolutional layers. To be specific, let $x_s(L)\in \R^{d_in}$ and $x_{s'}(L)\in \R^{d_{in}}$ be the outputs obtained after the $L$ convolutional layers. Note that $x_s(L)=(x_s(L,i),i\in[d_{in}])\in \R^w$ is a $d_{in}$-dimensional vector with $i=1,\ldots,d_{in}$ components, wherein, the $i^{th}$ component is obtained by circular convolution using a kernel of size $0<\hat{w}<d_{in}$. Further, we restrict our attention to the first $L$ layers which perform the convolution operations. We are interested in investigating the following: 
\begin{align}
\E{\ip{x_s(L),x_{s'}(L) }}=\sum_{i=1}^{d_{in}} \E{x_s(L,i)x_{s'}(L,i)}
\end{align}
Given the randomised and symmetric nature of the weight initialisation, without loss of generality, it is sufficient to study $\E{x_s(L,1)x_{s'}(L,1)}$, i.e., it is enough to consider the case of $L$ convolutions with kernel of size $\hat{w}$ followed by a global average pooling. We now make the following observations:

$1.$ There are $p=1,\ldots,\hat{P}=d_{in}\hat{w}^{d-1}$ paths.

$2.$ There are $k=1,\ldots,\hat{B}=\hat{w}^{d-1}$ unique path strengths. This is due to the fact that the same path strength repeats $d_{in}$ times. For instance, in \Cref{fig:circconv}, the path strength $\theta^{1}(1)\theta^{1}(2)\theta^{1}(3)\frac{1}{d_{in}}$ repeats $3$ times.

$3.$ Paths can be grouped into bundles $b_k,k\in[\hat{w}^{d-1}]$, wherein, bundle $b_k$ comprises of $d_{in}$ paths, all of which have the same path strength. Without loss of generality, $b_k$ comprises of paths $(k-1)d_{in}+1,\ldots, kd_{in}$.

$4.$ The path strength $w_t=(w_t(b_1),\ldots, w_t(b_{\hat{B}}))\in \R^{\hat{P}}$, where $w_t(b_k)=(w_t(p),p=(k-1)d_{in}+1,\ldots,kd_{in})\in \R^{d_{in}}$. 

$5.$ The output $x_s(L,1)=\phi^\top_{x_s,\G_t} w_t$.

\begin{lemma}\label{lm:invariance}
At $t=0$, under \Cref{assmp:main}, convolutional layers with global average pooling at the end causes translational invariance.
\begin{align*}
&\E{x_s(L,1)x_{s'}(L,1)}\\&=\frac{\sigma^{2(d-1)}}{d^2_{in}}\sum_{k=1}^{\hat{B}} \sum_{p_1,p_2\in b_k}  \Big( x(p_1(0),s) A(x_s,p_1)\\
&\quad\quad \quad\quad \quad\quad x(p_2(0),s') A(x_{s'},p_2) \Big)
\end{align*}
\end{lemma}

\textbf{Remark:} Now, for $i\in\{0,\ldots, d_{in}-1\}$, let $z^{(i)}\in \R^{d_{in}}$ be the clockwise rotation of $z\in \R^{d_{in}}$ by $i$ co-ordinates, and let $x^{(i)}\in \R^{d_{in}\times n}$ be the data matrix obtained by clockwise rotation of the columns of the data matrix $x\in \R^{d_{in}\times n}$ by $i$ co-ordinates. Then, we have

\begin{align*}
&\E{x_s(L,1)x_{s'}(L,1)}\\
&=\frac{\sigma^{2(d-1)}}{d^2_{in}}\sum_{k=1}^{\hat{B}} \sum_{i=1}^{d_{in}} \sum_{p\in b_k}   \Big(x(p(0),s) A(x,p) \\ 
&\quad\quad \quad\quad \quad\quad x^{(i)}(p(0),s') A(x^{(i)}_{s'},p) \Big)
\end{align*}
The term $\sum_{p\in b_k}  x(p(0),s) A(x,p) x^{(i)}(p(0),s') A(x^{(i)}_{s'},p)$ is translation invariant.


\begin{comment}
\textbf{Claim $2$:} At $t=0$, under Assumptions~\ref{assmp:mainone},\ref{assmp:maintwo}, convolutional layers with $\max$-pooling at the end causes translational invariance.

\begin{proof}
The proof follows in a manner similar to \textbf{Claim $1$} made for the case of global average pooling. However, the technical challenge is the following: in the case of $\max$-pooling, only one of the $d_{in}$ paths connecting the $(L-1)^{th}$ layer to the output node is \emph{on}. This path connects the ``$\max$" node in the $(L-1)^{th}$ layer to the output node. This can be accounted in the calculations by setting the path strength to be $0$ for those paths that do not pass through the ``$\max$" node in the $(L-1)^{th}$ layer.  We make the following observations about $M$:

$1.$ For each bundle $b_k, k\in[\hat{B}]$, $\exists$ unique indices $i(k), j(k)\in [d_{in}]$ such that $M((k-1)d_{in}+i(k), (k-1)d_{in}+j(k))=\sigma^{2(d-1)}$, and rest of the entries of $M$ are $0$.

$2.$ $M(p_1,p_2)=\frac{\sigma^{2(d-1)}}{d^2_{in}}$, if $p_1$ and $p_2$ belong to the same bundle continues to hold trivially due to observation $1$ (of the current claim).

And by going through reductions similar to \textbf{Claim $1$}, we have

\begin{align*}
&\E{x_s(L,1)x_{s'}(L,1)}\\&=\phi^\top_{x_s,\G_0} M \phi^\top_{x_{s'},\G_0}\\
&=\sum_{p_1,p_2=1}^{\hat{P}} \Big(x(p_1(0),s) A(x_s,p_1) \\
&\quad\quad \quad\quad \quad\quad x(p_2(0),s') A(x_{s'},p_2) M(p_1,p_2)\Big)\\
&=\frac{\sigma^{2(d-1)}}\sum_{k=1}^{\hat{B}} \sum_{p_1,p_2\in b_k}  \Big( x(p_1(0),s) A(x_s,p_1)\\
&\quad\quad \quad\quad \quad\quad x(p_2(0),s') A(x_{s'},p_2) \Big)
\end{align*}
Now, for $i\in\{0,\ldots, d_{in}-1\}$, let $z^{(i)}\in \R^{d_{in}}$ be the clockwise rotation of $z\in \R^{d_{in}}$ by $i$ co-ordinates, and let $x^{(i)}\in \R^{d_{in}\times n}$ be the data matrix obtained by clockwise rotation of the columns of the data matrix $x\in \R^{d_{in}\times n}$ by $i$ co-ordinates. Then, we have

\begin{align*}
&\E{x_s(L,1)x_{s'}(L,1)}\\
&=\frac{\sigma^{2(d-1)}}{d^2_{in}}\sum_{k=1}^{\hat{B}} \sum_{i=1}^{d_{in}} \sum_{p\in b_k}   \Big(x(p(0),s) A(x,p) \\ 
&\quad\quad \quad\quad \quad\quad x^{(i)}(p(0),s') A(x^{(i)}_{s'},p) \Big)
\end{align*}
The term $\sum_{p\in b_k}  x(p(0),s) A(x,p) x^{(i)}(p(0),s') A(x^{(i)}_{s'},p)$ is translation invariant.

\end{proof}
\end{comment}
%\section{Related Work}
\cite{dudnn} show that in fully connected DNNs with $w=\Omega(poly(n)2^{O(d)})$, and in residual neural networks (ResNets) with $w=\Omega(poly(n,d))$ gradient descent converges to zero training loss. \cite{dudnn} claim to demystify the second part of what we called the depth phenomena (``why deeper networks are harder to train"), since, the dependence on the number of layers improves exponentially for ResNets. Our optimisation results are weaker than \cite{dudnn} in the sense that we consider only DGNs with decoupling assumptions. However, we show both parts of the depth phenomena, in particular why increasing depth till a point helps training. 

\begin{comment}Further, the \emph{algebraic} nicety due to \emph{Hadamard} product decomposition of the Gram matrix is a useful take away. In addition, the connection to how the sub-network overlap is a conceptual gain. 
In comparison to \cite{dudln} the gain in our work is that, thanks to the path-view, we obtain a single expression for $\E{K_0}$ which can be applied to deep linear networks, GaLU networks and any networks whose gating values are known and fixed. Both \cite{dnn,dln} are analytically more involved, in that they provide guaranteed rates of converges with high probability, and in comparison, our work has stopped with the variance calculation.
\end{comment}
In comparison to \cite{sss} who were the first to initiate the study on GaLU networks, we believe, our work has made significant progress. We introduced adaptable gates, and showed via experiments, that, gate adaptation is key in learning, thereby showing a clear separation between GaLU and ReLU networks. To support the claim, we have used idea from \cite{arora}, in that, we measure $\nu_t=y^\top {K_t}^{-1}y$ to show that the eigen spaces indeed align with respect to the labelling function.

In comparison to \cite{lottery}, we also show in our experiments that the winning lottery is in the gating pattern, which, in the case of ReLU networks is inseparable from the weights. However, our experiments show that the weights can be reinitialised if we have the learned gating pattern.


%\section{Conclusion }
\begin{comment}
In this paper, we looked at the gradient descent (GD) procedure to minimise the squared loss in deep neural networks. Prior literature \cite{dudnn} makes trajectory analysis (wherein the dynamics of the error terms are studied) to show that GD achieves zero training error. In this paper, we introduced to important conceptual novelties namely deep gated networks (DGNs) and path-view, to obtain additional insights about GD in the context of trajectory analysis. In particular, our theory threw light on i) the depth phenomena and ii) gate adaptation, i.e., the role played by the dynamics of the gates in generalisation performance.

The path-view lead following gains: (i) an explicit expression of information propagation in DGNs where in the input signal and the wires, i.e., the deep network itself are separated. This is unlike the conventional layer by layer approach, wherein, the input is lost in the hidden layers, (ii) explicitly identifying the role of sub-networks in training and generalisation of deep networks, so much so that, we can go so far as to say that the actual \emph{hidden features are in the paths and the sub-networks and not just the final layer output}, (iii) explicit identification of twin gradient flow, wherein, one component of the gradient flow to train the paths keeping the sub-network constant and the other component of the gradient takes care of learning the gating values.

We looked at various DGNs with adaptable gates and we observed  in experiments that the adaptable/learned gates generalise better than non-adapting/non-learned gates.  Based on our theory and experiments, we conclude that \emph{understanding generalisation would involve a study of gate adaptation}.
\end{comment}
\begin{comment}
In this paper, we introduced to important conceptual novelties namely deep gated networks (DGNs) and path-view, to obtain additional insights about gradient descent in deep learning. The path-view lead to the following gains: (i) an explicit expression of information propagation in DGNs (ii) explicitly identifying the role of sub-networks in training and generalisation of deep networks, (iii) explicit identification of twin gradient flow, wherein, one component of the gradient flow to train the path strengths keeping the sub-network constant and the other component of the gradient takes care of learning the gating values. Using the path-view and the DGNs, we showed  i) the depth helps is equivalent to whitening of data and increasing depth beyond degrades the spectrum of the Gram matrix at initialisation, and ii) gate adaptation, i.e., the role played by the dynamics of the gates is important for generalisation performance.

We looked at various DGNs with adaptable gates and we observed  in experiments that the adaptable/learned gates generalise better than non-adapting/non-learned gates.  Based on our theory and experiments, we conclude that \emph{understanding generalisation would involve a study of gate adaptation}.
\end{comment}

In this paper, we introduced two important conceptual novelties namely deep gated networks (DGNs) and ``path-view", to obtain additional insights about gradient descent in deep learning. Using these two novel concepts, we achieved the following:

 (i) resolution to the depth phenomena for DGNs under decoupling assumption. In particular, our results showed that increasing depth is equivalent to whitening of data and increasing depth beyond a point degrades the spectrum of the Gram matrix at initialisation.
 
 (ii) each input example has a corresponding active sub-network, which are learned when the gates adapt.
 
 (iii) a preliminary theory to analyse gate adaptation. Our analysis points out to the presence of two complementary networks for each input example, one being the active sub-network which holds the memory for that input example and the other being the sensitivity sub-network of gates that are adapting.
 
(iv) we looked at various DGNs with adaptable gates and we observed  in experiments that the adaptable/learned gates generalise better than non-adapting/non-learned gates.  

Based on our theory and experiments, we conclude that :

(a) \emph{Hidden features are in the active sub-networks,} which are in turn decided by the gates.

(b) \emph{Understanding generalisation would involve a study of gate adaptation.}



\begin{comment}
Let $\gamma>0$ be a threshold value, and let $G_{x_s,\Tg_t}(l,i)$ denote the gating value node $i$ in layer $l$. We say that the gate to be \emph{transitioning} for input $s\in[n]$, and weight $\tg(m),m\in[d_{net}]$ if
 \begin{align}
 \left|\frac{\partial G_{x_s,\Tg_t}(l,i)}{\partial \tg(m)}\right|>\gamma,
 \end{align}
 and define a gate to be \emph{flipped} otherwise. Note that,
\begin{align}\label{eq:sensitivepath}
\begin{split}
&\partial_{m}A_{\Tg_t}(x_s,p)=\partial_{m}\Pi_{l=1}^{d-1} G_{x_s,\Tg_t}(l,p(l))\\
&=\sum_{l=1}^{d-1} \partial_{m} G_{x_s,\Tg_t}(l,p(l)) \left(\Pi_{l'\neq l} G_{x_s,\Tg_t}(l',p(l'))\right)
\end{split}
\end{align}

\textbf{Remark:}

i) As $\beta\uparrow\infty$, the soft-ReLU gate resembles the ReLU gate. Thus for a given input example $s$, the gates whose pre-activation inputs have a large absolute value will be close to either $0$ or $1$, and one can always find a high enough $\beta$ such that their sensitivity to $\tg(m)$ is less than $\gamma$.

ii) For an input examples $s,s'\in[n]$, if a path $p$ is active (even for one of the inputs), i.e., $A(x_s,p)\approx 1$, then none of the gates in the path will be sensitive, and hence the magnitude contribution of such as path to the summation in $\delta$ is close to $0$.

iii) For an input examples $s,s'\in[n]$, consider a non-active path, such that all gates close to $1$ except for one of the gates (i.e., the right hand side of \eqref{eq:sensitivepath} is non-zero), which is transitioning. Such paths will make a significant contribution to $\delta$ term. We call the set of such paths the sensitive sub-network.

Based on the above discussion one can say  that a DGN with adaptable gates (which includes standard DNN with ReLU gates), at initialisation, has two kinds of sub-networks for every input example i) the active sub-network comprised of path for which $A(x_s,p)=1$\footnote{or $A(x_s,p)$  is above a given threshold value in the case of soft gates} and ii) the sensitive sub-network which is formed by the set of paths that are sensitive for a given input.
\end{comment}
\bibliographystyle{plainnat}
\bibliography{refs}
%
\onecolumn
\begin{center}
Appendix/Supplementary Material
\end{center}
\section{Paths}\label{sec:path}
%\subsection{Zeroth Order Terms}
\textbf{Vectorised Notation:} Given a dataset $(x_s,y_s)_{s=1}^n\in \R^{d_{in}}\times \R$, let data be represented as matrices $x\in\R^{d_{in}\times n}$ and $y\in \R^n$ with the convention that $x_s=x(\cdot,s)\in\R^{d_{in}}$ and $y_s=y(s)\in \R$. For the purpose of this section we follow the vectorised notation in \Cref{tb:dgnvector}.

\FloatBarrier
\begin{table}[h]
\centering
\begin{tabular}{|c|c|}\hline
Input layer & $x(s,i,0) =x(i,s)$ \\\hline
Pre-activation& $q_t(s,i,l)= {\Theta_t(l,\cdot,i)}^\top x_t(s,\cdot,l-1) $ \\\hline
Layer output & $x_t(s,i,l)= q_t(s,i,l) G_t(s,i,l)$ \\\hline
Final output & $\hat{y}_t(x)={\Theta_t(d,\cdot,1)}^\top x_t(s,\cdot,d-1)$\\\hline
\end{tabular}
\caption{A deep gated network in the vectorised form. $l=1,\ldots,d-1$ denote the intermediate layers.}
\label{tb:dgnvector}
\end{table}




The idea behind the ``path view'' is to regard the given neural network as multitude of connections from input to output.  We now describe the zeroth and first order terms in the language of paths.
\begin{definition}[Neural Path]
Let $\P=[d_{in}]\times [w]^{d-1}$ be a cross product of index sets. Define a path $p$ by $p\stackrel{def}=(p(0),p(1),\ldots,p(d-1))\in \P$, where $p(0)\in [d_{in}]$, and $p(l)\in[w],\forall l\in[d-1]$. 
\end{definition}

A path $p$ starts at an input node $p(0)$ goes through nodes $p(l)$ in layer $l\in[d-1]$ and finishes at the output node .%(see \Cref{fig:path}%\todoch{Need to add a diagram like the recent lottery ticket weight Vivek Ramanujam paper.}). 
%In what follows, we use $p\rsa (\cdot)$ to denote the fact that path $p$ passes through node $(\cdot)$, where $(\cdot)$ is an input node, or a weight in some layer, or a hidden node.

\begin{definition}\label{def:strength}[Strength]
Each path is also associated with a strength given by: $w_t(p)=\Pi_{l=1}^d \Theta_t(l,p(l-1),p(l))$
%\begin{align}
%$w_t(p)=\Pi_{l=1}^d \Theta_t(l,p(l-1),p(l))$
%\end{align}
\end{definition}

\begin{definition}\label{def:activity}[Activation Level]
The activity of a path $p$ for input $s$ is given by: $A(s,p)=\Pi_{l=1}^d G(s,p(l),l)$
%\begin{align}
%$A(s,p)=\Pi_{l=1}^d G(s,p(l),l)$
%\end{align}
\end{definition}
In the case when $G\in \{0,1\}$ it also implies that $A\in \{0,1\}$.  %Thus $A(s,p)$ tells us whether the path $p$ is \emph{active} when $s^{th}$ input example is presented to the network.

\begin{definition}\label{def:feature}[Neural Feature]
Given a gating pattern $\G_t$, define 
\begin{align}
\phi_{x_s,\G_t}(p)\stackrel{def}=x(p(0),s) A_{\G_t}(x_s,p),
\end{align}
and let $\phi_{x_s,\G_t}=(\phi_{x_s\G_t}(p),p\in [P])\in \R^P$ be the hidden feature corresponding to input $x_s$. Let $\Phi_{x,\G_t}=\left[\phi_{x_1,\G_t}| \ldots |\phi_{x_n,\G_t}\right]\in \R^{P\times n}$ be the feature matrix obtained by stacking the features $\phi_{x_s,\G_t}$ of inputs $x_s\in \R^{d_{in}}$ column-wise.
\end{definition}

\comment{
\textbf{Predicted} output of the network is given in terms of the paths by $\hat{y}_{t}(s)=\sum_{p\in P} x(p(0),s) A_{\G_t}(x_s,p) w_t(p)$, i.e., 
\begin{align}\label{eq:zeroth}
\hat{y}_{t}(s)=\Phi_{x,\G_t}^\top w_{t}
\end{align}

%\begin{figure}
%\centering
%\includegraphics[scale=0.2]{mickey.png}
%\caption{Some picture to break the monotony}
%\end{figure}

%\textbf{Sub-networks:} %An important fact that can be seen as an immediately from the path based representation is that DGNs 
\textbf{Sub-networks:}  In DGNs similarity of two different inputs $x_s,x_{s'}\in \R^{d_{in}}, s,s' \in [n]$ depends on the overlap of path that are \emph{active} for both inputs. This can be seen by noting that $\phi_{x_s,\G_t}^\top \phi_{x_s,\G_t}=\sum_{i=1}^{d_{in}} x(i,s)x(i,s') \underset{p\rsa i}{\sum} A_{\G_t}(x_s,p) A_{\G_t}(x_{s'},p)$. Consider for instance a DGN whose gating values are in $\{0,1\}$, and say $n=2$, i.e., the dataset contains only two inputs $x_1,x_2\in \R^{d_{in}}$. From \eqref{eq:pathsim} it is clear that if inputs $x_1,x_2$ do not share any common \emph{active} paths throughout training, then they are \emph{orthogonal} to each other, because $A_{\G_t}(s,p) A_{\G_t}(s',p)=0, \forall p\in [P]$. This makes intuitive sense because in this case, it is as though there are two parallel networks (while the weights can be shared, the paths are not). Thus it is clear from the path based representation in \eqref{eq:zeroth} and \eqref{eq:pathsim}, the gating pattern $\G_t$ plays are crucial role in DGNs via the activation levels $A_{\G_t}(\cdot,\cdot)$.
%\subsection{First Order Terms}
The feature $\phi_{x_s,\G_t}$ in \Cref{def:feature} as well as the strength $w_t$ in \Cref{def:strength} are $P$-dimensional quantities. However, loosely speaking, the DGN has only as much \emph{degrees of freedom} as the number of trainable parameters (which we denote by $d_{net}$). 
\begin{definition}[Neural Tangent Features] The  $d_{net}\times n$ NTF matrix is defined as $\Psi_t(m,s)\stackrel{def}=\frac{\partial \hat{y}_{t}(x_s)}{\partial \theta(m)},m\in [d_{net}], s\in [n]$. 
\comment{
\begin{align}\label{eq:split}
\begin{split}
&\Psi_t(m,s) = \frac{\partial \hat{y}_t(x_s)}{\partial \theta(m)}\\
&=\frac{\partial }{\partial \theta(m)}\left(\sum_{p\in P} x(p(0),s) w_{t}(p) A_{\G_t}(s,p)\right),\\
&=\underbrace{\left(\sum_{p\in P} x(p(0),s) \frac{\partial w_{t}(p)}{\partial \theta(m)} A_{\G_t}(s,p)\right)}_{\text{sensitivity of strength}}\\
&\quad\quad\quad\quad\quad\quad\quad\quad+\\
&=\underbrace{\left(\sum_{p\in P} x(p(0),s) w_{t}(p) \frac{\partial A_{\G_t}(s,p)}{\partial \theta(m)}\right)}_{\text{sensitivity of activations}}
\end{split}
\end{align}

\begin{align}\label{eq:split}
\begin{split}
&\Psi_t(m,s) = \frac{\partial \hat{y}_t(x_s)}{\partial \theta(m)}\\
&=\frac{\partial }{\partial \theta(m)}\left(\sum_{p\in P} x(p(0),s) w_{t}(p) A_{\G_t}(s,p)\right),\\
&=\underbrace{\left(\sum_{p\in P} x(p(0),s) \frac{\partial w_{t}(p)}{\partial \theta(m)} A_{\G_t}(s,p)\right)}_{{\Psi^w_{t}(m,s)}}\\
&\quad\quad\quad\quad\quad\quad\quad\quad+\\
&=\underbrace{\left(\sum_{p\in P} x(p(0),s) w_{t}(p) \frac{\partial A_{\G_t}(s,p)}{\partial \theta(m)}\right)}_{{\Psi^a_t(m,s)}}
\end{split}
\end{align}
}
\comment{
\begin{align}\label{eq:split}
\begin{split}
\Psi_t(m,s) &= \underbrace{\left(\sum_{p\in P} x(p(0),s) \frac{\partial w_{t}(p)}{\partial \theta(m)} A_{\G_t}(s,p)\right)}_{{\Psi^w_{t}(m,s)}}\\
&\quad\quad\quad\quad\quad\quad\quad\quad+\\
&=\underbrace{\left(\sum_{p\in P} x(p(0),s) w_{t}(p) \frac{\partial A_{\G_t}(s,p)}{\partial \theta(m)}\right)}_{{\Psi^a_t(m,s)}}
\end{split}
}
\begin{align}\label{eq:split}
\begin{split}
\Psi_t(m,s) &= \Psi^w_{t}(m,s)+\Psi^a_{t}(m,s), \,\text{where}\\
\Psi^w_{t}(m,s)&={\left(\sum_{p\in P} x(p(0),s) \frac{\partial w_{t}(p)}{\partial \theta(m)} A_{\G_t}(s,p)\right)}\\
\Psi^a_{t}(m,s)&={\left(\sum_{p\in P} x(p(0),s) w_{t}(p) \frac{\partial A_{\G_t}(s,p)}{\partial \theta(m)}\right)}
\end{split}
\end{align}


\end{definition}

\textbf{Strength and Gate adaptation:} From \eqref{eq:split} it is clear that there are two \emph{atomic} components to the gradient of the output $\hat{y}_t(x_s)$ with respect to any trainable weight $\theta(m), m=1,\ldots, d_{net}$, namely \emph{neural tangent feature of strength} denoted by $\Psi^w_{t}\in \R^{d_{net}\times n}$ and \emph{neural tangent feature of activations} denoted by $\Psi^a_{t}\in \R^{d_{net}\times n}$. %Before we separately define these two \emph{atomic} components, we would like to mention that as a result of the $\frac{\partial A_{\G_t}(s,p)}{\partial \theta(m)}$ term, gates adapt all throughout training, which in turn affect the output (see \eqref{eq:zeroth}) via $A_{\G_t}$.

}


\comment{
\begin{definition}[Neural Tangent Features of Path Activations (NTFPA)]
\begin{align}
\varphi^a_{t,p}\stackrel{def}{=}\left(\frac{\partial w_{t}(p)}{\partial \theta(m)},m\in[d_{net}]\right)\in \R^{d_{net}},
\end{align}
\end{definition}

\begin{remark}
Let $\theta(m)$ belong to layer $l'\in [d-1]$, then 
\begin{align*}
\varphi^a_{t,p}(m)&=0, \forall p\bcancel{\rsa}\theta(m)
\end{align*}
\end{remark}

Using the sensitivity of strengths and activations at the level of resolution of paths, we now define the neural tangent feature (NTF) for the strengths and activations.

\begin{definition}\label{def:ntf}[Neural Tangent Features]
\begin{align}
\begin{split}
\Psi^w_t(m,s) &=\left(\sum_{p\in P} x(p(0),s) \frac{\partial w_{t}(p)}{\partial \theta(m)} A_{\G_t}(s,p)\right)\\
\Psi^a_t(m,s) &=\left(\sum_{p\in P} x(p(0),s) w_{t}(p) \frac{\partial A_{\G_t}(s,p)}{\partial \theta(m)}\right)\\
\Psi_t(m,s)&=\Psi^w_t(m,s)+ \Psi^a_t(m,s)
\end{split}
\end{align}
\end{definition}


\begin{definition}[Interaction Coefficient]
\begin{align*}
&\lambda^{s,s'}_t(i)\stackrel{def}=\underset{p_1,p_2\rsa\theta(m),i}{\sum_{p_1,p_2\in P:}}  \varphi_{t,p_1}(m)A(s,p_1)  \varphi_{t,p_2}(m) A(s',p_2)
\end{align*}
\end{definition}

\begin{lemma}
\begin{align*}
{K^w_t(s,s')}=\sum_{i=1}^{d_{in}} x(i,s)x(i,s') \lambda^{s,s'}_t(i)
\end{align*}
\end{lemma}

\begin{assumption}\label{assmp:main}\hfill
\begin{enumerate}
\item $\G_0$ is statistically independent of $\Theta_0$.
\item $\Theta_0\stackrel{iid}\sim Ber(\frac{1}{2})$ over the set $\{-\sigma,+\sigma\}$. 
\end{enumerate}
\end{assumption}

\textbf{Remarks on Assumption~\ref{assmp:main}}
In a standard DNN with ReLU activations, the activations and weights are not statistically independent because conditioned on the fact that a ReLU is \emph{on}, the incoming edges cannot be simultaneously all $-\sigma$. We side step this issue by the first condition in Assumption~\ref{assmp:main}, wherein, we assume that  gating is statistically independent of the weights $\Theta_0$. This clears the way to carry out the algebra of paths, which can be boiled down in simple words as the effect on weights in the direction of one path does not affect the contribution of any other path in expectation. This is captured in Lemma~\ref{lm:pathdot} below.
\begin{figure}
\centering
\includegraphics[scale=0.2]{mickey.png}
\caption{Shows that the incoming weights of a ReLU gate which are \emph{on} are not symmetrically distributed.}
\end{figure}

\begin{lemma}\label{lm:pathdot}
Let $\theta(m)$ be an arbitrary weight in layer $l'\in [d-1]$, under Assumption~\ref{assmp:init} we have for paths $p,p_1,p_2\rsa\theta(m), p_1\neq p_2$
\begin{align*}
\E{\varphi_{\Tb,p_1}(m)\varphi_{\Tb,p_2}(m)}= &0\\
\E{\varphi_{\Tb,p}(m)\varphi_{\Tb,p}(m)}= &\left(\frac{2\sigma^2}{w}\right)^{d-1}
\end{align*}
\end{lemma}
\begin{proof}
If $\theta$
Note that $\varphi_{\Tb,p}=\underset{l\neq l'}{\underset{l=1}{\overset{d}{\Pi}}} \Tb(l,p(l-1),p(l))$. Hence
\begin{align*}
&\E{\varphi_{\Tb,p_1}(m)\varphi_{\Tb,p_2}(m)}\\
&=\E{\underset{l\neq l'}{\underset{l=1}{\overset{d}{\Pi}}} \Bigg(\Tb(l,p_1(l-1),p_1(l))\Tb(l,p_2(l-1),p_2(l)) \Bigg)}\\
&=\underset{l\neq l'}{\underset{l=1}{\overset{d}{\Pi}}}\E{\Tb(l,p_1(l-1),p_1(l))\Tb(l,p_2(l-1),p_2(l))}
\end{align*}

Since $p_1\neq p_2$, in one of the layers $\tilde{l}\in[d-1],\tilde{l}\neq l'$ they do not pass through the same weight. Using this fact
\begin{align*}
&\E{\varphi_{\Tb,p_1}(m)\varphi_{\Tb,p_2}(m)}\\
&=\left(\underset{l\neq l',\tilde{l}}{\underset{l=1}{\overset{d}{\Pi}}}\E{\Tb(l,p_1(l-1),p_1(l))\Tb(l,p_2(l-1),p_2(l))}\right)\\
&\Bigg(\E{\Tb(\tilde{l},p_1(\tilde{l}-1),p_1(\tilde{l}))}\E{\Tb(\tilde{l},p_2(\tilde{l}-1),p_2(\tilde{l}))}\Bigg)\\
&=0
\end{align*}
\end{proof}


\begin{definition}[Path Similarity]
\begin{align*}
\mu^{s,s'}(i)=\sum_{m=1}^{d_{net}} \underset{p\rsa\theta(m)}{\sum_{p\in P: p(0)=i}}A(s,p) A(s',p)
\end{align*}

\end{definition}

\begin{lemma}
\begin{align*}
\mathbf{E}_{\Theta_0}\left[\lambda^{s,s'}_0(i)\right]=\sigma^{2(d-1)}\mu^{s,s'}(i)
\end{align*}
\end{lemma}

\begin{theorem}\label{th:dgnexp}
 Under Assumption~\ref{assmp:main}
 \begin{align*}
\mathbf{E}_{\Theta_0}\left[K_0(s,s')\right]=\sigma^{2(d-1)}\sum_{i=1}^{d_{in}}x(i,s) x(i,s')\mu^{s,s'}(i)
\end{align*}

\end{theorem}
\begin{proof}
See Appendix.
\end{proof}

\begin{theorem}\label{th:dgnvar}
 Under Assumption~\ref{assmp:main} $Var\left[K_0\right]\leq $
\end{theorem}

}
% This document was modified from the file originally made available by
% Pat Langley and Andrea Danyluk for ICML-2K. This version was created
% by Iain Murray in 2018, and modified by Alexandre Bouchard in
% 2019. Previous contributors include Dan Roy, Lise Getoor and Tobias
% Scheffer, which was slightly modified from the 2010 version by
% Thorsten Joachims & Johannes Fuernkranz, slightly modified from the
% 2009 version by Kiri Wagstaff and Sam Roweis's 2008 version, which is
% slightly modified from Prasad Tadepalli's 2007 version which is a
% lightly changed version of the previous year's version by Andrew
% Moore, which was in turn edited from those of Kristian Kersting and
% Codrina Lauth. Alex Smola contributed to the algorithmic style files.

\subsection{Results in \Cref{sec:optimisation}}
\textbf{Stament and Proof of Lemma~\ref{lm:sigwire}}
\begin{lemma}[Signal vs Wire Decomposition]
Let $\kappa_t(s,s',i)\stackrel{def}=\underset{p_1,p_2\rsa i}{\sum_{p_1,p_2\in P:}} A_{\G_t}(x_s,p_1) A_{\G_t}(x_{s'},p_2) \ip{\varphi_{t,p_1}, \varphi_{t,p_2}}$. The Gram matrix $K_t$ is then given by 
\begin{align}\label{eq:ktalg}
{K_t(s,s')}=\sum_{i=1}^{d_{in}} x(i,s)x(i,s') \kappa_t(s,s',i)
\end{align}
\end{lemma}

\begin{proof}
Note that
\begin{align*}
\hat{y}_{t}(x_s)=\sum_{p\in\P}x(p(0),s) A(x_s,p) w_t(p)
\end{align*}
Differentiating with respect to any of the weights $\theta(m),m\in[d_{net}]$, we have
\begin{align*}
\frac{\partial \hat{y}_{t}(x_s)}{\partial \theta(m)}&=\frac{\partial \sum_{p\in\P}x(p(0),s) A(x_s,p) w_t(p)}{\partial \theta(m)}\\
\Psi_t(m,s)&=\sum_{p\in\P}x(p(0),s) A(x_s,p) \frac{\partial w_t(p)}{\partial \theta(m)}\\
&=\sum_{p\in\P}x(p(0),s) A(x_s,p) \varphi_{t,p}(m)
\end{align*}

Since, only the path strengths are changing, the Gram matrix $K_t$ is given by 
\begin{align*}
K_t(s,s')&={\Psi_t(\cdot,s)}^\top \Psi_t(\cdot,s')\\
&=\sum_{m=1}^{d_{net}} \Psi_t(m,s) \Psi_t(m,s')\\
&=\sum_{m=1}^{d_{net}} \left(\sum_{p_1\in\P}x(p_1(0),s) A(x_s,p_1) \varphi_{t,p_1}(m)\right)\left(\sum_{p_2\in\P}x(p_2(0),s') A(x_{s'},p_2) \varphi_{t,p_2}(m)\right)\\
&=\sum_{m=1}^{d_{net}} \underset{p_1,p_2\rsa\theta(m)}{\sum_{p_1,p_2\in P:}} x(p_1(0),s) A(x_s,p_1)x(p_2(0),s') A(x_{s'},p_2) \varphi_{t,p_1}(m) \varphi_{t,p_2}(m)\\
&=\sum_{i=1}^{d_{in}}\underset{p_1,p_2\rsa i}{\sum_{p_1,p_2\in P:}} x(p_1(0),s) A(x_s,p_1)x(p_2(0),s') A(x_{s'},p_2) \ip{\varphi_{t,p_1}, \varphi_{t,p_2}}\\
&=\sum_{i=1}^{d_{in}}\underset{p_1,p_2\rsa i}{\sum_{p_1,p_2\in P:}} x(i,s) A(x_s,p_1)x(i,s') A(x_{s'},p_2) \ip{\varphi_{t,p_1}, \varphi_{t,p_2}}\\
&=\sum_{i=1}^{d_{in}} x(i,s)x(i,s') \underset{p_1,p_2\rsa i}{\sum_{p_1,p_2\in P:}} A(x_s,p_1) A(x_{s'},p_2) \ip{\varphi_{t,p_1}, \varphi_{t,p_2}}
\end{align*}
\end{proof}


\textbf{Statement and Proof of Lemma~\ref{lm:pathdot}}
\begin{lemma}
Under Assumption~\ref{assmp:mainone}, for paths $p,p_1,p_2\in \P, p_1\neq p_2$, at initialisation we have (i) $\E{\ip{\varphi_{0,p_1}, \varphi_{0,p_2}}}= 0$, (ii) ${\ip{\varphi_{0,p}, \varphi_{0,p}}}= d\sigma^{2(d-1)}$
\end{lemma}

\begin{proof}
\begin{align*}
\ip{\varphi_{t,p_1}, \varphi_{t,p_2}}= \sum_{m=1}^{d_{net}} \varphi_{t,p_1}(m)\varphi_{t,p_2}(m)
\end{align*}
Let $\theta(m),m\in[d_{net}]$ be any weight such that $p\rsa \theta(m)$, and w.l.o.g let $\theta(m)$ belong to layer $l'\in[d]$. 
If either $p_1\bcancel{\rsa}\theta(m)$ or $p_2\bcancel{\rsa}\theta(m)$, then it follows that $\varphi_{t,p_1}(m)\varphi_{t,p_2}(m)=0$. In the case when $p_1,p_2\rsa\theta(m)$, we have
\begin{align*}
&\E{\varphi_{0,p_1}(m)\varphi_{0,p_2}(m)}\\
&=\E{\underset{l\neq l'}{\underset{l=1}{\overset{d}{\Pi}}} \Bigg(\Tb_0(l,p_1(l-1),p_1(l))\Tb_0(l,p_2(l-1),p_2(l)) \Bigg)}\\
&=\underset{l\neq l'}{\underset{l=1}{\overset{d}{\Pi}}}\E{\Tb_0(l,p_1(l-1),p_1(l))\Tb_0(l,p_2(l-1),p_2(l))}
\end{align*}
where the $\E{\cdot}$ moved inside the product because at initialisation the weights (of different layers) are independent of each other.
Since $p_1\neq p_2$, in one of the layers $\tilde{l}\in[d-1],\tilde{l}\neq l'$ they do not pass through the same weight, i.e., $\Tb_0(\tilde{l},p_1(\tilde{l}-1),p_1(\tilde{l}))$ and $\Tb_0(\tilde{l},p_2(\tilde{l}-1),p_2(\tilde{l}))$ are distinct weights. Using this fact
\begin{align*}
&\E{\varphi_{0,p_1}(m)\varphi_{0,p_2}(m)}\\
&=\underset{l\neq l',\tilde{l}}{\underset{l=1}{\overset{d}{\Pi}}}\E{\Tb_0(l,p_1(l-1),p_1(l))\Tb_0(l,p_2(l-1),p_2(l))}\\
&=\E{\Tb_0(\tilde{l},p_1(\tilde{l}-1),p_1(\tilde{l}))}\E{\Tb_0(\tilde{l},p_2(\tilde{l}-1),p_2(\tilde{l}))}\\
&=0
\end{align*}

The proof of (ii) is complete by noting that $\sum_{m=1}^{d_{net}} \varphi_{t,p}(m)\varphi_{t,p}(m)$ has $d$ non-zero terms for a single path $p$ and at initialisation we have 
\begin{align*}
&{\varphi_{0,p}(m)\varphi_{0,p}(m)}\\
&={\underset{l\neq l'}{\underset{l=1}{\overset{d}{\Pi}}} \Tb^2_0(l,p(l-1),p(l))}\\
&=\sigma^{2(d-1)}
\end{align*}
\end{proof}

\textbf{Statement and Proof of Theorem~\ref{th:dgnexp}}
\begin{theorem}[DIP in DGN]
Under Assumption~\ref{assmp:mainone}, ~\ref{assmp:maintwo}, and $\frac{4d}{w^2}<1$ it follows that
 \begin{align*}
&\E{K_0}=d\sigma^{2(d-1)}(x^\top x \odot \lambda_0)\\
&Var\left[K_0\right]\leq O\left(d^2_{in}\sigma^{4(d-1)}\max\{d^2w^{2(d-2)+1}, d^3w^{2(d-2)}\}\right)
\end{align*}
\end{theorem}

\begin{proof}
The first of the above two claims follow from the algebraic expression for $K_t$ and Lemma~\ref{lm:pathdot}. We now look at the variance calculation. The idea is that we expand  $Var\left[K_0(s,s')\right]=\E{K_0(s,s')^2} -\E{K_0(s,s')}^2$ and identify the the terms which cancel due to subtraction and then bound the rest of the terms.% Further, in what follows, we will assume $d_{in}=1$ without loss of generality.

Let $\theta(m)$ belong to layer $l'(m)$, then 
\begin{align}\label{eq:kexpect}
\E{K_0(s,s')}&=\sum_{m=1}^{d_{net}}\E{\left(\sum_{p_1 \in P}x(p_1(0),s)A(s,p_1)\frac{\partial w_{\Tb}(p_1)}{\partial \theta(m)}\right)\left(\sum_{p_2\in P}x(p_2(0),s)A(s',p_2)\frac{\partial w_{\Tb}(p_2)}{\partial \theta(m)}\right)}\nn\\
&=\sum_{m=1}^{d_{net}}\E{\sum_{p_1,p_2\in P}x(p_1(0),s)A(s,p_1)\frac{\partial w_{\Tb}(p_1)}{\partial \theta(m)}x(p_2(0),s')A(s',p_2)\frac{\partial w_{\Tb}(p_2)}{\partial \theta(m)}}\nn\\
&=\sum_{m=1}^{d_{net}}\underset{p_1,p_2\rsa\theta(m)}{\sum_{p_1,p_2\in P}}x(p_1(0),s)A(s,p_1)x(p_2(0),s')A(s',p_2) \E{\underset{l\neq l'(m)}{\underset{l=1}{\overset{d-1}{\Pi}}} \Tb_0(l,p_1(l-1),p_1(l)) \Tb_0(l,p_2(l-1),p_2(l))}\nn\\
&\stackrel{(a)}=\sum_{m=1}^{d_{net}}\underset{p_1,p_2\rsa\theta(m)}{\sum_{p_1,p_2\in P}}x(p_1(0),s)A(s,p_1)x(p_2(0),s')A(s',p_2) \underset{l\neq l'(m)}{\underset{l=1}{\overset{d-1}{\Pi}}} \E{\Tb_0(l,p_1(l-1),p_1(l)) \Tb_0(l,p_2(l-1),p_2(l))}
\end{align}
where $(a)$ follows from the fact that at initialisation the layer weights are independent of each other. Note that the right hand side of \eqref{eq:kexpect} only terms with $p_1=p_2$ will survive the expectation.

In the expression in \eqref{eq:kexpectsquare} note that $p_1=p_2$ and $p_3=p_4$.
\begin{align}\label{eq:kexpectsquare}
&\E{K_0(s,s')}^2=\nn\\
&\left(\sum_{m=1}^{d_{net}}\underset{p_1,p_2\rsa\theta(m)}{\sum_{p_1,p_2\in P}}x(p_1(0),s)A(s,p_1)x(p_2(0),s')A(s',p_2) \underset{l\neq l'(m)}{\underset{l=1}{\overset{d-1}{\Pi}}} \E{\Tb_0(l,p_1(l-1),p_1(l)) \Tb_0(l,p_2(l-1),p_2(l))}\right)\nn\\
&\left(\sum_{m'=1}^{d_{net}}\underset{p_3,p_4\rsa\theta(m')}{\sum_{p_3,p_4\in P}}x(p_3(0),s)A(s,p_3)x(p_4(0),s')A(s',p_4) \underset{l\neq l'(m')}{\underset{l=1}{\overset{d-1}{\Pi}}} \E{\Tb_0(l,p_3(l-1),p_3(l)) \Tb_0(l,p_4(l-1),p_4(l))}\right)\nn\\
&=\nn\\
&\sum_{m,m'=1}^{d_{net}}\underset{p_3,p_4\rsa\theta(m')}{\underset{p_1,p_2\rsa\theta(m)}{\sum_{p_1,p_2,p_3,p_4\in P}}}\Bigg[\bigg(x(p_1(0),s)A(s,p_1)x(p_2(0),s')A(s',p_2)x(p_3(0),s)A(s,p_3)x(p_4(0),s')A(s',p_4)\bigg)\nn\\
&\bigg( \underset{l\neq l'(m)} {\underset{l\neq l'(m')}{\underset{l=1}{\overset{d-1}{\Pi}}}} \E{\Tb_0(l,p_1(l-1),p_1(l)) \Tb_0(l,p_2(l-1),p_2(l))}\E{\Tb_0(l,p_3(l-1),p_3(l)) \Tb_0(l,p_4(l-1),p_4(l))} \bigg)\nn\\
&\bigg( \E{\Tb_0(l,p_1(l'(m')-1),p_1(l'(m'))) \Tb_0(l,p_2(l'(m')-1),p_2(l'(m')))}\bigg)\nn\\
&\bigg(\E{\Tb_0(l,p_3(l'(m)-1),p_3(l'(m))) \Tb_0(l,p_4(l'(m)-1),p_4(l'(m)))} \bigg)\Bigg]\nn\\
\end{align}

In the expression in \eqref{eq:ksquareexpect}, paths $p_1,p_2,p_3,p_4$ do not have constraints, and can be distinct.
\begin{align}\label{eq:ksquareexpect}
&\E{K^2_0(s,s')}=\nn\\
&\sum_{m,m'=1}^{d_{net}}\underset{p_3,p_4\rsa\theta(m')}{\underset{p_1,p_2\rsa\theta(m)}{\sum_{p_1,p_2,p_3,p_4\in P}}}\Bigg[\bigg(x(p_1(0),s)A(s,p_1)x(p_2(0),s')A(s',p_2)x(p_3(0),s)A(s,p_3)x(p_4(0),s')A(s',p_4)\bigg)\nn\\
&\bigg( \underset{l\neq l'(m)} {\underset{l\neq l'(m')}{\underset{l=1}{\overset{d-1}{\Pi}}}} \E{\Tb_0(l,p_1(l-1),p_1(l)) \Tb_0(l,p_2(l-1),p_2(l))\Tb_0(l,p_3(l-1),p_3(l)) \Tb_0(l,p_4(l-1),p_4(l))} \bigg)\nn\\
&\bigg( \E{\Tb_0(l,p_1(l'(m')-1),p_1(l'(m'))) \Tb_0(l,p_2(l'(m')-1),p_2(l'(m')))}\bigg)\nn\\
&\bigg(\E{\Tb_0(l,p_3(l'(m)-1),p_3(l'(m))) \Tb_0(l,p_4(l'(m)-1),p_4(l'(m)))} \bigg)\Bigg]\nn\\
\end{align}

We now state the following facts/observations.

$\bullet$ \textbf{Fact 1:} Any term that survives the expectation (i.e., does not become $0$) and participates in \eqref{eq:ksquareexpect} is of the form $\sigma^{4(d-1)}\big(x(p_1(0),s)A(s,p_1)x(p_2(0),s')A(s',p_2)x(p_3(0),s)A(s,p_3)x(p_4(0),s')A(s',p_4)\big)$, where $p_1,p_2,p_3,p_4$ are free variables, and participates in \eqref{eq:kexpectsquare} is of the form $\sigma^{4(d-1)}\big(x(p_1(0),s)A(s,p_1)x(p_2(0),s')A(s',p_2)x(p_3(0),s)A(s,p_3)x(p_4(0),s')A(s',p_4)\big)$, where $p_1=p_2,p_3=p_4$.

$\bullet$ \textbf{Fact 2:} The number of paths through a particular weight $\theta(m)$ in one of the middle layers is $d_{in}w^{d-3}$, and the number of paths through a particular weight $\theta(m)$ in either the first or the last layer is $d_{in}w^{d-2}$ .

$\bullet$ \textbf{Fact 3:} Let $\P'$ be an arbitrary set of paths constrained to pass through some set of weights. Let $\P''$ be the set of paths obtained by adding an additional constraint that the paths also should pass through a particular weight say $\theta(m)$. Now, if $\theta(m)$ belongs to :

$1.$ a middle layer, then $|\P''|=\frac{|\P'|}{w^2}$.

$2.$ the first layer or the last layer, then $|\P''|=\frac{|\P'|}{w}$.

$\bullet$ \textbf{Fact 4:} For any $p_1,p_2,p_3,p_4$ combination that survives the expectation in \eqref{eq:ksquareexpect} can be written as 

\begin{align*}
&\bigg( \underset{l\neq l'(m)} {\underset{l\neq l'(m')}{\underset{l=1}{\overset{d-1}{\Pi}}}} \E{\Tb_0(l,p_1(l-1),p_1(l)) \Tb_0(l,p_2(l-1),p_2(l))\Tb_0(l,p_3(l-1),p_3(l)) \Tb_0(l,p_4(l-1),p_4(l))} \bigg)\nn\\
&\bigg( \E{\Tb_0(l,p_1(l'(m')-1),p_1(l'(m'))) \Tb_0(l,p_2(l'(m')-1),p_2(l'(m')))}\bigg)\nn\\
&\bigg(\E{\Tb_0(l,p_3(l'(m)-1),p_3(l'(m))) \Tb_0(l,p_4(l'(m)-1),p_4(l'(m)))} \bigg)\Bigg]\nn\\
&=\\
&\bigg( \underset{l\neq l'(m)} {\underset{l\neq l'(m')}{\underset{l=1}{\overset{d-1}{\Pi}}}} \Tb^2_0(l,\rho_a(l-1),\rho_a(l)) \Tb^2_0(l,\rho_b(l-1),\rho_b(l)) \bigg)\nn\\
&\bigg( \Tb^2_0(l,\rho_a(l'(m')-1),\rho_a(l'(m')))\bigg)\nn\\
&\bigg({\Tb^2_0(l,\rho_b(l'(m)-1),\rho_b(l'(m)))} \bigg),
\end{align*}

where $\rho_a\rsa \theta(m)$ and $\rho_b\rsa \theta(m')$ are what we call as \emph{base} (case) paths. 


$\bullet$ \textbf{Fact 5:} For any given base paths $\rho_a$ and $\rho_b$ there could be multiple assignments possible for $p_1,p_2,p_3,p_4$.

$\bullet$ \textbf{Fact 6:}  Terms in \eqref{eq:ksquareexpect}, wherein, the base case is generated as $p_1=p_2=\rho_a$ and $p_3=p_4=\rho_b$ (or $p_1=p_2=\rho_b$ and $p_3=p_4=\rho_a$), get cancelled with the corresponding terms in \eqref{eq:kexpectsquare}.

$\bullet$ \textbf{Fact 7:}  When the bases paths $\rho_a$ and $\rho_b$ do not intersect (i.e., do not pass through the same weight in any one of the layers), the only possible assignment is $p_1=p_2=\rho_a$ and $p_3=p_4=\rho_b$ (or $p_1=p_2=\rho_b$ and $p_3=p_4=\rho_a$), and such terms are common in \eqref{eq:ksquareexpect} and \eqref{eq:kexpectsquare}, and hence do not show up in the variance term.


$\bullet$ \textbf{Fact 7:} Let base paths $\rho_a$ and $\rho_b$ cross at layer $l_1, \ldots, l_k, k \in [d-1]$, and let $\rho_a=(\rho_a(1),\ldots,\rho_a(k+1))$ where $\rho_a(1)$ is a sub-path string from layer $1$ to $l_1$, and $\rho_a(2)$ is the sub-path string from layer $l_1+1$ to $l_2$ and so on, and $\rho_a(k+1)$ is the sub-path string from layer $l_k+1$ to the output node. Then the set of paths that can occur in $\E{K_0(s,s')^2}$ are of the form:
\begin{enumerate}
\item $p_1=p_2=\rho_a, p_3=p_4=\rho_b$ (or $p_1=p_2=\rho_b, p_3=p_4=\rho_a$) which get cancelled in the $\E{K_0(s,s')}^2$ term.
\item $p_1=\rho_a$, $p_3=\rho_b$, $p_2=(\rho_b(1),\rho_a(2),\rho_a(3),\ldots,\rho_a(k+1))$, $p_4=(\rho_a(1),\rho_b(2),\rho_b(3),\ldots,\rho_b(k+1))$, which are obtained by splicing the base paths in various combinations. Note that for such spliced paths $p_1\neq p_2$ and $p_3\neq p_4$ and hence do not occur in the expression for $\E{K_0(s,s')}^2$ in \eqref{eq:kexpectsquare}.
\end{enumerate}


$\bullet$ \textbf{Fact 8:} For $k$ crossings of the base paths there are $4^{k+1}$ splicings possible, and those many terms are extra in the $\E{K_0(s,s')^2}$ calculation in \eqref{eq:ksquareexpect} comparison to the $\E{K_0(s,s')}^2$ calculation. We now enumerate cases of possible crossings, and reason out the magnitude of their contribution to the variance term using the \textbf{Fact 1} to \textbf{Fact 8}.


\textbf{Case $1$} $k=1$ crossing in either first or last layer. There are $2w$ weights in the first and the last layer, and the number of base path combinations is $w^{d-2}\times w^{d-2}$, and for each of these cases, $m,m'$ could take $O(d^2)$ possible values. And the multiplication of the weights themselves contribute to $\sigma^{4(d-1)}$. Putting them together we have
\begin{align*}
d^2_{in}\sigma^{4(d-1)}\times (2w)\times d^2\times (w^{d-2}\times w^{d-2})\times 4^2 = 32d^2_{in}\sigma^{4(d-1)}d^2 w^{2(d-2)+1}
\end{align*}

\textbf{Case $2$} $k=1$ crossing in one of the middle layers. There are $w^2(d-2)$ weights in the first and the last layer, and the number of base path combinations is $w^{d-3}\times w^{d-3}$, and for each of these cases, $m,m'$ could take $O(d^2)$ possible values. And the multiplication of the weights themselves contribute to $\sigma^{4(d-1)}$. Putting them together we have
\begin{align*}
d^2_{in}\sigma^{4(d-1)}\times w^2(d-2)\times d^2\times (w^{d-3}\times w^{d-3})\times 4^2\leq 16d^2_{in}\sigma^{4(d-1)} d^3 w^{2(d-3)}
\end{align*}

\textbf{Case $3$} $k=2$ crossings one in the first layer and other in the last layer. This case can be covered using Case $1$ and then further restricting that the base paths should also in the other layer. So, we have
\begin{align*}
32d^2_{in}\sigma^{4(d-1)}d^2 w^{2(d-2)+1} \times \underbrace{w}_{\text{possible weights in other layer}} \times \underbrace{w^{-1}\times w^{-1}}_{\text{reduction in paths due to additional restriction}} \times 4 = (32d^2_{in}\sigma^{4(d-1)}d^2 w^{2(d-2)+1})\times (4w^{-1}),
\end{align*}
where the $4$ is for the $4$ extra possible ways of splicing the base paths.

\textbf{Case $4$} $k=2$ crossings first one in the first layer or the last layer, and the second one in the middle layer. This can be obtained by looking at the Case $1$ and then adding the further restriction that the base paths should cross each other in the middle layer. 
\begin{align*}
32d^2_{in}\sigma^{4(d-1)}d^2 w^{2(d-2)+1}\times w^2(d-2) \times (w^{-2}w^{-2}) \times 4= (32d^2_{in}\sigma^{4(d-1)}d^2 w^{2(d-2)+1} )\times (4dw^{-2}) 
\end{align*}

\textbf{Case $5$} $k=2$ crossings in the middle layer. This can be obtained by taking Case $2$ and then adding the further restriction that the base paths should cross each other in the middle layer. 
\begin{align*}
16d^2_{in}\sigma^{4(d-1)} d^3 w^{2(d-3)}\times w^2(d-2) w^{-2}w^{-2}\times 4\leq (16d^2_{in}\sigma^{4(d-1)} d^3 w^{2(d-3)}) \times (4dw^{-2})
\end{align*}


\textbf{Case $6$} $k=3$ crossings first one in the first layer or the last layer, and the other two in the middle layers. This can be obtained by considering Case $4$ and then adding the further restriction that the base paths should cross each other in the middle layer. 
\begin{align*}
(32d^2_{in}\sigma^{4(d-1)}d^2 w^{2(d-2)+1} )\times (4dw^{-2}) \times (4dw^{-2}) 
\end{align*}

\textbf{Case $7$} $k=3$ crossings first two in the first and last layers and the third one in the middle layers. This can be obtained by considering Case $3$ and then adding the further restriction that the base paths should cross each other in the middle layer. 

\begin{align*}
(32d^2_{in}\sigma^{4(d-1)}d^2 w^{2(d-2)+1})\times (4w^{-1})\times (4dw^{-2}) 
\end{align*}

\textbf{Case $8$} $k=3$ crossings in the middle layer. This can be obtained by considering Case $5$ and then adding the further restriction that the base paths should cross each other in the middle layer. 
\begin{align*}
 (16d^2_{in}\sigma^{4(d-1)} d^3 w^{2(d-3)}) \times (4dw^{-2})\times (4dw^{-2}) 
\end{align*}


The cases can be extended in a similar way, increasing the number of crossings.  Now, assuming $\frac{4d}{w^2}<1$, the bounds in the various terms can be lumped together as below:

$\bullet$ We can add the bounds for Case $1$, Case $4$, Case $6$ and other cases obtained by adding more crossings (one at a time) in the middle layer to Case $6$. This gives rise to a term which is upper bounded by 
\begin{align*}
d^2_{in}\sigma^{4(d-1)}d^2w^{2(d-2)+1}\left(\frac{1}{1-4dw^{-2}}\right)
\end{align*}

$\bullet$ We can add the bounds for Case $3$, Case $7$ and other cases obtained by adding more crossings (one at a time) in the middle layer to Case $6$. This gives rise to a term which is upper bounded by 
\begin{align*}
d^2_{in}\sigma^{4(d-1)}d^3w^{2(d-2)} \left(\frac{1}{1-4dw^{-2}}\right)
\end{align*}



$\bullet$ We can add the bounds for Case $2$, Case $5$, Case $8$ and other cases obtained by adding more crossings (one at a time) in the middle layer to Case $6$. This gives rise to a term which is upper bounded by 
\begin{align*}
d^2_{in}\sigma^{4(d-1)}d^2w^{2(d-2)} \left(\frac{1}{1-4dw^{-2}}\right)
\end{align*}

Putting together we have the variance to be bounded by 
\begin{align*}
Cd^2_{in}\sigma^{4(d-1)}\max\{d^2w^{2(d-2)+1}, d^3w^{2(d-2)}\},
\end{align*}
for some constant $C>0$.
\end{proof}


\textbf{Statement and Proof of Lemma~\ref{lm:dgn-fra}}
\begin{lemma}
 Under Assumption~\ref{assmp:mainone},~\ref{assmp:maintwo} and gates sampled iid $Ber(\mu)$, we have, $\forall s,s'\in[n]$

(i) $\mathbb{E}_p\left[\lambda_0(s,s)\right]=\bar{\lambda}_{self}=(\mu w)^{d-1}$

ii) $\mathbb{E}_p\left[\lambda_0(s,s')\right]=\bar{\lambda}_{cross}= (\mu^2w)^{d-1}$
\end{lemma}

\begin{proof}
The proof of (i) follows by noting that the average number of gates that are \emph{on} in each layer is $(\mu w)$, and there are $(\mu w)^{d-1}$ paths starting from a given input node $i\in[d_{in}]$ and ending at the output node. The proof of (ii) follow by noting that on an average $\mu^2w$ gates overlap per layer for two different inputs.
\end{proof}



\begin{lemma} 
Under Assumptions~\ref{assmp:mainone},~\ref{assmp:maintwo}, in soft-GaLU networks we have: (i) $\E{K_0}=\E{K^w_0}+\E{K^a_0}$, 
 (ii) $\E{K^w_0}=\sigma^{2(d-1)} (x^\top x)\odot \lambda$,  (iii) $\E{K^a_0}=\sigma^{2d}  (x^\top x)\odot \delta$
\end{lemma}

\begin{proof}
Follows from Lemma~\ref{lm:pathdot}, and noting that $\Tg_0$ and $\Tw_0$ are iid.
\end{proof}

\section{Deep Linear Networks}

In this case, $G(s,l,i)=1,\forall s\in[n],i\in[w],l\in[d-1]$. Note that all the paths are always active irrespective of which input is presented to the DLN. We can define the effective weight that multiplies each of the input dimensions as 
\begin{align}
\eta_{t}(i)\stackrel{def}= \sum_{p\in P: p(0)=i} w_{t}(p), i\in [d_{in}]
\end{align}
Using the above definition of $\eta=(\eta(i),i\in[d_{in}])\in \R^{d_{in}}$, the hidden feature representation can be simplified as 
\begin{align}
\hat{y}_t&=\Phi^\top_{x,1_{\dagger}} w_{t} \\&=x^\top \eta_{t}
\end{align}
 Thus it is clear that the DLN does not provide any high dimensional feature representation and the input features are retained as such. All that the depth adds is just a non-linear re-parameterisation of the weights. It also follows that $\lambda_0(s,s')=w^{d-1},\forall ,s,s'\in [n]$.

\begin{corollary}\label{th:dln} Under Assumption~\ref{assmp:maintwo}, for a DLN with $d_{in}=1$, and dataset with $n=1$ we have, \begin{align} \mathbf{E}_{\Tb}\left[K_0\right]=d(w\sigma^2)^{(d-1)}\end{align}
\end{corollary}


\textbf{Experiment 8:} We consider a dataset with $n=1$ and $(x,y)=(1,1)$, i.e., $d_{in}=1$, let $w=100$ and look at various value of depth namely  $d=2,4,6,8,10$. We set $\sigma=\sqrt{\frac{1}{w}}$ and the weights are drawn according to Assumption~\ref{assmp:mainone}. We set the learning rate to be $\alpha=\frac{0.1}{d}$, and for this setting we expect the error dynamics to be the following $\frac{e^2_{t+1}}{e^2_t}=0.81$.
\comment{
\begin{align*}
e_{t+1}&=e_t-\frac{0.1}{d}d(w\sigma^2)^{2(d-1)}e_t\\
&=0.9^te_t
\end{align*}
}
The results are shown in \Cref{fig:dln}. We observe that irrespective of the depth the error dynamics is similar (since $\alpha=\frac{0.1}{d}$ ). However, we observe faster (in comparison to the ideal rate of $0.81$) convergence of error to zero since the magnitude of $K_t$ increases with time (see \Cref{fig:dln}).

\FloatBarrier
\begin{figure*}[h]
\resizebox{\textwidth}{!}{
\begin{tabular}{cccc}
\includegraphics[scale=0.4]{figs/dln-k-vs-d.png}
&
\includegraphics[scale=0.4]{figs/dln-conv.png}
&
\includegraphics[scale=0.4]{figs/dln-gram.png}
&
\includegraphics[scale=0.4]{figs/dln-norm.png}
\end{tabular}
}
\caption{In all the plots $d_{in}=1, n=1, w=100,\sigma^2=\frac{1}{w}$ averaged over $5$ runs. The left most plot shows $K_0$ as a function of depth. The second from left plot shows the convergence rate. The third plot from left shows the growth of $K_t$ over the course of training, and the right most plot shows the growth of weights ($L_2$-norm) with respect to time.}
\label{fig:dln}
\end{figure*}

\comment{\section{DGN-FRG}
\textbf{Effect of $p$} is shown in \Cref{fig:peff}. For $w=100$, we observe that for  the e.c.d.f gets better as the value of $p$ reduces till $p=0.3$, after which it starts degrading. This is due to the fact that the variance gets worse with $\frac{1}p$ (since $\sigma=\sqrt{\frac{1}{pw}}$. It can be seen that for $w=50$ the variance is more and hence the e.c.d.f gets better as we reduce $p$ only till $p=0.4$, after which it starts to degrade.

\FloatBarrier
\begin{figure*}[h]
\resizebox{\columnwidth}{!}{
\begin{tabular}{cc}
\includegraphics[scale=0.4]{figs/dgn-frg-ecdf-p-w100.png}
&
\includegraphics[scale=0.4]{figs/dgn-frg-ecdf-p-w10.png}
\end{tabular}
}
\caption{Shows e.c.d.f for various values of $p$.}
\label{fig:peff}
\end{figure*}}


\textbf{Statement and Proof of Lemma~\ref{lm:invariance}}
\begin{lemma}
At $t=0$, under Assumptions~\ref{assmp:mainone},\ref{assmp:maintwo}, convolutional layers with global average pooling at the end causes translational invariance.
\begin{align*}
&\E{x_s(L,1)x_{s'}(L,1)}\\&=\frac{\sigma^{2(d-1)}}{d^2_{in}}\sum_{k=1}^{\hat{B}} \sum_{p_1,p_2\in b_k}  \Big( x(p_1(0),s) A(x_s,p_1)\\
&\quad\quad \quad\quad \quad\quad x(p_2(0),s') A(x_{s'},p_2) \Big)
\end{align*}
\end{lemma}

\begin{proof}
\begin{align*}
\E{x_s(L,1)x_{s'}(L,1)}&=\E{\phi^\top_{x_s,\G_0} w_0w_0^\top \phi^\top_{x_{s'},\G_0}}\\
&=\phi^\top_{x_s,\G_0}\E{ w_0 w_0^\top} \phi^\top_{x_{s'},\G_0},
\end{align*}
where we use the fact that the gates $\G_0$ are statistically independent of the weights. Now let $M=\E{ w_0 w_0^\top}$, we make the following observations about $M$:

$1.$ $M(p_1,p_2)=0$, if $p_1$ and $p_2$ belong to the different bundles.

$2.$ $M(p_1,p_2)=\frac{\sigma^{2(d-1)}}{d^2_{in}}$, if $p_1$ and $p_2$ belong to the same bundle.

Using the above two observations, we have at  $t=0$:

\begin{align*}
&\E{x_s(L,1)x_{s'}(L,1)}\\&=\phi^\top_{x_s,\G_0} M \phi^\top_{x_{s'},\G_0}\\
&=\sum_{p_1,p_2=1}^{\hat{P}} \Big(x(p_1(0),s) A(x_s,p_1) \\
&\quad\quad \quad\quad \quad\quad x(p_2(0),s') A(x_{s'},p_2) M(p_1,p_2)\Big)\\
&=\frac{\sigma^{2(d-1)}}{d^2_{in}}\sum_{k=1}^{\hat{B}} \sum_{p_1,p_2\in b_k}  \Big( x(p_1(0),s) A(x_s,p_1)\\
&\quad\quad \quad\quad \quad\quad x(p_2(0),s') A(x_{s'},p_2) \Big)
\end{align*}
\end{proof}





\end{document}
