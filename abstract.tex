\begin{abstract}
Understanding the role of (stochastic) gradient descent (SGD) in the training and generalisation of deep neural networks (DNNs) with ReLU activation has been the object study in the recent past. In this paper, we make use of deep gated networks (DGNs) as a framework to obtain insights about DNNs with ReLU activation. In DGNs, a single neuronal unit has two components namely the pre-activation input (equal to the inner product the weights of the layer and the previous layer outputs), and a gating value which belongs to $[0,1]$ and the output of the neuronal unit is equal to the multiplication of pre-activation input and the gating value. The standard DNN with ReLU activation, is a special case of the DGNs, wherein the gating value is $1/0$ based on whether or not the pre-activation input is positive or negative. We theoretically analyse and experiment with several variants of DGNs, each variant suited to understand a particular aspect of either training or generalisation in DNNs with ReLU activation.
Our theory throws light on two questions namely i) why increasing depth till a point helps in training and ii) why increasing depth beyond a point hurts training? We also present experimental evidence to show that gate adaptation, i.e., the change of gating value through the course of training is key for generalisation.
\end{abstract}