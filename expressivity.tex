\section{Information Flow in Deep Gated Networks}
\textbf{Paths:} We have a total of $P=d_{in}w^{(d-1)}$ paths. Let us say that an enumeration of the paths is given by $[P]=\{1,\ldots,P\}$. Let $\I_{l}\colon [P]\ra [w],l=0,\ldots,d-1$ provide the index of the hidden unit through which a path $p$ passes in layer $l$ (with the convention that $\I_d(p)=1,\forall p\in [P]$). We then define:\\
$1.$ The value of a path $p$ by $v_t(p)\stackrel{def}=\Pi_{l=1}^d \Theta_t(l,\I_{l-1}(p),\I_l(p))$.\\
$2.$ The activity of a path $p$ for an input $x_s\in \R^{d_{in}}$ by $A_{t}(x,p)\stackrel{def}{=}\Pi_{l=1}^{d-1} G_{x_s,t}(l,\I_l(p))$.\\
\begin{comment}
\FloatBarrier
\begin{table}[h]
\begin{minipage}{0.5\columnwidth}
\resizebox{\columnwidth}{!}{
\begin{tabular}{|c|l|}\hline								 								 													
NPF		&$\phi_{x,t}=(x(\I_0(p))A_t(x,p) ,p\in[P])\in \R^P$\\\hline	
OASN	&$\lambda_t(x,x')=\sum_{p\rsa i} A_t(x,p) A_t(x',p)$\\\hline
NPK		&$H_t(x,x')=\ip{\phi_{x,t},\phi_{x',t}}$\\\hline		
VTP		&$\varphi^v_{p,t}=(\partial_{\theta}v_t(p),\theta\in\Theta)\inrdnet$ \\\hline	
VG		&$\psi^v_{x,t}=\nabla_{\Theta} \hat{y}_t(x)\in\R^{d_{net}}$\\\hline
\end{tabular}
}
\end{minipage}
%\hspace{15pt}
\begin{minipage}{0.5\columnwidth}
\resizebox{\columnwidth}{!}{
\begin{tabular}{|c|l|}\hline								 								 													
ATP		&$\varphi^a_{x,p,t}=(\partial_{\tg}A_t(x,p),\tg\in\Tg)\inrdnet$ \\\hline	
FG		&$\psi^{\phi}_{x,t}=\nabla_{\Tg} \hat{y}_t(x)\in\R^{d_{net}}$\\\hline
OSSN 	&$\delta_t(x,x')=\sum_{p\rsa i} \ip{\varphi^a_{x,p,t},\varphi^a_{x',p,t}}$\\\hline
NTF		&$\psi_{x,t}=(\psi^v_{x,t},\psi^{\phi}_{x,t})\in\R^{2d_{net}}$\\\hline
NTK 		&$K_t(x,x')=\ip{\psi_{x,t}\psi_{x',t}}$\\\hline
\end{tabular}
}
\end{minipage}
\caption{Shows all zeroth-order and first-order quantities related to information flow in a DGN.}
\label{tb:terms}
\end{table}
\end{comment}
\subsection{Zeroth-Order: Active Sub-Network, Neural Path Feature and Kernel}
The \emph{neural path feature (NPF)} of an input example $x_s\in \R^{d_{in}}$ is given by $\phi_{x_s,\G_t}=(x_s(\I_0(p))A_t(x_s,p) ,p\in[P])\in\R^P$. Here, for a path $p$, $\I_0(p)$ is the input node at which the path starts, and $A_t(x_s,p)$ is its activity. By arranging the NPF of the $n$ input examples in a matrix $\Phi_t=(\phi_{x_s,\G_t},s\in[n])\in\R^{P\times n}$, we can express the predicted output of a DGN as: 
\begin{align}\label{eq:npfbasic}
\hat{y}_t=\Phi_t^\top v_t,
\end{align}
where, the value of the path $v_t$ is the equivalent of the so called \emph{weight-vector} in a standard linear approximation. 
The significance of the NPF is that it separates the \emph{signal} form the \emph{wire}. Note that $\Phi$ encodes the signal: say for a DNN with ReLU activations, the co-ordinate corresponding to path $p$ is either $x(\I_0(p))$ if the path is active for that input (i.e., $A_t(x,p)=1$) or $0$ if the path is inactive for that input  (i.e., $A_t(x,p)=0$). The value of the path encodes the \emph{wire}, i.e., the information contained in the weights of the network.
% \textbf{Active Sub-Network:}  For input $x_s\in\R^{d_{in}}$, let $\N^{\A}_{x_s,t}(\tau_{\A})=\{p\in[P]:A_t(x,p)>\tau_{\A}\}$ be the \emph{active sub-network} comprising of the set of paths whose activity is greater than some threshold value $\tau_{\A}>0$. In the case of ReLU activations, $\N^{\A}_{x_s,t}(0)$ completely determines the NPF $\phi_{x_s,\G_t}$.  $\lambda_t(x_s,x_{s'})$ is a measure of overlap of active sub-networks for input examples $s,s'\in[n]$.
\begin{lemma}\label{lm:npk}
Let $\lambda_t(s,s')\stackrel{def}{=}\sum_{p\rsa i} A_t(x_s,p) A_t(x_{s'},p)$, $\forall s,s'\in[n]$, any $i\in [d_{in}]$ and let the \textbf{neural path kernel} (NPK) matrix is defined as $H_t\stackrel{def}=\Phi^\top_t\Phi_t$. It follows that $H_t= (x^\top x')\odot\lambda_t$, where $\odot$ stands for the Hadamard product.
\end{lemma}
\textbf{Active Sub-Network:} Let $\N^{\A}_{x_s,t}(\tau_{\A})=\{p\in[P]:A_t(x,p)>\tau_{\A}\}$ be the \emph{active sub-network} comprising of the set of paths whose activity is greater than some threshold value $\tau_{\A}>0$. In the case of hard gates, $\N^{\A}_{x_s,t}(0)$ denotes the set of all the active paths, and $\lambda_t(s,s')$ in \Cref{lm:npk} denotes the overlap of the sub-networks that are active for both $s,s'\in[n]$.\\
\subsection{First-Order: Sensitive Sub-Network, Twin Gradients, Neural Tangent Feature and Kernel}
Consider a DGN, wherein, the gating network is parameterised by $\Tg\inrdnet$ and $\beta>0$.  We have 
\begin{align*}\nabla_{\Theta,\Tg}\hat{y}_t(x_s)=\psi_{x_s,t}=(\psi^v_{x_s,t}, \psi^{\phi}_{x_s,t})=\left((\nabla_{\Theta|_{\Theta=\Theta_t}}v_{\Theta})^\top \Phi_{\Tg_t}, (\nabla_{\Tg|_{\Tg=\Tg_t}}\Phi_{\Tg_t})^\top v_{\Theta_t}\right)\in \R^{2d_{net}}
\end{align*}
\textbf{Sensitive Sub-Network:} Let $\N^{\S}_{x,t}(\tau_{\S})=\{p\in[P]: |\partial_{\tg} A_t(x,p)|>\tau_{\S}\}$ be sensitive sub-network, whose path activity gradient with respect to any of $\Tg\inrdnet$ is greater than some threshold $\tau_{\S}>0$. Using the property that the slope of the sigmoid diminishes in the extremities, we can reason that for appropriate choices of $\tau_{\A}$ (sufficiently close to $1$) and $\tau_{\S}$ (large enough), $\N^{\S}(\tau_{S})\cup \N^{\A}(\tau_{\A})=\emptyset$. Thus, there are two separate networks for the two separate gradient flows, i.e., the value gradient flow via the active sub-network and the feature gradient flow via the sensitive sub-network.\\
\textbf{Value and Feature gradients:} The \emph{value gradient}  $\psi^v_{x,t}$ flows through the network of active paths. For a path $p$, the derivative of its value with respect to any weight in the path is given by:
${\partial_{\Theta\left(l,\I_{l'-1}(p),\I_{l'}(p)\right)} v_t(p)}|_{\Theta=\Theta_t}= {\underset{l=1,l\neq l'}{\overset{d}{\Pi}}} \Theta_t\left(l,\I_{l-1}(p),\I_{l}(p)\right)
$.
Note that, if a path $p$ does not pass through a $\theta\in\Theta$, then ${\partial v_t(p)}/{\partial \theta}=0$. Further, since $\psi^v_{x,t}=(\nabla_{\Theta|_{\Theta=\Theta_t}}v_{\Theta})^\top \Phi_{\Tg_t}$, the value derivatives of those paths which are not active, i.e., $A_t(x_s,p)$ close to $0$ will not contribute significantly in $\psi^v_{x,t}$.\\
The \emph{feature gradient}  $\psi^{\phi}_{x,t}$ flows through the network of sensitive paths. Note that for a weight $\tg\in \Tg$, $\partial_{\tg} \phi_{x_s,t}(p) = x(\I_0(p))\partial_{\tg} A_t(x_s,p)$, wherein:
$
{\partial_{\tg} A_{t}(x,p)}= \sum_{l=1}^{d-1} \Big({\partial_{\tg} G_{x,\Tg_t}(l,\I_l(p))} \Big)\Big(\Pi_{l'\neq l} G_{x,\Tg_t}(l',I_{l'}(p))\Big)
$.
Note that if all the activations in a path are close to $0$ or $1$, then $\partial_{\tg} A_t(x_s,p)$ is close to $0$.\\
\textbf{ReLU artefact:} In the case of $\beta=\infty$ (see \Cref{tb:dgn}), i.e, when the gating values belong to $\{0,1\}$, it follows that the activity $A_t\in\{0,1\}$ and hence $\partial_{\tg} A_t(\cdot,\cdot)=0$. Thus, in the case of DNNs with ReLU activations, there is no flow of the feature gradient, however, since the gating parameter $\Tg$ identical with $\Theta$, (i.e., $\Tg_t=\Theta_t,\forall t$), $A_t$ changes with time and so does the NPF. \\
\begin{comment}
\begin{definition}\label{def:delta}
$\delta_t(s,s')=\sum_{p\rsa i} \ip{\varphi^a_{x_s,p,t},\varphi^a_{x_{s'},p,t}}$, for $s,s'\in[n]$, using any $i\in[d_{in}]$.
\end{definition}
\textbf{Gradient Flow via Sub-Networks:} Note that in the case of $\beta=\infty$ (see \Cref{tb:dgn}), i.e, when the gating values belong to $\{0,1\}$, it follows that the activity $A_t\in\{0,1\}$, and hence AD and FG are $0$. Thus, in the case of DNNs with ReLU activations, there is no flow of the feature gradient, however, since the gating parameter $\Tg$ identical with $\Theta$, (i.e., $\Tg_t=\Theta_t,\forall t$), $A_t$ changes with time and so does the NPF. In the case of $\beta>0$, for an input $x\in\R^{d_{in}}$, let $\N^{\S}_{x,t}(\tau_{\S})=\{p\in[P]: |\partial_{\tg} A_t(x,p)|>\tau_{\S}\}$ be the set of paths, which have activations whose gradient to any of $\Tg\inrdnet$ is greater than some threshold $\tau>0$. Note that when all the gates are close to $1$ in a path $p$, then it follows that $\partial A_t(x_s,p)$ will be very small. Thus, using the property that the slope of the sigmoid diminishes in the extremities, we can reason that for appropriate choices of $\tau_{\A}$ (sufficiently close to $1$) and $\tau_{\S}$ (large enough), $\P^{\S}(\tau_{S})\cup \P^{\A}(\tau_{\A})=\emptyset$. Thus, there are two separate networks for the two separate gradient flows, i.e., the value gradient flow via the active sub-network and the feature gradient flow via the sensitive sub-network.\\
\end{comment}
\textbf{Neural Tangent Feature and Kernel:} Let $\Psi^v_t=(\psi^v_{x_s,t},s\in[n])\in\R^{d_{net}\times n}$ and $\Psi^{\phi}_t=(\psi^{\phi}_{x_s,t},s\in[n])\in\R^{d_{net}\times n}$ be the NTF matrices, then the corresponding NTK matrices are given in \Cref{tb:ntks}
\FloatBarrier
\begin{table}[h]
\begin{minipage}{0.5\columnwidth}
%\resizebox{\columnwidth}{!}{
\begin{tabular}{|c|l|}\hline								 								 													
NTF		&$\psi_{x_s,t}=(\psi^v_{x_s,t},\psi^{\phi}_{x_s,t})\in\R^{2d_{net}}$\\\hline
NTK 		&$K_t=K^v_t+K^{\phi}_t$\\\hline
\end{tabular}
%}
\end{minipage}
%\hspace{15pt}
\begin{minipage}{0.5\columnwidth}
%\resizebox{\columnwidth}{!}{
\begin{tabular}{|c|l|}\hline								 								 													
NTF		&$\psi_{x_s,t}=(\phi^v_{x_s,t}+\psi^{\phi}_{x_s,t})\in\R^{d_{net}}$\\\hline
NTK 		&$K_t=K^v_t+K^{\phi}_t+{\Psi^{\phi}}^\top_t\Psi^v_t+ {\Psi^{v}}^\top_t\Psi^{\phi}_t$\\\hline
\end{tabular}
%}
\end{minipage}
\caption{Shows NTK and NTF for the distinct ($\Tg\neq\Theta$) on the left and shared parameterisation ($\Tg=\Theta$) on the right. Here, $K^v_t={\Psi^{v}}^\top_t\Psi^v_t$ and $K^{\phi}_t={\Psi^{\phi}}^\top_t\Psi^{\phi}_t$.}
\label{tb:ntks}
\end{table}

\begin{comment}
The \textbf{neural tangent feature} (NTF) is given by $\psi_{x_s,t}=(\psi^v_{x_s,t},\psi^{\phi}_{x_s,t})\in\R^{2d_{net}}$, when the gating network is parameterised by $\Tg$ which is distinct from $\Theta$. When the parameters are shared ($\Tg_t=\Theta_t\forall t\geq 0$), then we have $\psi_{x_s,t}=(\phi^v_{x_s,t}+\psi^{\phi}_{x_s,t})\in\R^{d_{net}}$.\\
When the gating network is parameterised by $\Tg$ which is distinct from $\Theta$, the \textbf{neural tangent kernel} (NTK) matrix is given by $K_t=K^v_t+K^{\phi}_t$, where $K^v_t={\Psi^{v}}^\top_t\Psi^v_t$ and $K^{\phi}_t={\Psi^{\phi}}^\top_t\Psi^{\phi}_t$, with $\Psi^v_t=(\psi^v_{x_s,t},s\in[n])\in\R^{d_{net}\times n}$ and $\Psi^{\phi}_t=(\psi^{\phi}_{x_s,t},s\in[n])\in\R^{d_{net}\times n}$. In the case of shared parameterisation ($\Tg_t=\Theta_t\forall t\geq 0$), we have $K_t=K^v_t+K^{\phi}_t+{\Psi^{\phi}}^\top_t\Psi^v_t+ {\Psi^{v}}^\top_t\Psi^{\phi}_t$.
\end{comment}