\section{Paths, Features and Kernels}\label{sec:expressivity}
\textbf{Paths:} In a fully connected DGN, $d$ layers and $w$ hidden units per layer, we have a total of $P=d_{in}w^{(d-1)}$ paths. Let us say that an enumeration of the paths is given by $[P]=\{1,\ldots,P\}$. Let $\I_{l},l=0,\ldots,d-1$ provide the index of the hidden unit through which a path $p$ passes in layer $l$ (with the convention that $\I_d(p)=1,\forall p\in [P]$). We then define:\\
$1.$ The value of a path $p$ by $v_t(p)\stackrel{def}=\Pi_{l=1}^d \Theta_t(l,\I_{l-1}(p),\I_l(p))$.\\
$2.$ The activity of a path $p$ for an input $x\in \R^{d_{in}}$ by $A_{t}(x,p)\stackrel{def}{=}\Pi_{l=1}^{d-1} G_{x,t}(l,\I_l(p))$.\\
\textbf{Value Derivative:} For a path $p$, the derivative of its value with respect to any weight in the path is: 
\begin{align}\label{eq:vft}{\partial v_t(p)}/{\partial \Theta\left(l,\I_{l'-1}(p),\I_{l'}(p)\right)}|_{\Theta=\Theta_t}= \underset{l=1}{\underset{l\neq l'}{\overset{d}{\Pi}}} \Theta_t\left(l,\I_{l-1}(p),\I_{l}(p)\right)
\end{align}
If a path $p$ does not pass through a $\theta\in\Theta$, then ${\partial v_t(p)}/{\partial \theta}=0$.\\
\textbf{Activity Derivative:} For a path $p$, and an input $x\inrdin$ the derivative of its activity with respect to any weight is given by 
\begin{align}
\frac{\partial A_{t}(x,p)}{\partial \tg}= \sum_{l=1}^{d-1} \Big(\frac{\partial G_{x,\Tg_t}(l,\I_l(p))}{\partial \tg} \Big)\Big(\Pi_{l'\neq l} G_{x,\Tg_t}(l',I_{l'}(p))\Big)
\end{align}
\begin{assumption}\label{assmp:init}
The weights $\Theta_0$ and $\Tg_0$ are sampled i.i.d from a distribution such that for any $\theta_0\in\{\Theta_0\cup \Tg_0\}$,  we have $\E{\theta_0}=0$, and  $\E{\theta^2_0}=\sigma^2$, and $\E{\theta^4_0}={\sigma'}^2$.
\end{assumption}
\begin{lemma}\label{lm:disentangle}[Disentanglement]
Under \Cref{assmp:init}, $\forall\,\theta\in\Theta$, for paths $p,p'\in [P], p\neq p'$:  i) $E{\partial_{\theta}v_0(p)\partial_{\theta}v_0(p')}= 0$ and ii)$\E{\partial_{\theta}v^2_0(p)}= \sigma^{2(d-1)}$, iii) $\E{v_0(p)v_0(p')}=0$, iv) $\E{v^2_0(p)}=\sigma^{2d}$.
\end{lemma}
%\subsection{Zeroth-order feature and kernel}
We now introduce the \emph{neural path feature} (NPF) and the \emph{neural path kernel} (NPK), which are zeroth-order quantities defined using the gating information $\G_t$.\\
The NPF of an input $x\in \R^{d_{in}}$ is given by $\phi_{x,t}=(x(\I_0(p))A_t(x,p) ,p\in[P])\in\R^P$. By arranging the NPF of the $n$ input examples in a matrix $\Phi_t=\left[\phi_{x_1,t},\ldots, \Phi_{x_n,t}\right]$, we can express the predicted output of a DGN as: \begin{align}\label{eq:npfbasic}\hat{y}_t=\Phi_t^\top v_t,\end{align}
where, the value of the path $v_t$ is the equivalent of the so called \emph{weight-vector} in a standard linear approximation. 
\begin{comment}
The significance of the NPF are:\hfill\\
$1.$ \textbf{Signal-Wire Separation}: Note that $\Phi$ encodes the signal: say for a DNN with ReLU activations, the co-ordinate corresponding to path $p$ is either $x(\I_0(p))$ if the path is active for that input (i.e., $A_t(x,p)=1$) or $0$ if the path is inactive for that input  (i.e., $A_t(x,p)=0$). The value of the path thus encodes the \emph{wire}, i.e., the information contained in the weights of the network. \hfill\\
$2.$ \textbf{Deep Information Propagation:} The path view provides a novel way of looking at information propagation in DNNs, eschewing the conventional `layer-by-layer' expression for information flow.\hfill\\
$3.$ \textbf{Representation:} As a parallel to the standard linear approximation, the NPF can be regarded as the \emph{hidden feature} and $v_t$ as the weight vector.
\end{comment}
\begin{definition}\label{def:lambda}
For inputs $x,x'\in\R^{d_{in}}$, using any $i\in [d_{in}]$,  define $\lambda_t(x,x')\stackrel{def}{=}\sum_{p\rsa i} A_t(x,p) A_t(x',p)$.
 \end{definition} 
\begin{lemma}\label{lm:npk}[Neural Path Kernel] 
Let $x=(x_s,s\in [n])\in\R^{d_{in}\times n}$ be the data matrix and let the neural path kernel matrix be defined as $H_t\stackrel{def}=\Phi^\top_t\Phi_t$. It follows that $H_t= (x^\top x)\odot(\lambda_t)$. 
\end{lemma}
\begin{comment}
$\lambda_t(s,s')$ is to be understood as the measure of overlap of the sub-networks that are active for both the inputs $x,x'\in\R^{d_{in}}$. Note that the definition of $\lambda_t$ is independent of $i\in [d_{in}]$: owing to symmetry the same number of paths start from any given input node, and looking forward, the paths see exactly the same gates and gating values in the subsequent layers.
\end{comment}
%\subsection{First-order feature and kernel}
\begin{comment}
Given that the output of a DGN is expressed as $\hat{y}_t(x)=\phi_{x,t}^\top v_t$, we define the following:\\
$1.$ \textbf{Value gradient:} $\psi_{x,t}^v=(\nabla_{\Theta}v_t)^\top  \phi_{x,t}$, which learns the value of the paths keeping the NPF fixed. \\
$2.$ \textbf{Feature gradient:}  $\psi_{x,t}^{\phi}=(\nabla_{\Tg}\phi_{x,t})^\top v_t $, which learns the NPFs keeping the value of the paths fixed. Note that the feature gradient is $0$ in the case of hard-gates.
To the best of our knowledge, for the first time in the literature, we note that the gradient flow has two components. Note that, in the value and feature gradients, quantities $\nabla_{\Theta}v_t$ and $\nabla_{\Tg}\phi_{x,t}$ are $P\times d_{net}$ matrices, which are obtained by computing the derivative of $v_t(p)$ and $\phi_{x,t}(p)$ for all paths $p\in[P]$ with respect to the respective $d_{net}$ parameters.\\
\end{comment}
\textbf{Neural Tangent Feature (NTF):} In the case of soft-GaLU networks, i.e., when two distinct set of parameters $\Theta\inrdnet, \Tg\inrdnet$ are used, we have the NTF to be $\psi_{x,t}=[\psi^v_{x,t},\psi^{\phi}_{x,t}]\in \R^{2d_{net}}$. In the case of implicit parameterisation, i.e., when $\Tg=\Theta$, we have the NTF to be $\psi_{x,t}=\psi^v_{x,t}+\psi^{\phi}_{x,t}$, and hence $\psi_{x,t}\inrdnet$.\\
\textbf{Neural Tangent Kernel (NTK):} In the case of soft-ReLU networks, NTK is given by $K_t(x,x')=K^v_t(x,x')+K^{\phi}_t(x,x')$, where $K^v_t(x,x')={\psi^v}^\top_{x,t}\psi^v_{x',t}$ and $K^{\phi}_t(x,x')={\psi^{\phi}}^\top_{x,t}\psi^{\phi}_{x',t}$. In the case of implicit parameterisation, the NTK is given by $K_t(x,x')=K^v_t(x,x')+K^{\phi}_t(x,x')+{\psi^v}^\top_{x,t} \psi^{\phi}_{x',t} + {\psi^{\phi}}^\top_{x,t} \psi^{v}_{x',t}$.
\begin{assumption}\label{assmp:decouple}
$\G_0$ is statistically independent of $\Theta_0\inrdnet$ or in the case of parameterised gating, $\Tg_0\inrdnet$ is statistically independent of $\Theta_t\inrdnet$.
\end{assumption}
\begin{definition}\label{def:delta}
$\delta_t(x,x')\stackrel{def}= \underset{{p\rsa i}}{\sum} \sum_{\tg\in\Tg}\frac{\partial A_{t}(x,p)}{\partial \tg} \frac{\partial A_{t}(x',p)}{\partial \tg}$, for $x,x'\in\R^{d_{in}}$, using any $i\in[d_{in}]$.
\end{definition}
Note that in \Cref{def:delta}, $\delta_t$ contains $\partial A_t(x,p)$ terms as opposed to $A_t(x,p)$ term in $\lambda_t$ defined in \Cref{def:lambda}.
\begin{theorem}\label{th:main} Let $x=(x_s,s\in[n])\in \R^{d_{in}\times n}$ be the data matrix. Then, under \Cref{assmp:init,assmp:decouple}, we have:\\
(i) $\E{K_0}=\E{K^v_0}+\E{K^{\phi}_0}$, where $\E{K^{v}_0}=\sigma^{2(d-1)} (x^\top x)\odot \lambda_0$, and $\E{K^{\phi}_0}=\sigma^{2d}  (x^\top x)\odot \delta_0$.\\
(ii) In addition, if ${4d}/{w^2}<1$, then $Var\left[K^v_0\right]\leq O\left(d^2_{in}\sigma^{4(d-1)}\max\{d^2w^{2(d-2)+1}, d^3w^{2(d-2)}\}\right)$,\\
where $K_0, K^v_0, K^{\phi}_0, \lambda_0, \delta_0$ are matrices defined on the dataset.
\end{theorem}

