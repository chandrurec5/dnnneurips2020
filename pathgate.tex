\section{Notation: Deep Gated Network and Paths}\label{sec:pathgate}

\begin{table}[!htb]
\begin{minipage}{0.5\columnwidth}
\resizebox{1.0\columnwidth}{!}{
\begin{tabular}{|l|}\hline
\multicolumn{1}{|c|}{Path Quantities}\\\hline 
$\I_0\colon [P]\ra [d_{in}]$, $\I_{l}\colon [P]\ra [w], l\in[d-1]$\\\hline
$v_t(p)\stackrel{def}=\Pi_{l=1}^d \Theta_t(l,\I_{l-1}(p),\I_l(p))$ \\\hline
$A_{t}(x,p)\stackrel{def}{=}\Pi_{l=1}^{d-1} G_{x,t}(l,\I_l(p))$\\\hline
\end{tabular}
}
\end{minipage}
\begin{minipage}{0.5\columnwidth}
\resizebox{\columnwidth}{!}{
\begin{tabular}{|l|l|}\hline
\multicolumn{2}{|c|}{Deep Gated Network}\\\hline 
Input layer & $z_{x,\Theta_t}(0)=x$ \\\hline
Pre-activation & $q_{x,\Theta_t}(l)={\Theta_t(l)}^\top z_{x,\Theta_t}(l-1)$\\\hline
Layer output & $z_{x,\Theta_t}(l)=q_{x,\Theta_t}(l)\odot G_{x,t}(l)$ \\\hline
Final output & $\hat{y}_t(x)={\Theta_t(d)}^\top z_{x,\Theta_t}(d-1)$\\\hline
\end{tabular}
}
\end{minipage}
\caption{Here $\Theta(l,i,j)$ denotes the weight connecting node $i$ of layer $l-1$ to node $j$ of layer $l$, and $\odot$ stands for the \emph{Hadamard} product.}
\label{tb:dgn-path}
\end{table}
We consider a dataset given by $(x_s,y_s)_{s=1}^n\in \R^{d_{in}}\times \R$, and fully connected deep neural network with $d$ layers (i.e., depth $=d$), and $w$ hidden units per layer (i.e., width $=w$). The total number of paths is given by $P=d_{in}w^{(d-1)}$, and an enumeration of the paths is given by $[P]=\{1,\ldots,P\}$.  $\I_{l},l=0,\ldots,d-1$, provide the index of the hidden unit through which a path $p$ passes in layer $l$ (with the convention that $\I_d(p)=1,\forall p\in [P]$). The value of a path $p$ is denoted by $v_t$ and its activation for input $x\in \R^{d_{in}}$ is given by $A_t(x,p)$. \hfill\\
In the deep gated network (DGN) in \Cref{tb:dgn-path}, $q(l),z(l)$ and $G(l)$ are $w$-dimensional quantities. In a DGN, the gating values $G_{x,t}(l)$ are handled as independent variables. Thus at time $t$, by specifying the gating values for all the $n$ input examples as $\G_t\stackrel{def}=\{G_{x_{s},t}(l,i), \forall s\in[n],l\in[d-1],i\in[w]\}$, we can recover the outputs $\hat{y}_t(x_s)\in \R$ for all the inputs $\{x_s\}_{s=1}^n$ in the dataset. In the case of DNNs with ReLU units, the gating values are implicit, i.e., $G_{x,t}(l)=\mathbbm{1}_{\{q_{x,\Theta_t}(l)>0\}}$. At time $t$, we denote a DGN parameterised by $\Theta_t\in \R^{d_{net}}$ and gating values $\G_t$ by $\N_G(\Theta_t;\G_t)$, and a DNN with ReLU activations parameterised by $\Theta_t\in\R^{d_{net}}$ by $\N_R(\Theta_t)$. By $\G(\N)$ we denote the fact that gating variables  $\G_t$ are derived from network $\N$.\hfill\\
\textbf{Note on ConvNets:} For input $x\in\R^{d_in}$, consider an architecture with $d_{out}$ channels, and each of which comprises of $d_{conv}$ convolutional layers. Let the convolutional layers perform circular $1$-dimensional convolution with kernel size $d_{ker}<d_{in}$ and unit stride. In the corners, instead of \emph{zero-padding}, in circular convolutional operations, we follow the convention that $\forall l=0,\ldots, d_{conv}-1, k\in[d_{in}], z^{(c)}_{x,t}(l,d_{in}+k)=z^{(c)}_{x,t}(l,k)$, and $z^{(c)}_{x,t}(l,-k)=z^{(c)}_{x,t}(l,d_{in}-k)$, where $z^{(c)}_{x,t}(l)$ is the $d_{in}$ dimensional output of the $l^{th}$ convolutional layer of the channel $c\in[d_{out}]$. The convolutional layers are followed by a \emph{global average-/max-pooling} layer, and arranging all the outputs after pooling we obtain $z_{x,t}(d_{conv}+1)$ to be the $d_{out}$-dimensional output, which is then followed by fully connected layers. Owing to the inherent circular (cylindrical) symmetry at all the convolutional layers, the $d_{out}$ dimensional feature produced after the pooling layers is invariant to circular shifts in the input.
