\section{A Note on ConvNets:} 
For input $x\in\R^{d_in}$, consider an architecture with $d_{out}$ channels, and each of which comprises of $d_{conv}$ convolutional layers. Let the convolutional layers perform circular $1$-dimensional convolution with kernel size $d_{ker}<d_{in}$ and unit stride. In the corners, instead of \emph{zero-padding}, in circular convolutional operations, we follow the convention that $\forall l=0,\ldots, d_{conv}-1, k\in[d_{in}], z^{(c)}_{x,t}(l,d_{in}+k)=z^{(c)}_{x,t}(l,k)$, and $z^{(c)}_{x,t}(l,-k)=z^{(c)}_{x,t}(l,d_{in}-k)$, where $z^{(c)}_{x,t}(l)$ is the $d_{in}$ dimensional output of the $l^{th}$ convolutional layer of the channel $c\in[d_{out}]$. The convolutional layers are followed by a \emph{global average-/max-pooling} layer, and arranging all the outputs after pooling we obtain $z_{x,t}(d_{conv}+1)$ to be the $d_{out}$-dimensional output, which is then followed by fully connected layers. Owing to the inherent circular (cylindrical) symmetry at all the convolutional layers, the $d_{out}$ dimensional feature produced after the pooling layers is invariant to circular shifts in the input.

\section{Proofs}
\textbf{Statement of \Cref{lm:npk}}
 Let $x=(x_s,s\in [n])\in\R^{d_{in}\times n}$ be the data matrix and let the neural path kernel matrix be defined as $H_t\stackrel{def}=\Phi^\top_t\Phi_t$. It follows that $H_t= (x^\top x)\odot(\lambda_t)$. 
\begin{proof}
\begin{align*}
\phi^\top_{x_s,t}\phi_{x_s',t}&=\sum_{p\in[P]}x(\I_0(p),s)A_t(x_s,p) x(\I_0(p),s')A_t(x_{s'},p)\\
&=\sum_{i=1}^{d_{in}}\sum_{p\rsa i}x(i,s)A_t(x_s,p) x(i,s')A_t(x_{s'},p)\\
&=\sum_{i=1}^{d_{in}}x_s(i)x_{s'}(i)\sum_{p\rsa i} A_t(x_s,p)A_t(x_{s'},p)\\
&=(x^\top_s x_{s'})\lambda_t(s,s')
\end{align*}
\end{proof}

\textbf{Statement of \Cref{lm:disentangle}}
Under \Cref{assmp:main}, for paths $p,p'\in \P, p\neq p'$, we have  i) $\E{\ip{\varphi_{p,0}, \varphi_{p',0}}}= 0$ and ii)${\ip{\varphi_{p,0}, \varphi_{p,0}}}= d\sigma^{2(d-1)}$.
\begin{proof}
\begin{align*}
\ip{\varphi_{p,t}, \varphi_{p',t}}= \sum_{m=1}^{d_{net}} \varphi_{p,t}(m)\varphi_{p',t}(m)
\end{align*}
Let the $d_{net}$ weights be enumerable as $\theta(1),\ldots,\theta(d_{net})$, and let $\theta(m),m\in[d_{net}]$ be any weight such that $p\rsa \theta(m)$, and w.l.o.g let $\theta(m)$ belong to layer $l'(m)\in[d]$. 
If either $p\bcancel{\rsa}\theta(m)$ or $p'\bcancel{\rsa}\theta(m)$, then it follows that $\varphi_{p,t}(m)\varphi_{p',t}(m)=0$. In the case when $p,p'\rsa\theta(m)$, we have
\begin{align*}
&\E{\varphi_{0,p}(m)\varphi_{0,p'}(m)}\\
&=\E{\underset{l\neq l'(m)}{\underset{l=1}{\overset{d}{\Pi}}} \Bigg(\Tb_0(l,\I_{l-1}(p),\I_l(p))\Tb_0(l,\I_{l-1}(p'),\I_l(p')) \Bigg)}\\
&=\underset{l\neq l'(m)}{\underset{l=1}{\overset{d}{\Pi}}}\E{\Tb_0(l,\I_{l-1}(p),\I_l(p))\Tb_0(l,\I_{l-1}(p'),\I_l(p'))}
\end{align*}
where the $\E{\cdot}$ moved inside the product because at initialisation the weights (of different layers) are independent of each other.
Since $p\neq p'$, in one of the layers $\tilde{l}\in[d-1],\tilde{l}\neq l'(m)$ they do not pass through the same weight, i.e., $\Tb_0(\tilde{l},\I_{\tilde{l}-1}(p),\I_{\tilde{l}}(p))$ and $\Tb_0(\tilde{l},\I_{\tilde{l}-1}(p'),\I_{\tilde{l})}(p')$ are distinct weights. Using this fact
\begin{align*}
&\E{\varphi_{0,p}(m)\varphi_{0,p'}(m)}\\
&=\underset{l\neq l',\tilde{l}}{\underset{l=1}{\overset{d}{\Pi}}}\E{\Tb_0(l,\I_{l-1}(p),\I_l(p))\Tb_0(l,\I_{l-1}(p'),\I_l(p'))}\E{\Tb_0(\tilde{l},\I_{\tilde{l}-1}(p),\I_{\tilde{l}}(p))}\E{\Tb_0(\tilde{l},\I_{\tilde{l}-1}(p'),\I_{\tilde{l}}(p')}\\\\
&=0
\end{align*}

The proof of (ii) is complete by noting that $\sum_{m=1}^{d_{net}} \varphi_{p,t}(m)\varphi_{p,t}(m)$ has $d$ non-zero terms for a single path $p$ and at initialisation we have 
\begin{align*}
&{\varphi_{0,p}(m)\varphi_{0,p}(m)}\\
&={\underset{l\neq l'}{\underset{l=1}{\overset{d}{\Pi}}} \Tb^2_0(l,\I_{l-1}(p),\I_l(p))}\\
&=\sigma^{2(d-1)}
\end{align*}
\end{proof}

\textbf{Statement of \Cref{th:exp}}
Under \Cref{assmp:main}, we have $\E{K_0}=d\sigma^{2(d-1)}(x^\top x) \odot (\lambda_0)$.

\begin{proof} Let $\varphi_t=\left[\varphi_{p,t},p\in[P]\right]\in\R^{d_{net}\times P}$ be the VTF matrix at time $t$, then if follows that the NTK (Gram) matrix is  $K_t=\Psi^\top_t\Psi_t=\Phi^\top_t\varphi_t\varphi^\top_t\Phi_t$. Now, taking expectation we have $\E{K_0}=\E{\Phi^\top_t\varphi_t\varphi^\top_t\Phi_t}$, and using \Cref{assmp:main}-(i), one can pull out the $\Phi_t$ terms outside of the expectation, i,e., $\E{K_0}=\Phi^\top_t\E{\varphi_t\varphi^\top_t}\Phi_t$, and using \Cref{assmp:main}-(ii), we can show that $\E{\varphi_t\varphi^\top_t}=d\sigma^{2(d-1)}I$. The statement of \Cref{th:exp} follows by using \Cref{lm:npk}.
\end{proof}


\textbf{Statement of \Cref{th:var}}
Under \Cref{assmp:main} and the condition that ${4d}/{w^2}<1$, it follows that\hfill\\
$Var\left[K_0\right]\leq O\left(d^2_{in}\sigma^{4(d-1)}\max\{d^2w^{2(d-2)+1}, d^3w^{2(d-2)}\}\right)$.
\begin{proof}
The idea is that we expand  $Var\left[K_0(s,s')\right]=\E{K_0(s,s')^2} -\E{K_0(s,s')}^2$ and identify the the terms which cancel due to subtraction and then bound the rest of the terms.% Further, in what follows, we will assume $d_{in}=1$ without loss of generality.
 Let $\theta(m)$ belong to layer $l'(m)$, then 
\begin{align}\label{eq:kexpect}
&\E{K_0(s,s')}\nn\\
&=\sum_{m=1}^{d_{net}}\E{\left(\sum_{p_1 \in[P]}x(\I_0(p_1),s)A_0(s,p_1)\frac{\partial v_0(p_1)}{\partial \theta(m)}\right)\left(\sum_{p_2\in[P]}x(\I_0(p_2),s)A_0(s',p_2)\frac{\partial v_0(p_2)}{\partial \theta(m)}\right)}\nn\\
&=\sum_{m=1}^{d_{net}}\E{\sum_{p_1,p_2\in[P]}x(\I_0(p_1),s)A_0(s,p_1)\frac{\partial v_0(p_1)}{\partial \theta(m)}x(\I_0(p_2),s')A_0(s',p_2)\frac{\partial v_0(p_2)}{\partial \theta(m)}}\nn\\
&\stackrel{(a)}=\sum_{m=1}^{d_{net}}\underset{p_1,p_2\rsa\theta(m)}{\sum_{p_1,p_2\in[P]}}x(\I_0(p_1),s)A_0(s,p_1)x(\I_0(p_2),s')A_0(s',p_2) \E{\underset{l\neq l'(m)}{\underset{l=1}{\overset{d-1}{\Pi}}} \Tb_0(l,\I_{l-1}(p_1),\I_{l}(p_1)) \Tb_0(l,\I_{l-1}(p_2),\I_{l}(p_2))}\nn\\
&\stackrel{(b)}=\sum_{m=1}^{d_{net}}\underset{p_1,p_2\rsa\theta(m)}{\sum_{p_1,p_2\in[P]}}x(\I_0(p_1),s)A_0(s,p_1)x(\I_0(p_2),s')A_0(s',p_2) \underset{l\neq l'(m)}{\underset{l=1}{\overset{d-1}{\Pi}}} \E{\Tb_0(l,\I_{l-1}(p_1),\I_{l}(p_1)) \Tb_0(l,\I_{l-1}(p_2),\I_{l}(p_2))}
\end{align}
where $(a)$ follows from the fact that for $p\bcancel{\rsa}\theta(m)$, $\frac{\partial v_0(p)}{\partial \theta(m)}=0$, and $(b)$ follows from the fact that at initialisation the layer weights are independent of each other. Note that the right hand side of \eqref{eq:kexpect} only terms with $p_1=p_2$ will survive the expectation.

In the following expression in \eqref{eq:kexpectsquare}, note that only terms of the form $p_1=p_2$ and $p_3=p_4$ are non-zero.
\begin{align}\label{eq:kexpectsquare}
&\E{K_0(s,s')}^2=\nn\\
&\left(\sum_{m=1}^{d_{net}}\underset{p_1,p_2\rsa\theta(m)}{\sum_{p_1,p_2\in[P]}}x(\I_0(p_1),s)A_0(s,p_1)x(\I_0(p_2),s')A_0(s',p_2) \underset{l\neq l'(m)}{\underset{l=1}{\overset{d-1}{\Pi}}} \E{\Tb_0(l,\I_{l-1}(p_1),\I_{l}(p_1)) \Tb_0(l,\I_{l-1}(p_2),\I_{l}(p_2))}\right)\nn\\
&\left(\sum_{m'=1}^{d_{net}}\underset{p_3,p_4\rsa\theta(m')}{\sum_{p_3,p_4\in[P]}}x(\I_0(p_3),s)A_0(s,p_3)x(\I_0(p_4),s')A_0(s',p_4) \underset{l\neq l'(m')}{\underset{l=1}{\overset{d-1}{\Pi}}} \E{\Tb_0(l,\I_{l-1}(p_3),\I_{l}(p_3)) \Tb_0(l,\I_{l-1}(p_4),\I_{l}(p_4))}\right)\nn\\
&=\nn\\
&\sum_{m,m'=1}^{d_{net}}\underset{p_3,p_4\rsa\theta(m')}{\underset{p_1,p_2\rsa\theta(m)}{\sum_{p_1,p_2,p_3,p_4\in[P]}}}\Bigg[\bigg(x(\I_0(p_1),s)A_0(s,p_1)x(\I_0(p_2),s')A_0(s',p_2)x(\I_0(p_3),s)A_0(s,p_3)x(\I_0(p_4),s')A_0(s',p_4)\bigg)\nn\\
&\bigg( \underset{l\neq l'(m)} {\underset{l\neq l'(m')}{\underset{l=1}{\overset{d-1}{\Pi}}}} \E{\Tb_0(l,\I_{l-1}(p_1),\I_{l}(p_1)) \Tb_0(l,\I_{l-1}(p_2),\I_{l}(p_2))}\E{\Tb_0(l,\I_{l-1}(p_3),\I_{l}(p_3)) \Tb_0(l,\I_{l-1}(p_4),\I_{l}(p_4))} \bigg)\nn\\
&\bigg( \E{\Tb_0(l,\I_{l'(m')-1}(p_1),\I_{l'(m')}(p_1)) \Tb_0(l,\I_{l'(m')-1}(p_2),\I_{l'(m')}(p_2))}\bigg)\nn\\
&\bigg(\E{\Tb_0(l,\I_{l'(m)-1}(p_3),\I_{l'(m)}(p_3)) \Tb_0(l,\I_{l'(m)-1}(p_4),\I_{l'(m)}(p_4))} \bigg)\Bigg]\nn\\
\end{align}

In the expression in \eqref{eq:ksquareexpect}, paths $p_1,p_2,p_3,p_4$ do not have constraints, and can be distinct.
\begin{align}\label{eq:ksquareexpect}
&\E{K^2_0(s,s')}=\nn\\
&\sum_{m,m'=1}^{d_{net}}\underset{p_3,p_4\rsa\theta(m')}{\underset{p_1,p_2\rsa\theta(m)}{\sum_{p_1,p_2,p_3,p_4\in[P]}}}\Bigg[\bigg(x(\I_0(p_1),s)A_0(s,p_1)x(\I_0(p_2),s')A_0(s',p_2)x(\I_0(p_3),s)A_0(s,p_3)x(\I_0(p_4),s')A_0(s',p_4)\bigg)\nn\\
&\bigg( \underset{l\neq l'(m)} {\underset{l\neq l'(m')}{\underset{l=1}{\overset{d-1}{\Pi}}}} \E{\Tb_0(l,\I_{l-1}(p_1),\I_{l}(p_1)) \Tb_0(l,\I_{l-1}(p_2),\I_{l}(p_2))\Tb_0(l,\I_{l-1}(p_3),\I_{l}(p_3)) \Tb_0(l,\I_{l-1}(p_4),\I_{l}(p_4))} \bigg)\nn\\
&\bigg( \E{\Tb_0(l,\I_{l'(m')-1}(p_1),\I_{l'(m')}(p_1)) \Tb_0(l,\I_{l'(m')-1}(p_2),\I_{l'(m')}(p_2))}\bigg)\nn\\
&\bigg(\E{\Tb_0(l,\I_{l'(m)-1}(p_3),\I_{l'(m)}(p_3)) \Tb_0(l,\I_{l'(m)-1}(p_4),\I_{l'(m)}(p_4))} \bigg)\Bigg]\nn\\
\end{align}

We now state the following facts/observations.

$\bullet$ \textbf{Fact 1:} Any term that survives the expectation (i.e., does not become $0$) and participates in \eqref{eq:ksquareexpect} is of the form $\sigma^{4(d-1)}\big(x(\I_0(p_1),s)A_0(s,p_1)x(\I_0(p_2),s')A_0(s',p_2)x(\I_0(p_3),s)A_0(s,p_3)x(\I_0(p_4),s')A_0(s',p_4)\big)$, where $p_1,p_2,p_3,p_4$ are free variables. Any term that survives the expectation (i.e., does not become $0$) and participates in participates in \eqref{eq:kexpectsquare} is of the form $\sigma^{4(d-1)}\big(x(\I_0(p_1),s)A_0(s,p_1)x(\I_0(p_2),s')A_0(s',p_2)x(\I_0(p_3),s)A_0(s,p_3)x(\I_0(p_4),s')A_0(s',p_4)\big)$, where $p_1=p_2,p_3=p_4$.

$\bullet$ \textbf{Fact 2:} The number of paths through a particular weight $\theta(m)$ in one of the middle layers is $d_{in}w^{d-3}$, and the number of paths through a particular weight $\theta(m)$ in either the first or the last layer is $d_{in}w^{d-2}$ .

$\bullet$ \textbf{Fact 3:} Let $\P'$ be an arbitrary set of paths constrained to pass through some set of weights. Let $\P''$ be the set of paths obtained by adding an additional constraint that the paths also should pass through a particular weight say $\theta(m)$. Now, if $\theta(m)$ belongs to :

$1.$ a middle layer, then $|\P''|=\frac{|\P'|}{w^2}$.

$2.$ the first layer or the last layer, then $|\P''|=\frac{|\P'|}{w}$.

$\bullet$ \textbf{Fact 4:} For any $p_1,p_2,p_3,p_4$ combination that survives the expectation in \eqref{eq:ksquareexpect} can be written as 

\begin{align*}
&\bigg( \underset{l\neq l'(m)} {\underset{l\neq l'(m')}{\underset{l=1}{\overset{d-1}{\Pi}}}} \E{\Tb_0(l,\I_{l-1}(p_1),\I_{l}(p_1)) \Tb_0(l,\I_{l-1}(p_2),\I_{l}(p_2))\Tb_0(l,\I_{l-1}(p_3),\I_{l}(p_3)) \Tb_0(l,\I_{l-1}(p_4),\I_{l}(p_4))} \bigg)\nn\\
&\bigg( \E{\Tb_0(l,\I_{l'(m')-1}(p_1),\I_{l'(m')}(p_1)) \Tb_0(l,\I_{l'(m')-1}(p_2),\I_{l'(m')}(p_2))}\bigg)\nn\\
&\bigg(\E{\Tb_0(l,\I_{l'(m)-1}(p_3),\I_{l'(m)}(p_3)) \Tb_0(l,\I_{l'(m)-1}(p_4),\I_{l'(m)}(p_4))} \bigg)\Bigg]\nn\\
&=\\
&\bigg( \underset{l\neq l'(m)} {\underset{l\neq l'(m')}{\underset{l=1}{\overset{d-1}{\Pi}}}} \Tb^2_0(l,\I_{l-1}(\rho_a),\I_{l}(\rho_a)) \Tb^2_0(l,\I_{l-1}(\rho_b),\I_{l}(\rho_b)) \bigg)\nn\\
&\bigg( \Tb^2_0(l,\I_{l'(m')-1}(\rho_a),\I_{l'(m')}(\rho_a))\bigg)\nn\\
&\bigg({\Tb^2_0(l,\I_{l'(m)-1}(\rho_b),\I_{l'(m)}(\rho_b))} \bigg),
\end{align*}

where $\rho_a\rsa \theta(m)$ and $\rho_b\rsa \theta(m')$ are what we call as \emph{base} (case) paths. 


$\bullet$ \textbf{Fact 5:} For any given base paths $\rho_a$ and $\rho_b$ there could be multiple assignments possible for $p_1,p_2,p_3,p_4$.

$\bullet$ \textbf{Fact 6:}  Terms in \eqref{eq:ksquareexpect}, wherein, the base case is generated as $p_1=p_2=\rho_a$ and $p_3=p_4=\rho_b$ (or $p_1=p_2=\rho_b$ and $p_3=p_4=\rho_a$), get cancelled with the corresponding terms in \eqref{eq:kexpectsquare}.

$\bullet$ \textbf{Fact 7:}  When the bases paths $\rho_a$ and $\rho_b$ do not intersect (i.e., do not pass through the same weight in any one of the layers), the only possible assignment is $p_1=p_2=\rho_a$ and $p_3=p_4=\rho_b$ (or $p_1=p_2=\rho_b$ and $p_3=p_4=\rho_a$), and such terms are common in \eqref{eq:ksquareexpect} and \eqref{eq:kexpectsquare}, and hence do not show up in the variance term.


$\bullet$ \textbf{Fact 7:} Let base paths $\rho_a$ and $\rho_b$ intersect/cross at layer $l_1, \ldots, l_k, k \in [d-1]$, and let $\rho_a=(\rho_a(1),\ldots,\rho_a(k+1))$ where $\rho_a(1)$ is a sub-path string from layer $1$ to $l_1$, and $\rho_a(2)$ is the sub-path string from layer $l_1+1$ to $l_2$ and so on, and $\rho_a(k+1)$ is the sub-path string from layer $l_k+1$ to the output node. Then the set of paths that can occur in $\E{K_0(s,s')^2}$ are of the form:
\begin{enumerate}
\item $p_1=p_2=\rho_a, p_3=p_4=\rho_b$ (or $p_1=p_2=\rho_b, p_3=p_4=\rho_a$) which get cancelled in the $\E{K_0(s,s')}^2$ term.
\item $p_1=\rho_a$, $p_3=\rho_b$, $p_2=(\rho_b(1),\rho_a(2),\rho_a(3),\ldots,\rho_a(k+1))$, $p_4=(\rho_a(1),\rho_b(2),\rho_b(3),\ldots,\rho_b(k+1))$, which are obtained by splicing the base paths in various combinations. Note that for such spliced paths $p_1\neq p_2$ and $p_3\neq p_4$ and hence do not occur in the expression for $\E{K_0(s,s')}^2$ in \eqref{eq:kexpectsquare}.
\end{enumerate}


$\bullet$ \textbf{Fact 8:} For $k$ crossings of the base paths there are $4^{k+1}$ splicings possible, and those many terms are extra in the $\E{K_0(s,s')^2}$ expression in \eqref{eq:ksquareexpect}, when compared to the $\E{K_0(s,s')}^2$ expression. We now enumerate cases of possible crossings, and reason out the magnitude of their contribution to the variance term using the \textbf{Fact 1} to \textbf{Fact 8}.


\textbf{Case $1$} $k=1$ crossing in either first or last layer. There are $2w$ weights in the first and the last layer, and the number of base path combinations is $w^{d-2}\times w^{d-2}$, and for each of these cases, $m,m'$ could take $O(d^2)$ possible values. And the multiplication of the weights themselves contribute to $\sigma^{4(d-1)}$. Putting them together we have
\begin{align*}
d^2_{in}\sigma^{4(d-1)}\times (2w)\times d^2\times (w^{d-2}\times w^{d-2})\times 4^2 = 32d^2_{in}\sigma^{4(d-1)}d^2 w^{2(d-2)+1}
\end{align*}

\textbf{Case $2$} $k=1$ crossing in one of the middle layers. There are $w^2(d-2)$ weights in the first and the last layer, and the number of base path combinations is $w^{d-3}\times w^{d-3}$, and for each of these cases, $m,m'$ could take $O(d^2)$ possible values. And the multiplication of the weights themselves contribute to $\sigma^{4(d-1)}$. Putting them together we have
\begin{align*}
d^2_{in}\sigma^{4(d-1)}\times w^2(d-2)\times d^2\times (w^{d-3}\times w^{d-3})\times 4^2\leq 16d^2_{in}\sigma^{4(d-1)} d^3 w^{2(d-3)}
\end{align*}

\textbf{Case $3$} $k=2$ crossings one in the first layer and other in the last layer. This case can be covered using Case $1$ and then further restricting that the base paths should also in the other layer. So, we have
\begin{align*}
32d^2_{in}\sigma^{4(d-1)}d^2 w^{2(d-2)+1} \times \underbrace{w}_{\text{possible weights in other layer}} \times \underbrace{w^{-1}\times w^{-1}}_{\text{reduction in paths due to additional restriction}} \times 4 = (32d^2_{in}\sigma^{4(d-1)}d^2 w^{2(d-2)+1})\times (4w^{-1}),
\end{align*}
where the $4$ is for the $4$ extra possible ways of splicing the base paths.

\textbf{Case $4$} $k=2$ crossings first one in the first layer or the last layer, and the second one in the middle layer. This can be obtained by looking at the Case $1$ and then adding the further restriction that the base paths should cross each other in the middle layer. 
\begin{align*}
32d^2_{in}\sigma^{4(d-1)}d^2 w^{2(d-2)+1}\times w^2(d-2) \times (w^{-2}w^{-2}) \times 4= (32d^2_{in}\sigma^{4(d-1)}d^2 w^{2(d-2)+1} )\times (4dw^{-2}) 
\end{align*}

\textbf{Case $5$} $k=2$ crossings in the middle layer. This can be obtained by taking Case $2$ and then adding the further restriction that the base paths should cross each other in the middle layer. 
\begin{align*}
16d^2_{in}\sigma^{4(d-1)} d^3 w^{2(d-3)}\times w^2(d-2) w^{-2}w^{-2}\times 4\leq (16d^2_{in}\sigma^{4(d-1)} d^3 w^{2(d-3)}) \times (4dw^{-2})
\end{align*}


\textbf{Case $6$} $k=3$ crossings first one in the first layer or the last layer, and the other two in the middle layers. This can be obtained by considering Case $4$ and then adding the further restriction that the base paths should cross each other in the middle layer. 
\begin{align*}
(32d^2_{in}\sigma^{4(d-1)}d^2 w^{2(d-2)+1} )\times (4dw^{-2}) \times (4dw^{-2}) 
\end{align*}

\textbf{Case $7$} $k=3$ crossings first two in the first and last layers and the third one in the middle layers. This can be obtained by considering Case $3$ and then adding the further restriction that the base paths should cross each other in the middle layer. 

\begin{align*}
(32d^2_{in}\sigma^{4(d-1)}d^2 w^{2(d-2)+1})\times (4w^{-1})\times (4dw^{-2}) 
\end{align*}

\textbf{Case $8$} $k=3$ crossings in the middle layer. This can be obtained by considering Case $5$ and then adding the further restriction that the base paths should cross each other in the middle layer. 
\begin{align*}
 (16d^2_{in}\sigma^{4(d-1)} d^3 w^{2(d-3)}) \times (4dw^{-2})\times (4dw^{-2}) 
\end{align*}


The cases can be extended in a similar way, increasing the number of crossings.  Now, assuming $\frac{4d}{w^2}<1$, the bounds in the various terms can be lumped together as below:

$\bullet$ We can add the bounds for Case $1$, Case $4$, Case $6$ and other cases obtained by adding more crossings (one at a time) in the middle layer to Case $6$. This gives rise to a term which is upper bounded by 
\begin{align*}
d^2_{in}\sigma^{4(d-1)}d^2w^{2(d-2)+1}\left(\frac{1}{1-4dw^{-2}}\right)
\end{align*}

$\bullet$ We can add the bounds for Case $3$, Case $7$ and other cases obtained by adding more crossings (one at a time) in the middle layer to Case $6$. This gives rise to a term which is upper bounded by 
\begin{align*}
d^2_{in}\sigma^{4(d-1)}d^3w^{2(d-2)} \left(\frac{1}{1-4dw^{-2}}\right)
\end{align*}



$\bullet$ We can add the bounds for Case $2$, Case $5$, Case $8$ and other cases obtained by adding more crossings (one at a time) in the middle layer to Case $6$. This gives rise to a term which is upper bounded by 
\begin{align*}
d^2_{in}\sigma^{4(d-1)}d^2w^{2(d-2)} \left(\frac{1}{1-4dw^{-2}}\right)
\end{align*}

Putting together we have the variance to be bounded by 
\begin{align*}
Cd^2_{in}\sigma^{4(d-1)}\max\{d^2w^{2(d-2)+1}, d^3w^{2(d-2)}\},
\end{align*}
for some constant $C>0$.
\end{proof}
