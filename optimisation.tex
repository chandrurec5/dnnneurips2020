\section{Main Result}
In this section, we assume that the NPKs are given to us and fixed (i.e, $G_t=\G_0,\forall t\geq 0$ is given to us). We show that, at randomised weight initialisation, the NTK is equal to (but for a scaling factor) to the NPK. We then use the \emph{Hadamard} structure of the NPK to comment about optimisation and generalisation. 
\begin{assumption}\label{assmp:decouple}
(i) $\Theta_0\inrdnet$  is statistically independent of $\G_0$
\end{assumption}
\begin{theorem}\label{th:main} Under \Cref{assmp:init,assmp:decouple}, we have:\\
(i) $\E{K_0}=\sigma^{2(d-1)}H_0=\sigma^{2(d-1)}(x^\top x)\odot(\lambda_0)$,\\
(ii) In addition, if ${4d}/{w^2}<1$, then $Var\left[K^v_0\right]\leq O\left(d^2_{in}\sigma^{4(d-1)}\max\{d^2w^{2(d-2)+1}, d^3w^{2(d-2)}\}\right)$,\\
\end{theorem}
\begin{comment}
\textbf{Proof of \Cref{th:main}-(i):} Let $\varphi_t=(\varphi_{p,t},p\in[P])\in \R^{d_{net}\times P}$ matrix, then since $K_t=\Psi^\top_t\Psi$, where $\Psi_t=\varphi_t \Phi_t$, we have $\E{K_t}=\E{\Phi^\top_t \varphi^\top_t \varphi_t \Phi_t}$. At initialisation, using the \Cref{assmp:main}-(i), we can pull out $\Phi^\top_t$ and $\Phi_t$ outside of the expectation to have \begin{align}\label{eq:pullout}\E{K_0}=\Phi^\top_0\E{ \varphi^\top_t \varphi_t }\Phi_0,\end{align} and from \Cref{lm:disentangle}, it follows that $\E{ \varphi^\top_t \varphi_t }=d\sigma^{2(d-1)}I$, and hence $\E{K_0}=d\sigma^{2(d-1)}\Phi^\top_0\Phi_0=d\sigma^{2(d-1)}H_0$.\\
\end{comment}
\textbf{Discussion:}\\
$1.$ \emph{Active Sub-Network and Gradient Flow:}  Each input example has its own associated set of active sub-network, and while training a particular example, the gradient flows through the weights of the corresponding active sub-network. Now, the active sub-networks corresponding to different examples have some overlap, and hence there is bound to be \emph{cross-talk} of the gradients flowing through them. This overlap is captured by $\lambda_t(s,s')$ which is the measure of overlap of the sub-networks that are active for both the inputs $x,x'\in\R^{d_{in}}$. Under. \Cref{assmp:main}, the inter-path interaction $\varphi^\top_t\varphi_t$ gets disentangled, result in the claim \Cref{th:main}-(i).\\
$2.$ As seen from \Cref{th:main}, $\lambda_0$ directly controls the spectral properties of the NTK matrix $K_0$. Characterising $\lambda_0$ for the general case is left as future work. In what follows, we present an informal reasoning that increasing depth causes whitening of $\lambda_0$. However, in the later part of this section, we consider a special case, wherein, we give an explicit characterisation of the spectrum of $\E{K_0}$.\\
$3.$ \emph{Why increasing depth till a point helps in training? } From \Cref{th:main}-(ii) it follows that for $w\ra\infty$, $K_0\ra\E{K_0}$. We now argue that when $\sigma=\sqrt{\frac{2}{w}}$, increasing depth causes whitening of $\lambda_0$, and hence $K_0$ .\hfill\\
$\bullet$ Let us first look at the diagonal terms of $\lambda_0$. It is reasonable to assume that, owing to the symmetric nature of the weights, roughly $\mu=\frac{1}{2}$ fraction of the gates are \emph{on} every layer. Thus $\lambda_0(s,s)\approx (w/2)^{d-1}$. Now, due our choice of $\sigma=\sqrt{\frac{2}{w}}$, the diagonal entries will be close to $1$.\hfill\\
$\bullet$ We now turn our attention towards the non-diagonal entries of $\lambda_0$. Define $\tau(s,s',l)\stackrel{def}=\sum_{i=1}^w G_{x_s,t}(l,i)G_{x_{s'},t}(l,i)$ be the overlap of the active gates in layer $l$ for input examples $s,s'\in[n]$, and  let $\eta\stackrel{def}=\max_s\left(\max_{s',l} \frac{\tau(s,s',l)}{\tau(s,s,l)}\right)$ be the maximum overlap between gates of a layer (maximum taken over over input pairs $s,s'\in[n]$ and layers $l\in [d]$).  Then it follows that $\max_{s,s'\in [n]} \frac{\bar{\lambda}_{cross}(s,s')}{\bar{\lambda}_{self}(s)}\leq \eta^{d-1}$. Thus, the non-diagonal entries decay an exponential rate in comparison to the diagonal entries.\hfill\\
$4.$ \emph{Why increasing the depth beyond hurts training?} Note that for $\sigma=O\left(\sqrt{\frac{1}{w}}\right)$, for a fixed depth $d$, as width $w$ increases, $K_0\ra\E{K_0}$. However, the variance expression in \Cref{th:main}-$(ii)$ involves $d^2$ and $d^3$ terms, and hence for a fixed width as depth increases, the entries of $K_0$ deviates from $\E{K_0}$, and as a result the spectrum of $K_0$ degrades, thereby hurting training performance.\\
$5.$ \emph{Generalisation and Feature Learning:} As seen in \Cref{tb:npfs}, we know that, different NPFs give different generalisation performance. In this light, \Cref{th:main} complements the results by \cite{arora2019exact,cao2019generalization}, in that, one can plug-in the NPK in the place of NTK in their generalisation bounds. Further, the gap in the performance of the DNN and the NTK counterparts can be explained by the fact that the NPK keeps changing during training (see \Cref{fig:gen}). We will discuss learning of NPF in \Cref{sec:featlearn}.\\
$6.$ \Cref{assmp:main} is not satisfied by ReLU activations, i.e., conditioned on the fact that a ReLU is \emph{on}, the incoming weights cannot all be simultaneously negative. This implies that the $\Phi^\top_t$ and $\Phi_t$ terms cannot be pulled out of the expectation as in \eqref{eq:pullout}.\\
\begin{wrapfigure}{h}{0.27\textwidth}
\includegraphics[scale=0.22]{figs/dgn-fra-ecdf-ideal.png}
\caption{Ideal spectrum.}
\label{fig:ideal-spectrum}
\end{wrapfigure}
\begin{comment}
\begin{wrapfigure}{h}{0.27\textwidth}
\includegraphics[scale=0.22]{figs/dgn-fra-ecdf-ideal.png}
\includegraphics[scale=0.22]{figs/dgn-fra-ecdfbyd-w25.png}
\includegraphics[scale=0.22]{figs/dgn-fra-ecdfbyd-w500.png}
\includegraphics[scale=0.21]{figs/dgn-fra-conv-w25.png}
\includegraphics[scale=0.21]{figs/dgn-fra-conv-w500.png}
\caption{Ideal spectrum}
\label{fig:dgn-frg-gram-ecdf}
\end{wrapfigure}
\end{comment}
\textbf{Fixed Random Gating (Explicit Spectrum for $\E{K_0}$):} For each input example in the dataset, we sample gating values from $Ber(\mu)$ taking values in $\{0,1\}$, and collect it in $\G_0$. In this case, it is easy to check that $\mathbb{E}_{\mu}\left[\lambda_0(s,s)\right]=(\mu w)^{(d-1)},\forall s\in[n]$ and $\mathbb{E}_{\mu}\left[\lambda_0(s,s')\right]=(\mu^2 w)^{(d-1)},\forall s,s'\in[n]$. The
dataset is given by $(x_s,y_s)_{s=1}^n\in \R\times \R$, where $x_s=1,\forall s\in [n]$, and $y_s\sim unif([-1,1])$, $n=200$. The input Gram matrix $x^\top x$ is a $n\times n$ matrix with all entries equal to $1$ and its rank is equal to 1, and hence $H_0=\lambda_0$.\\
\textbf{Spectrum (Theory):} For $\sigma=\sqrt{\frac{1}{\mu w}}$, and by further averaging $\mathbb{E}_{\mu}\left[K_0(s,s)/d\right]=1$, and $\mathbb{E}_{\mu}\left[K_0(s,s')/d\right]=\mu^{(d-1)}$. Now, let $\rho_i\geq 0,i \in [n]$ be the eigenvalues of $\frac{\E{K_0}}{d}$, and let $\rho_{\max}$ and $\rho_{\min}$ be the largest and smallest eigenvalues. \WFclear One can easily show that $\rho_{\max}=1+(n-1)\mu^{d-1}$ and corresponds to the eigenvector with all entries as $1$, and $\rho_{\min}=(1-\mu^{d-1})$ repeats $(n-1)$ times, which corresponds to eigenvectors given by $[0, 0, \ldots, \underbrace{1, -1}_{\text{$i$ and $i+1$}}, 0,0,\ldots, 0]^\top \in \R^n$ for $i=1,\ldots,n-1$.\\
\textbf{Spectrum (numerical):} We look at the cumulative eigenvalue (e.c.d.f) obtained by first sorting the eigenvalues in ascending order then looking at their cumulative sum. The ideal behaviour (top plot of \Cref{fig:dgn-frg-gram-ecdf}) as predicted from theory is that for indices $k\in[n-1]$, the e.c.d.f should increase at a linear rate, i.e., the cumulative sum of the first $k$ indices is equal to $k(1-\mu^{d-1})$, and the difference between the last two indices is $1+(n-1)\mu^{d-1}$. In \Cref{fig:dgn-frg-gram-ecdf}, we plot the actual e.c.d.f for various depths $d=2,4,6,8,12,16,20$ and $w=25,500$ a (second and third from top in \Cref{fig:dgn-frg-gram-ecdf}). \hfill\\
\textbf{Convergence (numerical):} In order to compare how the rate of convergence varies with the depth, we set the step-size $\alpha=\frac{0.1}{\rho_{\max}}$, $w=100$. We use the vanilla SGD-optimiser. Note the$ \frac{1}{\rho_{\max}}$ in the stepsize, ensures that the uniformity of maximum eigenvalue across all the instances, and the convergence should be limited by the smaller eigenvalues. We also look at the convergence rate of the ratio $\frac{\norm{e_t}^2_2}{\norm{e_0}^2_2}$. We notice that for $w=25$, increasing depth till $d=8$ improves the convergence, however increasing beyond $d=8$ worsens the convergence rate. For $w=500$, increasing the depth till $d=12$ improves convergence, and $d=16,20$ are worse than $d=12$.  This matches with the depth phenomena observed in practical DNNs and also matches our theory.
\begin{figure}
\resizebox{\textwidth}{!}{
\begin{tabular}{cccc}
%\includegraphics[scale=0.4]{figs/dgn-fra-ecdf-ideal.png}
%&
\includegraphics[scale=0.5]{figs/dgn-fra-ecdfbyd-w25.png}
&
\includegraphics[scale=0.5]{figs/dgn-fra-conv-w25.png}
&
\includegraphics[scale=0.5]{figs/dgn-fra-ecdfbyd-w500.png}
&
\includegraphics[scale=0.5]{figs/dgn-fra-conv-w500.png}
\end{tabular}
}
\caption{Shows the plots for fixed random gating with $\mu=\frac{1}{2}$ and $\sigma=\sqrt{\frac{2}{w}}$. }
\label{fig:dgn-frg-gram-ecdf}
\end{figure}
