\section{Path-View: Capturing Feature Learning}\label{sec:pathgate}
\textbf{Notation:} We consider fully connected deep networks with $d$ layers, and $w$ hidden units per layer, which accepts an input $x\in\R^{d_{in}}$, and produces an output $\hat{y}_{\Theta}(x)\in \R$ where $\Theta\in\R^{d_{net}}$ ($d_{net}=w^2(d-2)+w(d_{in}+1)$). We denote by $\Theta(l,j,i)$ the weight connecting the $j^{th}$ hidden unit of layer $l-1$ to the $i^{th}$ hidden unit of layer $l\in[d]$.
\FloatBarrier
\begin{table}[h]
\centering
\begin{tabular}{|c|lll|}\hline
Input Layer &$z_{x,\Theta}(0)$ &$=$ &$x$ \\\hline
Pre-Activation & $q_{x,\Theta}(l,i)$& $=$ & ${\Theta(l,\cdot,i)}^\top z_{x,\Theta}(l-1), l\in[d-1],i\in[w]$\\\hline
Gating Values &$G_{x,\Theta}(l,i)$& $=$ & $\gamma(q_{x,\Theta}(l,i)), l\in[d-1],i\in[w]$ \\\hline
Hidden Layer &$z_{x,\Theta}(l,i)$ & $=$ & $\chi(q_{x,\Theta}(l,i))= q_{x,\Theta}(l,i)\cdot G_{x,\Theta}(l,i), l\in[d-1],i\in[w]$ \\\hline
Final Output & $\hat{y}_{\Theta}(x)$ & $=$ & ${\Theta(d)}^\top z_{x,\Theta}(d-1)$\\\hline
\end{tabular}
\caption{Here $\Theta(1)\in\R^{w\times d_{in}}, \Theta(l)\in\R^{w\times w},\forall l\in\{2,\ldots,d-1\}, \Theta(d)\in\R^{w\times 1}$. }
\label{tb:basic}
\end{table}
In what follows, we consider two kinds of gates namely, i) hard gates, taking values in $\{0,1\}$, and ii) soft gates, taking values in $(0,1)$. For a pre-activation input $q\in\R$, the  hard and soft gates are given by $\gamma_{r}(q)=\mathbbm{1}_{\{\q>0\}}$ and $\gamma_{sr}(q)=\frac{1}{1+\exp(-\beta q)}$, where $\beta>0$ is a positive constant. Using the hard and soft gates, we can specify the ReLU and `soft-ReLU' activations as $\chi_{r}(q)=q\cdot\gamma_r(q)$ and $\chi_{sr}(q)=q\cdot\gamma_{sr}$. Here, in $\gamma$ and $\chi$, the subscripts `$r$' and `$sr$' stand for `ReLU' and `soft-ReLU' respectively. As we shall see, replacing hard gates with soft gates enables us to differentiate the gating value.

\textbf{Paths:} A path starts from an input node, passes through exactly one weight (and one hidden node) in each layer and ends at the output node. We have a total of $P=d_{in}w^{(d-1)}$ paths. Let us say that an enumeration of the paths is given by $[P]=\{1,\ldots,P\}$. Let $\I_{l}\colon [P]\ra [w],l=0,\ldots,d-1$ provide the index of the hidden unit through which a path $p$ passes in layer $l$ (with the convention that $\I_d(p)=1,\forall p\in [P]$). The activity of a path $p$ for an input $x\in \R^{d_{in}}$ by $A_{\Theta}(x,p)\stackrel{def}{=}\Pi_{l=1}^{d-1} G_{x,\Theta}(l,\I_l(p))$.

\subsection{Neural Path Feature, Neural Path Value and Neural Tangent Feature}
The \emph{neural path feature} (NPF) of an input $x\in\R^{d_{in}}$ is given by 
\begin{align}\label{eq:npfdef}
\phi_{x,\Theta}\stackrel{def}=(x_s(\I_0(p))A_{\Theta}(x_s,p) ,p\in[P])\in\R^P,
\end{align}
where, for a path $p$, $\I_0(p)$ is the input node at which the path starts, and $A_{\Theta}(x_s,p)$ is its activity. \\
%By arranging the NPF of the $n$ input examples in a matrix $\Phi_t=(\phi_{x_s,\G_t},s\in[n])\in\R^{P\times n}$, we can express 
The \emph{neural path value} (NPV) if given by  
\begin{align}\label{eq:npvdef}
v_{\Theta}\stackrel{def}=(\Pi_{l=1}^d \Theta(l,\I_{l-1}(p),\I_l(p)),p\in[P])\in\R^P
\end{align}
The zeroth and first-order terms of a DNN can be written as:
\begin{align}
\label{eq:zero}\text{(Zeroth-Order)}\quad: \quad \hat{y}_{\Theta}(x)&=\ip{\phi_{x,\Theta},v_{\Theta}}=\sum_{p\in [P]}x(\I_0(p))A_{\Theta}(x,p)v_{\Theta}(p)\\
\label{eq:first}\text{(First-Order)}\quad: \quad \partial \hat{y}_{\Theta}(x)&=\underbrace{\ip{\phi_{x,\Theta},\partial v_{\Theta}}}_{\text{value derivative}}+ \underbrace{\ip{\partial\phi_{x,\Theta},v_{\Theta}}}_{\text{feature derivative}}\\&=\sum_{p\in [P]}x(\I_0(p)) A_{\Theta}(x,p) \partial v_{\Theta}(p)+\sum_{p\in [P]}x(\I_0(p)) \partial A_{\Theta}(x,p) v_{\Theta}(p)\nn
\end{align}
The \emph{neural tangent feature} (NTF) of an input $x\in\R^{d_{in}}$ is given by $\psi_{x,\Theta}=\left(\partial_{\theta}\hat{y}_{\Theta}(x),\theta\in\Theta\right)\in\R^{d_{net}}$. From \eqref{eq:first} we have $\psi_{x,\Theta}=\psi^v_{x,\Theta}+\psi^{\phi}_{x,\Theta}$, where $\psi^v_{x,\Theta}=\left(\ip{\phi_{x,\Theta}, \partial_{\theta} v_{\Theta}},\theta\in\Theta\right)\in\R^{d_{net}}$ and  $\psi^{\phi}_{x,\Theta}=\left(\ip{\partial_{\theta}\phi_{x,\Theta}, v_{\Theta}},\theta\in\Theta\right)\in\R^{d_{net}}$. 

\textbf{Two learning problems and two gradients:} \eqref{eq:zero} and \eqref{eq:first} reveal that there are two learning problems namely i) learning the NPVs for a fixed NPF and ii) learning the NPFs. Corresponding to the two learning problems are the two gradients namely i) the value gradient $\psi^v_{x,\Theta}$ which learns the NPVs, and ii) the feature gradient $\psi^{\phi}_{x,\Theta}$ responsible for learning the NPFs. 

\textbf{Feature Learning:}  The $\psi^{\phi}_{x,\Theta}$ term is new in our analysis,  in relation to which, the following critical points need to be noted:

$1.$ The gating values are modelled explicitly via the $G_{x,\Theta}$ terms. This enabled us to identify and define the NPFs in the first place. Further, each activation is written as a product of its pre-activation input and a gating $\gamma$.

$2.$ $\ip{\partial\phi_{x,\Theta},v_{\Theta}}=\sum_{p\in [P]}x(\I_0(p)) \partial A_{\Theta}(x,p)v_{\Theta}(p)$, and in the case of standard ReLU activations, the gating values $\gamma_r\in\{0,1\}$ and hence $\partial A_{\Theta}(x,p)=0$. However, the gating values and hence the path activities $A_{\Theta_t}(\cdot,\cdot)$ changes during training. This artefact arising due to the non-differentiability of ReLU is fixed by the \emph{soft-ReLU} activation which is differentiable. This soft-ReLU `trick' enables us to capture the terms related to neural path feature learning in our analysis. Thus a network with soft-ReLU gates $\gamma_{sr}$ is supposed to be viewed as an \emph{analytically relaxed} version of the network with standard ReLU activation.

$3.$ Prior works used a \emph{softplus} activation given by $\chi_{sp}(q)=\frac{1}{\beta}\log(1+\exp(\beta q)),\beta>0$ as a differentiable substitute for the ReLU activation. While the soft-ReLU and softplus activations are similar in both of them being differentiable, a critical difference lies in the derivatives. In the case of softplus, we have $\partial_{q}\chi_{sp}(q)=\frac{1}{1+\exp(-\beta q)}$, which is significant only for those gates which are \emph{on}, whereas, in the case of soft-ReLU, we have $\partial{\chi}_{sr}(q)=\partial{\chi}_{sp}(q)+\partial{\gamma}_{sr}(q)$, which is significant for those gates which are \emph{on} due to the $\partial{\chi}_{sp}(q)$ terms, and is also significant for those gates which are near $0$ due to the fact that $\partial_q{\gamma}_{sr}(q)=\frac{\beta q}{(1+\exp(\beta q))(1+\exp(-\beta q))}$. 

%The difference between hard and soft gates can be seen cartoons $(a)/(b)$ and $(c)$ in \Cref{fig:cartoon}, where negative/positive values lead to $0/1$ in the case of hard gating, and close to $0/1$ in the case of soft-gating.
\subsection{`Path-View': Conceptual Gains}\label{sec:cg}

\textbf{Active Sub-Networks and representational power:} For each input example, a set of gates are \emph{on} in each layer, and this gives rise to the sub-network of active paths for that input. This active sub-network can be said to hold the memory for a given input (see cartoon $(b)$ in \Cref{fig:cartoon}). The NPFs capture this notion of active sub-network formally.

\cite{ben} showed that DNNs can fit even random labels, and random pixels of standard datasets such as MNIST. However, we note that, for DNNs with ReLU, with no bias parameters, a dataset with $n=2$ points namely $(x,1)$ and $(x/2,-1)$ for some $x\in \R^{d_{in}}$ cannot be memorised. The reason is that the gating values are the same for both $x$ and $x/2$ (for that matter any positive scaling of $x$), and hence $\phi_{x/2,\Theta}= \phi_{x,\Theta}/2$, and thus it not possible to fit arbitrary values for $\hat{y}_t(x/2)$ and $\hat{y}_t(x)$, since $\hat{y}_t(x/2)= \hat{y}_t(x)/2$. Cartoons $(a)$ and $(b)$ in \Cref{fig:cartoon} illustrate this scale invariance for inputs $x=(1,-1,2)\in\R^3$ and $x/2=(0.5,-0.5,1)\in\R^3$:  when compared to $(a)$, pre-activation values and output of $(b)$ are scaled down by a factor $2$, and the gating values of $(a)$ and $(b)$ are identical.
\begin{figure}
\centering
%\begin{minipage}{0.8\columnwidth}
\resizebox{\columnwidth}{!}{
\begin{tabular}{ccc}
\includegraphics[scale=0.5]{figs/nn-subnet.png}
\includegraphics[scale=0.5]{figs/nnsoft.png}
\includegraphics[scale=0.5]{figs/nnconv.png}
%\includegraphics[scale=0.5]{figs/gradflow.png}
%\includegraphics[scale=0.5]{figs/nntwin-blck.png}
%\includegraphics[scale=0.5]{figs/nn.png}
%\includegraphics[scale=0.5]{figs/nn.png}
\end{tabular}
}
%\end{minipage}
%\begin{minipage}{0.18\columnwidth}
%\resizebox{\columnwidth}{!}{
%\includegraphics[scale=0.5]{figs/nnconv.png}
%}
%\end{minipage}
\caption{Cartoon illustration of usefulness of path-view and DGN framework.}
\label{fig:cartoon}
\end{figure}
%\begin{comment}
\textbf{Translation Invariance in ConvNet with pooling:} Consider a convolution network (ConvNet) using $d_{conv}$ layers of circular convolutions (see \Cref{sec:conv}) with filter size $k'<d_{in}$ and unit stride, and let $x^{d_{conv}}(i)$ be the output of the $i^{th}$ channel after either $\max$-/global-average-pooling (GAP). Looking at the third and fourth (from left) illustrations in \Cref{fig:cartoon}, it is easy to check the translation invariance property: each of the red, blue, green lines have the same path values due to weight sharing in the convolutional layers, this leads to a circular symmetry in the path values, due to which, a translation in the input will cause all the internal variable to translate, and final invariance results from the invariance of the $\max$/average operation.
%\end{comment}
%\footnote{Here, instead of zero-padding in the corners, we follow the convention that index $d_{in}+k$ will be interpreted as $k$, for $k>0$, and $-k$ will be interpreted as $d_{in}-k$.}


\textbf{Feature Gradient and Sensitive Sub-Network:} In a DNN with soft-ReLU, $\partial A_{\Theta}(x,p)=\sum_{l=1}^d \partial G_{x,\Theta}(l)\Pi_{l'\neq l}G_{x,\Theta}(l')$ is significant for those paths $p$ for which one of the $d-1$ gating values is close to $0.5$ (say such a gate is in layer $l$, and for such a gate $\partial G_{x,\Theta}$ is significant) and rest of the $d-2$ gates are close to $1$, so that $\Pi_{l'\neq l}G_{x,\Theta}(l')$ is significant. The set of paths for which $\partial A_{\Theta}(x,p)$ is significant, form the sensitive sub-network for that input. The sensitive paths are shown in cartoon $(c)$ of \Cref{fig:cartoon}. In the case of, standard ReLU, sensitive paths are those which contain one of the gates with pre-activation close to $0$, and rest of the $d-2$ gates are $1$.
\subsection{Kernels: Neural Path Kernel and Neural Tangent Kernel}\label{sec:ker}
The \emph{neural path kernel} (NPK) matrix is given by $H_{\Theta}(s,s')=\ip{\phi_{x,\Theta},\phi_{x_{s'},\Theta}}, s, s'\in[n]$. The NPK matrix has a special \emph{Hadamard} structure as given below.
\begin{lemma}\label{lm:npk}
Let $\tau_{\Theta}(s,s',l)\stackrel{def}=\sum_{i=1}^w G_{x_s,\Theta}(l,i)G_{x_{s'},\Theta}(l,i)$ be the number of activations that are ``on'' for both inputs $s,s'\in[n]$ in layer $l$. Then define $\Lambda_{\Theta}(s,s')\stackrel{def}=\Pi_{l=1}^{d-1}\tau_{\Theta}(s,s',l)$.
Let $\Sigma \in \R^{n\times n}$ be the $n\times n $ input Gram matrix with  $\Sigma(s,s')=\ip{x_s,x_{s'}},s,s'\in[n]$. It follows that $H_{\Theta}= \Sigma\odot\Lambda_{\Theta}$, where $\odot$ stands for the Hadamard product. 
\end{lemma}
\textbf{Normalisation of NPK:} In \Cref{lm:npk} above, $\Lambda_{\Theta}\in\R^{n\times n}$ is the correlation matrix of the active sub-networks (consisting of the activations that are \emph{on}) of different input pairs $s,s'\in[n]$. $H_{\Theta}$ contains $\Lambda_{\Theta}$ whose entries are of the order $w^{(d-1)}$, and hence is not normalised with respect to the width of the network.  However, in the main result \Cref{th:main}, we will see that $\Lambda_{\Theta}$ will be scaled by a factor $\frac{1}{w^{d-1}}$. 

\textbf{NPK for ConvNets:} In the case of a ConvNet  with GAP of depth $d=d_{conc}+d_{fc}$ ($d_{conv}$ is the depth of the convolutional part, and $d_{fc}$ is the depth of the fully-connected part), $\Lambda_{\Theta}=\Lambda^{conv}_{\Theta}\odot\Lambda^{fc}_{\Theta}$, of which $\Lambda^{conv}_{\Theta}$ is translation invariant (see \Cref{sec:conv}). 

The \emph{neural tangent kernel} matrix is given by $K_{\Theta}(s,s')=\ip{\psi_{x_s,\Theta},\psi_{x_{s'},\Theta}}, s, s'\in[n]$ and can be further decomposed as:
\begin{align}\label{eq:kerneldecomp}
K_{\Theta}(s,s')=K^v_{\Theta}(s,s')+K^{\phi}_{\Theta}(s,s')+K^{cross}_{\Theta}(s,s'),
\end{align}
where $K^v_{\Theta}(s,s')=\ip{\psi^v_{x_s,\Theta},\psi^v_{x_{s'},\Theta}}$ is the kernel corresponding to learning NPVs, $K^{\phi}_{\Theta}(s,s')=\ip{\psi^{\phi}_{x_s,\Theta},\psi^{\phi}_{x_{s'},\Theta}}$ is the kernel corresponding to learning NPFs, and $K^{cross}_{\Theta}(s,s')=\ip{\psi^v_{x_s,\Theta},\psi^{\phi}_{x_{s'},\Theta}}+\ip{\psi^{\phi}_{x_s,\Theta},\psi^{v}_{x_{s'},\Theta}}$ is a symmetric matrix of the cross-terms in the expansion of $K_{\Theta}(s,s')$.
\begin{comment}
 Let $\Psi^v_{\Theta}=\left(\psi^v_{x_s,\Theta},s\in[n]\right)$ and $\Psi^{\phi}_{\Theta}=\left(\psi^{\phi}_{x_s,\Theta},s\in[n]\right)$ be $d_{net}\times n$ matrices, and let $K^{v}_{\Theta}=(\Psi^{v}_{\Theta})^\top \Psi^{v}_{\Theta}$, $K^{\phi}_{\Theta}=(\Psi^{\phi}_{\Theta})^\top \Psi^{\phi}_{\Theta}$, $K^{cross}_{\Theta}=(\Psi^v_{\Theta})^\top \Psi^{\phi}_{\Theta}+(\Psi^{\phi}_{\Theta})^\top\Psi^{v}_{\Theta}$ be $n\times n$ matrices.

$\bullet$ $K^v_{\Theta}$ is the kernel corresponding to learning NPVs. Let $\Phi_{\Theta}=\left(\phi_{x_s,\Theta},s\in[n]\right)\in\R^{P\times d_{net}}$ be the NPF matrix, and define a $P\times d_{net}$ matrix with entries $\nabla_{\Theta}v_{\Theta}(p,\theta)=\partial_{\theta}v_{\Theta}$. Then, $K^v_{\Theta}=\Phi^\top_{\Theta}(\nabla_{\Theta}v_{\Theta})(\nabla_{\Theta}v_{\Theta})^\top \Phi^\top_{\Theta}$. Thus, as the NPF matrix $\Phi_{\Theta_t}$ changes with time $K^v_{\Theta_t}$ changes with time.\\
$\bullet$ $K^{\phi}_{\Theta}$ is the kernel corresponding to learning NPFs. Note that $K^{\phi}_{\Theta}=\sum_{\theta\in\Theta}\partial_{\theta}\Phi^\top_{\Theta}vv^\top\partial_{\theta}\Phi_{\Theta}$.
\end{comment}
\begin{comment}
\subsection{Gradient Dynamics with Feature Learning}

 The GD dynamics with NPF learning can be given by:
\FloatBarrier
\begin{table}[h]
\begin{tabular}{| l | lll |}\hline
Parameter Dynamics & $\dot{\Theta}_t$&$=$&$-\sum_{s=1}^n \psi_{x_s,\Theta_t}e_t(s)=\sum_{s=1}^n (\psi^v_{x_s,\Theta_t}+\psi^{\phi}_{x_s,\Theta_t})e_t(s)$\\\hline
NPF Dynamics& $\dot{\phi}_{x_s,t}(p)$&$=$&$x(\I_0(p))\sum_{\theta\in\Theta}\partial_{\theta}A_{\Theta_t}(x_s,p)\dot{\theta}_t,\forall p\in[P], s\in[n]$\\\hline
NPV Dynamics& $\dot{v}_t(p)$&$=$&$\sum_{\theta\in\Theta}\partial_{\theta}v_{\Theta_t}(p)\dot{\theta}_t,\forall p\in[P]$\\\hline
%Kernel Decompostion& $K_{\Theta_t}$&$=$&$K^v_{\Theta_t}+K^{\phi}_{\Theta_t}+K^{cross}_{\Theta_t}$\\\hline
Error Dynamics& $\dot{e}_t$&$=$&$-K_{\Theta_t}e_t$, where $K_{\Theta_t}=K^v_{\Theta_t}+K^{\phi}_{\Theta_t}+K^{cross}_{\Theta_t}$\ \\\hline
\end{tabular}
\caption{Gradient Dynamics with NPF learning}
\label{tb:graddyna} 
\end{table}

\textbf{Closing Research Gap I:} $K^v_{\Theta}(s,s')=\ip{\psi^v_{x_s,\Theta},\psi^v_{x_{s'},\Theta}}=\phi^\top_{x_s,\Theta}(\partial_{\theta}v_{\Theta})(\partial_{\theta}v_{\Theta})^\top\phi_{x_{s'},\Theta}$, because of which, $K^v_{\Theta_t}$ changes with time as the NPFs  $\phi_{x_s,\Theta_t},s\in[n]$ change with time.  Thus, using the `path-view' and the analytical relaxation due the soft-ReLU we have included the NPF learning in the gradient dynamics via the $K^{\phi}_{\Theta}$ term. We show in \Cref{sec:fixednpf,sec:generalisation} that NPF learning is critical for generalisation. We achieve this by first considering a fixed NPF regime in \Cref{sec:fixednpf}, wherein, we force $\psi^{\phi}_{\cdot,\Theta}=0$, and hence $K^{\phi}_{\Theta}=K^{cross}_{\Theta}=0$ and analyse $K_{\Theta}=K^v_{\Theta}$ for randomised initialisation. We then show via experiments in \Cref{sec:generalisation} that in standard DNNs with ReLU activations, NPFs are learnt, and such learning results in better generalisation performance than the fixed NPF case.
\end{comment}
\begin{comment}
The relationship between the $H_{\Theta}$ and $K^v_{\Theta}$ is given by the following lemma.
\begin{lemma}
Let $\rho_{max}(A)$ and $\rho_{min}(A)$ denote respectively the maximum and minimum eigenvalue of an $n\times n$ symmetric matrix $A$. Let $\nabla_{\Theta}v_{\Theta}(p,\theta)=\partial_{\theta}v_{\Theta}(p),\forall \theta\in \Theta, p\in[P]$ be a $P\times d_{net}$ matrix . It follows that $\rho_{min}\left(K^v_{\Theta}\right)\leq \rho_{min}\left(H_{\Theta}\right)\rho_{max}\left((\nabla_{\Theta}v_{\Theta} )(\nabla_{\Theta}v_{\Theta})^\top\right)$.
\end{lemma}
\end{comment}
\begin{comment}
\begin{align}\label{eq:graddyna} 
\begin{split}
\dot{\Theta}_t&=-\sum_{s=1}^n \psi_{x,t}e_t(s)\quad(\text{Parameter Dynamics})\\
\dot{\phi}_{x_s,t}(p)&=x(\I_0(p))\sum_{\theta\in\Theta}\partial_{\theta}A_t(x_s,p)\dot{\theta}_t,\forall p\in[P], s\in[n]\quad(\text{NPF Dynamics})\\
\dot{v}_t(p)&=\sum_{\theta\in\Theta}\partial_{\theta}v_t(p)\dot{\theta}_t,\forall p\in[P]\quad(\text{NPV Dynamicss})\\
K_{\Theta_t}&=K^v_{\Theta_t}+K^{\phi}_{\Theta_t}+K^{cross}_{\Theta_t}\quad(\text{Kernel Decomposition})\\
\dot{e}_t&=-K_{\Theta_t}e_t\quad(\text{Error Dynamics})
\end{split}
\end{align}
\end{comment}
\begin{comment}
 \textbf{Translation Invariance:} Consider a convolution network using $l$ layers of circular convolutions (see \Cref{sec:conv}) with filter size $k'<d_{in}$ and unit stride, and let $x^l(i)$ be the output of the $i^{th}$ channel after either $\max$-/global-average-pooling. Looking at the third and fourth (from left) illustrations in \Cref{fig:cartoon}, it is easy to check the translation invariance property: each of the red, blue, green lines have the same path values due to weight sharing in the convolutional layers, this leads to a circular symmetry in the path values, due to which, a translation in the input will cause all the internal variable to translate, and final invariance results from the invariance of the $\max$/average operation.\\
$3.$ \textbf{Active Sub-Network:} 
%\subsection{Information Flow: First-Order}

\textbf{First-Order Information Flow}\\
$1.$ \textbf{Value Gradient} $\ip{\phi_{x,\Theta},\partial v_\Theta}$ (see \eqref{eq:zero}) flows through the active sub-network. To see this, $\ip{\phi_{x,\Theta},\partial v_{\Theta}}=\sum_{p\in[P]}x(\I_0(p))A_{\Theta}(x,p)\partial v$, i.e., only paths $p$ with $A_{\Theta}(x,p)=1$ contribute to the summation and $\partial v_{\Theta}(p)$ is non-zero only for those weights through which the path $p$ passes.\\
$2.$ \textbf{Feature Gradient}  $\ip{\partial\phi_{x,\Theta},v_{\Theta}}=\sum_{p\in [P]}x(\I(p)) \partial(A_{\Theta}(x,p)) v_{\Theta}(p)$. Note that, in the case of ReLU activations the gating values are either $0$ or $1$, and hence $\partial(A_{\Theta}(x,p))=0$. However, the gating values and hence the path activities $A_{\Theta_t}(\cdot,\cdot)$ changes during training. This artefact arising due to non-differentiability can be fixed by considering a \emph{soft-ReLU} activation. 
%where, for a pre-activation input $q\in\R$ the output is given by $q\frac{1}{1+\exp(-\beta q)}$ as opposed to $q\mathbbm{1}_{\{q>0\}}$ of the ReLU. 
Soft-ReLU `trick' enables us to capture the terms related to feature learning in our analysis. The difference between hard and soft gates can be seen cartoons $(a)/(b)$ and $(c)$ in \Cref{fig:cartoon}, where negative/positive values lead to $0/1$ in the case of hard gating, and close to $0/1$ in the case of soft-gating.\\
$3.$ \textbf{Feature Gradient and Sensitive Sub-Network:} In a DNN with soft-ReLU, $\partial A_{\Theta}(x,p)=\sum_{l=1}^d \partial G_{x,\Theta}(l)\Pi_{l'\neq l}G_{x,\Theta}(l')$ is significant for those paths $p$ for which one of the $d-1$ gating values is close to $0.5$ (say such a gate is in layer $l$, and for such a gate $\partial G_{x,\Theta}$ is significant) and rest of the $d-2$ gates are close to $1$, so that $\Pi_{l'\neq l}G_{x,\Theta}(l')$ is significant. The set of paths for which $\partial A_{\Theta}(x,p)$ is significant, form the sensitive sub-network for that input. The sensitive paths are shown in cartoon $(c)$ of \Cref{fig:cartoon}. In the case of, standard ReLU, sensitive paths are those which contain one of the gates with pre-activation close to $0$, and rest of the $d-2$ gates are $1$.
\end{comment}
\begin{comment}
\subsection{Deep Gated Network (DGN)}
The NPFs are zeroth-order features stored solely in the gates of a DNN. In this paper, we separate out the NPFs from the NPVs. To this end, we introduce the deep gated network (DGN) framework, wherein, the output of a hidden unit is obtained as a product of its pre-activation and a gating value. A DGN has two network namely i) the gating network which holds the gating values and hence the NPF, and ii) a value network which holds the NPVs. %We denote a DNG by $\N(\Theta_t;\G(\Tg_t,\beta)$ or simply $\N(\Theta_t;\Tg_t,\beta)$
\begin{table}[h]
\begin{minipage}{0.5\columnwidth}
\resizebox{\columnwidth}{!}{
\begin{tabular}{|c|c|}\hline
Gating Network: $\G(\Tg_t,\beta)$\\\hline
$z_{x,\Tg_t}(0)=x$  \\\hline
$q_{x,\Tg_t}(l)={\Tg_t(l)}^\top z_{x,\Tg_t}(l-1)$ \\\hline
$z_{x,\Tg_t}(l)=q_{x,\Tg_t}(l)\odot G_{x,\Tg_t}(l)$ \\\hline
{$\begin{aligned}\beta >0: G_{x,\Tg_t}(l,i)&=\frac{1+\epsilon}{1+\exp(-\beta q_{x,\Tg_t}(l,i))} \\ \beta=\infty: G_{x,\Tg_t}(l,i)&=\mathbbm{1}_{\{q_{x,\Tg_t}(l,i)>0\}}\end{aligned}$}\\\hline 
\end{tabular}
}
\end{minipage}
\begin{minipage}{0.5\columnwidth}
\resizebox{\columnwidth}{!}{
\begin{tabular}{|l|l|}\hline
\multicolumn{2}{|c|}{Value Network: $\N(\Theta_t;\G_t)$}\\\hline 
Input layer & $z_{x,\Theta_t}(0)=x$ \\\hline
Pre-activation & $q_{x,\Theta_t}(l)={\Theta_t(l)}^\top z_{x,\Theta_t}(l-1)$\\\hline
Layer output & $z_{x,\Theta_t}(l)=q_{x,\Theta_t}(l)\odot G_{x,t}(l)$ \\\hline
Final output & $\hat{y}_t(x)={\Theta_t(d)}^\top z_{x,\Theta_t}(d-1)$\\\hline
Gating Values& $\begin{aligned}\G_t\stackrel{def}=\{G_{x_{s},t}(l,i), \forall s\in[n],\\l\in[d-1],i\in[w]\}\end{aligned}$\\\hline
\end{tabular}
}
\end{minipage}
\caption{$q(l),z(l)$ and $G(l)$ are $w$-dimensional quantities}
\label{tb:dgn}
\end{table}
\end{comment}
%\section{Feature Learning}
In this section, we take a re-look at the NTK and NTK matrices, and explicitly write down the terms related to feature learning.\hfill\\
\textbf{ReLU artefact or NTF vs NPF:} Recall that the neural tangent feature is given by $\psi_{x,t}=\phi^\top_{x,t} {\partial} v_t + {\partial} \phi^\top_{x,t} v_t$. We note that, in the case when $A_t(x,p)\in\{0,1\}$ (such as in DNN with ReLU activations)  $\partial_{\phi}=0$, and hence is not accounted for in the SGD update as well as in the trajectory analysis. However, due to the SGD update, the gating value changes during training, and consequently the activations, the NPF, the NPK change during training as seen in the experiments in \Cref{sec:generalisation}. We side step this artefact by studying network with soft-gates: given a pre-activation input, a hard gate (uch as ReLU) uses $\mathbbm{1}_{\{q>0\}}$, while a soft-gate uses a soft-max sigmoidal function $\chi_\epsilon(q)=\frac{1+\epsilon}{1+\exp(-\beta q)}$. We consider sofr-ReLU networks denoted by $\N_{SR}(\Theta_t)$, and soft-GaLU networks denoted by $\N_{G}(\Theta_t, \G(\N_{SR}(\Tg_t)))$. We checked the performance of soft-gates by training, a soft-GaLU network and a soft-ReLU network, and the fact that adaptable gates generalise better than frozen gates continued to hold (see \Cref{fig:gen} which shows the performance of all the $4$ networks namely ReLU, frozen-GaLU, soft-GaLU and soft-ReLU).\hfill\\
\textbf{NTK and feature learning:} In the case of  soft-GaLU networks, since there are two set of parameters (total $2d_{net}$), we have the neural tangent feature to be $\psi_{x,t}=[\psi^v_{x,t},\psi^a_{x,t}]=[\partial_{\Theta} \hat{y}_t, \partial_{\Tg} \hat{y}_t]\in \R^{2d_{net}}$. Thus the NTK matrix is given by $K_t=K^w_t+K^a_t$,  where $K^w_t={\Psi^v_t}^\top \Psi^v_t$, and $K^a_t={\Psi^a_t}^\top \Psi^a_t$, where $\Psi^v_t=[\psi^v_{x_s,t},s\in[n]]$ and $\Psi^v_t=[\psi^a_{x_s,t},s\in[n]]$ are the NTF matices. In the case of  soft-ReLU networks, i.e, $\N_{SR}(\Theta_t)$  it is easy to check that $K_t={K^v_t}+{K^a_t}+{\Psi^v_t}^\top {\Psi^a_t}+{\Psi^a_t}^\top {\Psi^v_t}$.
\begin{definition}\label{def:delta}
For a soft-GaLU networks, using any $i\in[d_{in}]$, define $\delta_t(s,s')\stackrel{def}= \underset{{p\rsa i}}{\sum} \sum_{\tg\in\Tg}\frac{\partial A_{t}(x_s,p)}{\partial \tg} \frac{\partial A_{t}(x_{s'},p)}{\partial \tg}$.
\end{definition}
\begin{lemma} Under \Cref{assmp:main}, in soft-GaLU networks we have: (i) $\E{K_0}=\E{K^w_0}+\E{K^a_0}$, 
 (ii) $\E{K^w_0}=\sigma^{2(d-1)} (x^\top x)\odot \lambda_0$,  (iii) $\E{K^a_0}=\sigma^{2d}  (x^\top x)\odot \delta_0$
\end{lemma}
\textbf{Discussion}: \hfill\\
$1.$ Note that in \Cref{def:delta}, $\delta_t$ contains $\partial A_t(x,p)$ terms as opposed to $A_t(x,p)$ term in $\lambda_t$ defined in \Cref{def:lambda}. For a path $p$, if a path contains gates that are very close to either $0$ or $1$, then $\partial A_t(x,p)$ will be very small. Since $\frac{\partial A_{t}(x_s,p)}{\partial \tg}= \sum_{l=1}^{d-1} \Big(\frac{\partial G_{x_s,\Tg_t}(l,\I_l(p))}{\partial \tg} \Big)\Big(\Pi_{l'\neq l} G_{x_s,\Tg_t}(l',I_{l'}(p))\Big)$, for a path $p$ containing $d-1$ gates close to $1$, and only one gate whose pre-activation input is close to $0$ will have a significant $\partial A_t(x,p)$. \hfill\\
$2.$ Let us take the case of classification with cross-entropy loss. Now, say we have trained till some $T$ epochs with good classification (say close to $100\%$) accuracy. Considering the fact that $\hat{y}_T=\Phi_{\Tg_T}v_{\Theta_T}$, the gradient with respect to $\Tg$ will change the NPF matrix $\Phi_{\Tg_T}$ in such a manner to reduce the loss, i.e., increase the margin of each of the classified examples, subject to the `well-conditioned' ness of $K^a_T$. While this scenario is idealistic, in practice, both $\Tg_t$ as well as $\Theta_t$ change, and we need to understand how the joint optimisation and feature learning happens.\hfill\\
\begin{figure}[h]
\resizebox{\textwidth}{!}{
\begin{tabular}{cccc}
\includegraphics[scale=0.2]{figs/simple-original.png}
&
\includegraphics[scale=0.2]{figs/simple-feat.png}
&
\includegraphics[scale=0.2]{figs/adapt-original.png}
&
\includegraphics[scale=0.2]{figs/adapt-feat.png}
\end{tabular}
}
\caption{From the left: First two plots show the performance on the training set with original ($x_s$) and internal features ($\phi_{x_s,t}$) in the case of frozen-gates. The third and fourth plots show performance on the training set with original ($x_s$) and internal feature ($\phi_{x_s,t}$)  in the case of adaptable gates.}
\label{fig:feat}
\end{figure}

\textbf{Toy Experiment:} We check the performance of frozen-gates versus adaptable gates in a toy experiment. The dataset is $(x_s,y_s)_{s=1}^{n},n=1000$, wherein, for $s=,1,\ldots,500$, $x_s(1)\stackrel{iid}\sim U[0.1,1]$, and $x_s(2)\stackrel{iid}\sim U[-100,100]$, and $y_s=1$, and for $s=501,\ldots,1000$, $x_s(1)\stackrel{iid}\sim U[-0.1,-1]$, and $x_s(2)\stackrel{iid}\sim U[-100,100]$, and $y_s=-1$. We transformed the input $x\in \R^2$ into a feature vector $\phi_{x,t}=(x(1)G_t(1),x(2)G_t(2) )\in \R^2$, where $G_t(i)=\frac{1}{1+\exp(-\tg(i))},i=1,2$. The predicted output is $\hat{y}_{t}(x_s)=\phi^\top_t\Theta_t$, where $\Theta_t\in \R^2$. We use the loss function $L_t=\frac{1}{n}\sum_{s=1}^n\frac{1}{1+exp(-y_s\hat{y}_t(x_s)}$, and trained the model for $T=1000$ epochs, for the following two cases: i) frozen-gates, wherein, we set $\Tg_0=(10,10), \Tg_t=\Tg_0,\forall t\geq 0$, so that the input features are not transformed, and used gradient descent with stepsize $0.1$ to train $\Theta$ ii) adaptable gates, wherein, we performed gradient descent with stepsize equal to $0.1$ to train both $\Tg$ and $\Theta$. The results are shown in \Cref{fig:feat}, notice that the right most plot shows that in the case when the gates are adapting, they learn to suppress the second co-ordinate (the scale of $\phi_{x_s,T}$ is from $-15$ to $15$ as opposed to $-100$ to $100$ in $x_s$). 

\begin{comment}
\subsection{Kernels}\label{sec:kernels}
\subsubsection{Neural Path Kernel}
The NPF matrix is given by $\Phi_{\Theta}=(\phi_{x_s,\Theta},s\in[n])\in\R^{P\times n}$. The NPK matrix is then given by $H_{\Theta}=\Phi^\top_{\Theta}\Phi_{\Theta}$. The NPK matrix has a special structure as given in \Cref{lm:npk} below.
\begin{lemma}\label{lm:npk}
Let $p\rsa i$ denote the fact that path $p$ passes through input node $i\in[d_{in}]$, and let $\Lambda_{\Theta}(s,s')\stackrel{def}{=}\sum_{p\rsa i} A_{\Theta}(x_s,p) A_{\Theta}(x_{s'},p)$, $\forall s,s'\in[n]$, any $i\in [d_{in}]$. It follows that $H_{\Theta}= \Sigma\odot\Lambda_{\Theta}$, where $\odot$ stands for the Hadamard product, and $\Sigma \in \R^{n\times n}$ is the input Gram matrix.
\end{lemma}
In the \Cref{lm:npk} above, $\Lambda_{\Theta}\in\R^{n\times n}$ is the correlation matrix of the active sub-networks of different input pairs $s,s'\in[n]$. Note that the definition of $\Lambda$ is not dependent on the choice of input node $i$, because, the terms inside the summation depend only on the path followed from the first layer onwards and excludes the input node.\\
\subsubsection{Neural Tangent Kernel}
Using the `path-view', we obtain a different decomposition for the NTK matrix $K{\Theta}$, and compare our expression to the previously obtained expression in \Cref{eq:ntkold}. To this end, let us by define $d_{net}\times n$ matrices $\Psi^v_t$, and $\Psi^{\phi}_t$, whose entries are given by $\Psi^v_t(\theta,s)=\ip{\phi_{x_s,\Theta},\partial_{\theta}v_{\Theta}}$, and  $\Psi^{\phi}_t(\theta,s)=\ip{\partial_{\theta}\phi_{x_s,\Theta}, v_{\Theta}}$. Now, we have the following NTK decomposition (for a soft-ReLU DNN):
\begin{align}\label{eq:ntknew}
K_{\Theta}=K^v_{\Theta}+K^{\phi}_{\Theta}+(\Psi^v_\Theta)^\top \Psi^{\phi}_{\Theta} +(\Psi^{\phi}_\Theta)^\top \Psi^{v}_{\Theta},\,\text{where},
\end{align}
$K^v_{\Theta}$ is the NTK of path values due to the value gradient, $K^{\phi}_{\Theta}$ is the NTK of the features due to the feature gradient, and $(\Psi^v_\Theta)^\top \Psi^{\phi}_{\Theta} +(\Psi^{\phi}_\Theta)^\top \Psi^{v}_{\Theta}$, is a symmetric matrix which is the cross-term obtained as an interaction of the value and the feature gradients. %We now list the salient points about the decomposition in \eqref{eq:ntknew}. The recursive relationship in \eqref{eq:ntkold} is due to the fact that both zeroth-order (forward propagation) and first-order (backward propagation) information use a layer after layer expression for information flow. Also, \eqref{eq:ntkold} derives a limiting matrix ($w\ra\infty$) that occurs at a random $\Theta_0$. On the contrary, \eqref{eq:ntknew} holds for any parameter setting and for finite $w$ and $d$.\\
%\subsection{Neural Path Kernel and Optimisation}
%\subsection{Gate Dynamics and Feature Learning}
%$\bullet$ $(\Psi^v_\Theta)^\top \Psi^{\phi}_{\Theta} +(\Psi^{\phi}_\Theta)^\top \Psi^{v}_{\Theta}$, is a symmetric matrix which is the cross-term obtained as an interaction of the value and the feature gradients.
\section{Neural Tangent Kernel in Prior Works}
The NTK matrix has also been used to design pure-kernel based methods \cite{arora2019exact} and provide generalisation bounds \cite{cao2019generalization}.\\
\textbf{NTK in prior works:} Before we connect the NPK and the NTK using the `path-view', we will first look at prior works based on NTK. Prior works have used the following recursive definition or its variants, wherein, for layers $l=1,\ldots, d-1$, we define matrices:
\begin{align}\label{eq:ntkold}
&\tilde{K}^{(1)}(s,s')=\Sigma^{(1)}(s,s')=\Sigma(s,s'), M^{(l)}_{ss'}=\left[\begin{matrix}\Sigma^{(l)}(s,s) & \Sigma^{(l)}(s,s')\\ \Sigma^{(l)}(s',s) & \Sigma^{(l)}(s',s')\end{matrix}\right]\in \R^2,\\
&\Sigma^{(l+1)}(s,s')= 2\cdot\mathbb{E}_{(q,q')\sim N(0,M_{ss'}^{(l)})} \left[\chi(q)\chi(q')\right], \dot{\Sigma}^{(l+1)}(s,s')= 2\cdot\mathbb{E}_{(q,q')\sim N(0,M_{ss'}^{(l)})}\left[\dot{\chi}(q)\dot{\chi}(q')\right],\nn\\
&\tilde{K}^{(l+1)}=\tilde{K}^{(l)}\odot \dot{\Sigma}^{(l+1)}+\Sigma^{(l+1)},$K^{(d)}=\left(\tilde{K}^{(d)}+\Sigma^{(d)}\right)/2$\nn
\end{align}
where $s,s'\in[n]$ are two input examples in the dataset, $\Sigma$ is the data Gram matrix, $\dot{\chi}$ stands for the derivative of the activation function with respect to the pre-activation input, $N(0,M)$ stands for the mean-zero Gaussian distribution with co-variance matrix $M$. Prior works show that at randomised initialisation $\Theta_0$, $K_{\Theta_0}\ra K^{(d)}$ as $w\ra \infty$.
$K^{\phi}_{\Theta}$ is the NTK of the features, and describes the dynamics of feature learning due to the flow of the feature gradients, which, for each input example, decides which of the sensitive yet inactive paths should become active, and which of the sensitive yet active paths should become inactive. The NTK of features can be expanded as $K^{\phi}_{\Theta}=\sum_{\theta \in \Theta}\partial_{\theta} \Phi^\top_{\theta}v_{\Theta}v^\top_{\Theta}\partial_{\theta}\Phi_{\Theta}$. The significance here is that $\partial \Phi_{\Theta}$ terms contain, $\partial A_{\Theta}(\cdot,\cdot)$, which in turn contains $\partial G$ terms, which in turn contains $\dot{G}$ term which is the derivative of the gate with respect to its pre-activation input. NTK expression in prior works \eqref{eq:ntkold} only capture the $\chi$ and $\dot{\chi}$ terms. As shown in \Cref{fig:actgate}, only $\dot{G}$ (right-most plot) captures the gates that are in the verge of transition (either from $0$ to $1$ or from $1$ to $0$).
%$\bullet$ $(\Psi^v_\Theta)^\top \Psi^{\phi}_{\Theta} +(\Psi^{\phi}_\Theta)^\top \Psi^{v}_{\Theta}$, is a symmetric matrix which is the cross-term obtained as an interaction of the value and the feature gradients.
\FloatBarrier
\begin{figure}[h]
%\begin{minipage}{0.78\columnwidth}
\resizebox{\columnwidth}{!}{
\begin{tabular}{ccc}
\includegraphics[scale=0.4]{figs/act.png}
\includegraphics[scale=0.4]{figs/der-act.png}
\includegraphics[scale=0.4]{figs/gate.png}
\includegraphics[scale=0.4]{figs/der-gate.png}
%\includegraphics[scale=0.5]{figs/nntwin-blck.png}
%\includegraphics[scale=0.5]{figs/nn.png}
%\includegraphics[scale=0.5]{figs/nn.png}
\end{tabular}
}
%\end{minipage}
%\begin{minipage}{0.18\columnwidth}
%\resizebox{\columnwidth}{!}{
%\includegraphics[scale=0.5]{figs/nnconv.png}
%}
%\end{minipage}
\caption{Activations and Gates}
\label{fig:actgate}
\end{figure}
\end{comment}